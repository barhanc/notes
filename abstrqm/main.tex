\documentclass{myclass}
\usepackage[polish]{babel}

% TODO

% * Add solved examples precession (constant H), NMR (H = -hg/2 B s) (Rabi), general (?)
%
% * Composite systems made of 2-state systems (2 x 2 & later n x 2 or maybe n x 2 immediately(?))
% * Composite systems: entanglement, density operator, partial trace, no-cloning theorem,
%   decoherence, Bell's theory
%
% * later (after extensively studying composite systems, entanglement, Bell's theory) begin
%   description of quantum registers, gates, idea of computation using those devices (Deutsch,
%   Grover, QFTrans, Shor), applications to number theory, connection with Church-Turing hypothesis.
%   Examples
%
% * possible physical realizations of quantum computing (gates + adiabatic/topological(?)), I'll
%   probably focus on Josephson junctions and charge qubits as they seem most elegant (maybe
%   NMR/qdots as well)
%
% * A short overview of classical and quantum information theory (I don't want to go very deep there
%   at the moment): Shannon/Von Neumann entropy, quantum information transfer/cryptography
%   ------


\author{Bartosz Hanc}
% My notes regarding abstract foundations of quantum theory and introduction to quantum computing

\begin{document}

\section{ABSTRAKCYJNA TEORIA KWANTÓW}

\subsection{Elementy teorii przestrzeni Hilberta}

Przez \(\mathbb{V} := (V,\mathbb{C},+,\cdot)\) będziemy oznaczać przestrzeń wektorową nad ciałem
liczb zespolonych.

\begin{definition}
Odwzorowanie \(d: V \times V \mapsto \mathbb{R}\) będziemy nazywać \textit{metryką} w zbiorze \(V
\neq \emptyset\) iff
\begin{itemize}

\item \(\forall u,v \in V : d(u,v) \geq 0\), przy czym równość zachodzi iff \(u = v\)
(\textit{nieujemność})

\item \(\forall u,v \in V : d(u,v) = d(v,u)\) (\textit{symetria})

\item \(\forall u,v,w \in V : d(u,v) + d(v,w) \geq d(u,w)\) (\textit{nierówność trójkąta})

\end{itemize}
Parę \((V,d(\cdot,\cdot))\) będziemy nazywać \textit{przestrzenią metryczną}.
\end{definition}

\begin{definition}
Niech \((V,d)\) będzie przestrzenią metryczną. Mówimy, iż dany ciąg \((u_n)\) elementów zbioru \(V\)
jest zbieżny do \(g\in V\) tj. \(u_n \to g\) przy \(n \to \infty\) iff \(d(u_n,g) \to 0\) przy \(n
\to \infty\).   
\end{definition}

\begin{definition}
Ciąg \((u_n)\) elementów \(u_n \in V\) będziemy nazywać \textit{ciągiem Cauchy'ego} w przestrzeni
metrycznej \((V,d(\cdot,\cdot))\) iff spełnia on kryterium Cauchy'ego tj.
\begin{equation*}
    \forall \epsilon > 0 : \exists N : \forall n,m > N : d(u_n,u_m) < \epsilon\,.
\end{equation*}  
\end{definition}

\begin{theorem}
Każdy ciąg zbieżny w przestrzeni metrycznej \((V,d)\) jest ciągiem Cauchy'ego w tej przestrzeni.
\end{theorem}

\begin{definition}
Przestrzeń metryczną \((V,d(\cdot,\cdot))\) nazwiemy \textit{zupełną} iff każdy ciąg Cauchy'ego
\((u_n)\) elementów \(u_n \in V\) jest zbieżny do granicy \(g \in V\).
\end{definition}

\begin{definition}
Niech \(\mathbb{V}\) będzie przestrzenią wektorową. Odwzorowanie \(\braket{\cdot} : V \times V
\mapsto \mathbb{C}\) nazwiemy \textit{iloczynem wewnętrznym} wektorów iff
\begin{itemize}
    
    \item \(\forall u,v \in V : \braket{u}{v}^* = \braket{v}{u}\)
    
    \item \(\forall u,v_1,v_2 \in V : \forall \alpha, \beta \in \mathbb{C} : \braket{u}{\alpha v_1 +
    \beta v_2} = \alpha \braket{u}{v_1} + \beta \braket{u}{v_2}\)

    \item \(\forall u \in V : \braket{u} \geq 0\), przy czym równość zachodzi iff \(u =
    \mathsf{0}\). Zauważmy tutaj, iż z pierwszego aksjomatu \(\braket{u} \in \mathbb{R}\), gdyż
    \(\braket{u} = \braket{u}^* \implies \Im{\braket{u}} = 0\). 

\end{itemize}
Parę \((\mathbb{V},\braket{\cdot})\) będziemy nazywać \textit{przestrzenią unitarną}.
\end{definition}

\begin{theorem}
Każda przestrzeń unitarna jest metryczna z metryką indukowaną przez iloczyn wewnętrzny \(d(u,v) :=
\norm{u - v} = \sqrt{\braket{u-v}}\).
\end{theorem}

\begin{theorem}[\textit{Nierówność Cauchy'ego--Schwarza}]
Niech \((\mathbb{V},\braket{\cdot})\) -- przestrzeń unitarna. Wówczas
\begin{equation*}
    \forall u,v \in V : |\braket{u}{v}|^2 \leq \braket{u}\braket{v}\,.
\end{equation*}  
\end{theorem}

\begin{definition}
Przeliczalny zbiór wektorów \(\{v_1,...,v_n\}\) nazwiemy \textit{ortogonalnym} iff 
\begin{equation*}
    \forall i\neq j; i,j\in\{1,...,n\} : \braket{v_i}{v_j} = 0\,.
\end{equation*}
Ten sam zbiór wektorów nazwiemy \textit{ortonormalnym} iff 
\begin{equation*}
    \forall i,j\in\{1,...,n\} : \braket{v_i}{v_j} = \delta_{ij}\,,
\end{equation*}
gdzie \(\delta_{ij}\) jest deltą Kroneckera.
\end{definition}

\begin{theorem}
Każda przestrzeń unitarna \((\mathbb{V}, \braket{\cdot})\) posiada bazę ortonormalną, tj. bazę,
której wektory bazowe tworzą zbiór ortonormalny.
\end{theorem}

\begin{definition}
\textit{Przestrzenią Hilberta} \(\mathscr{H}=(\mathbb{V},\braket{\cdot})\) nazwiemy zupełną
przestrzeń unitarną.
\end{definition}

\begin{definition}
Niech \(\mathscr{H} = (\mathbb{V},\braket{\cdot})\) będzie przestrzenią Hilberta. Odwzorowanie
liniowe \(F:V\mapsto\mathbb{C}\) nazwiemy \textit{funkcjonałem liniowym} w przestrzeni
\(\mathscr{H}\).
\end{definition}

\begin{theorem}
Niech \(V^*\) oznacza zbiór wszystkich funkcjonałów liniowych \(F:V\mapsto\mathbb{C}\). Wówczas
\(\mathbb{V}^*:=(V^*,\mathbb{C},+,\cdot)\), gdzie
\begin{itemize}
    \item \(\forall F_1,F_2 \in V^* : \forall v \in V : (F_1+F_2)(v) = F_1(v) + F_2(v)\)

    \item \(\forall F \in V^* : \forall \alpha \in \mathbb{C} : \forall v \in V : (\alpha \cdot
    F)(v) = \alpha F(v)\)
\end{itemize}
jest przestrzenią wektorową, którą nazywamy \textit{przestrzenią dualną}.
\end{theorem}

\begin{theorem}[\textit{Riesza}]
Niech \(\mathscr{H}=(\mathbb{V},\braket{\cdot})\) będzie przestrzenią Hilberta, a \(\mathbb{V}^*\)
jej przestrzenią dualną. Wówczas istnieje wzajemnie jednoznaczne odwzorowanie wektorów \(v \in V\)
na funkcjonały liniowe \(F \in V^*\). Dodatkowo dla każdego funkcjonału \(F\) istnieje dokładnie
jeden wektor \(u \in V\) taki, że
\begin{equation*}
    \forall v \in V : F(v) = \braket{u}{v}\,.
\end{equation*}
\end{theorem}

\begin{definition}
\textit{Iloczynem Kroneckera} macierzy \(\oper{A} = [a_{ij}]_{n\times m}\) i \(\oper{B} =
[b_{ij}]_{n'\times m'}\) nazywamy macierz wymiaru \(nn'\times mm'\) postaci
\begin{equation*}
    \begin{split}
        \oper{A} \otimes \oper{B} &= \mqty[a_{11} & \cdots & a_{1m} \\ \vdots & \ddots & \vdots \\ a_{n1} & \cdots & a_{nm}] \otimes \mqty[b_{11} & \cdots & b_{1m'} \\ \vdots & \ddots & \vdots \\ b_{n'1} & \cdots & b_{n'm'}] = \mqty[a_{11}\oper{B} & \cdots & a_{1m}\oper{B} \\ \vdots & \ddots & \vdots \\ a_{n1}\oper{B} & \cdots & a_{nm}\oper{B}]
    \end{split}
\end{equation*}
\end{definition}

\begin{definition}
\textit{Iloczynem tensorowym} przestrzeni Hilberta \(\mathscr{H}_1\) i \(\mathscr{H}_2\) o bazach
ortonoromalnych odpowiednio \(\{\phi_i^{(1)}\}\) i \(\{\phi_j^{(2)}\}\) nazywamy przestrzeń Hilberta
\(\mathscr{H} = \mathscr{H}_1 \otimes \mathscr{H}_2\) taką, że:
\begin{itemize}
    
    \item Jej bazą ortonormalną jest zbiór \(\{\phi_i^{(1)} \otimes \phi_j^{(2)}\}\).

    \item Iloczyn wewnętrzny w przestrzeni \(\mathscr{H}_1 \otimes \mathscr{H}_2\) jest zdefiniowany
    jako
    \begin{equation*}
        \braket{\chi_1 \otimes \chi_2}{\psi_1 \otimes \psi_2} := \braket{\chi_1}{\psi_1}_1 \cdot \braket{\chi_2}{\psi_2}_2\,,
    \end{equation*}
    gdzie \(\chi_i,\psi_i \in \mathscr{H}_i\) to pewne wektory, a \(\braket{\cdot}_i\) to iloczyn
    wewnętrzny w \(\mathscr{H}_i\).
\end{itemize}
\end{definition}

\subsection{Notacja Diraca}

Niech \(\mathscr{H}\) będzie przestrzenią Hilberta. Wprowadzimy teraz kompaktową notacją wektorów i
funkcjonałów liniowych wymyśloną przez P.A.M. Diraca. Aby uprościć zapis, będziemy mówili o
wektorach należących do przestrzeni \(\mathscr{H}\) (używając nawet symbolu należenia \(\in
\mathscr{H}\)), mając oczywiście formalnie na myśli wektory należące do zbioru \(V\).

Wektory należące do \(\mathscr{H}\) będziemy oznaczać jako
\begin{equation*}
    \ket{\psi},\, \ket{\phi},\, ...\,,
\end{equation*}
przy czym \(\ket{\cdot}\) to tzw. \textit{ket} i formalnie jest to odwzorowanie \(\ket{\cdot} :
\mathsf{S} \mapsto V\), gdzie \(\mathsf{S}\) jest zbiorem znaków, których używamy do oznaczenia
konkretnych wektorów ze zbioru \(V\). Nie będziemy jednak przestrzegali tego formalnego znaczenia,
utożsamiając dla wygody również  sam symbol z wektorem.

Funkcjonały liniowe należące do przestrzeni dualnej będziemy oznaczać jako
\begin{equation*}
    \bra{\psi},\, \bra{\phi},\, ...\,,
\end{equation*}
przy czym \(\bra{\cdot}\) to tzw. \textit{bra} i formalnie jest to odwzorowanie \(\bra{\cdot} :
\mathsf{S}^* \mapsto V^*\), gdzie \(\mathsf{S}^*\) jest zbiorem znaków, których używamy do
oznaczenia konkretnych funkcjonałów ze zbioru \(V^*\). Ponieważ z tw. Riesza istnieje wzajemnie
jednoznaczne odwzorowanie funkcjonałów liniowych na wektory, więc możemy utożsamić \(\mathsf{S}^* =
\mathsf{S}\).

\subsection{Skończenie wymiarowa przestrzeń Hilberta nad ciałem liczb zespolonych}

Rozważymy teraz konstrukcję skończenie wymiarowej przestrzeni Hilberta złożonej ze skończenie
wymiarowej przestrzeni wektorowej \(\mathbb{V} = (\mathbb{C}^n, \mathbb{C}, +, \cdot)\), której
elementy będziemy w danej bazie \textit{ortonormalnej} \(\{\phi_1, \ldots, \phi_n\}\) zapisywać jako
\begin{equation*}
    \mathbb{V} \ni \ket{\psi} = \sum_{i=1}^n a_i \ket{\phi_i} = \mqty[a_1 \\ \vdots \\ a_n]\,,
\end{equation*}
gdzie \(a_i = \braket{\phi_i}{\psi}\in\mathbb{C}\) oraz iloczynu wewnętrznego zdefiniowanego jako
\begin{equation*}
    \braket{ \sum_{i=1}^n a_i \ket{\phi_i}}{ \sum_{i=1}^n b_i \ket{\phi_i}} := \sum_{i=1}^n a_{i}^*b_i\,.
\end{equation*}
Powstała w ten sposób skończenie wymiarowa przestrzeń unitarna jest trywialnie zupełna, a zatem
skonstruowaliśmy skończenie wymiarową przestrzeń Hilberta. Wektor w tej przestrzeni możemy utożsamić
(poprzez iloczyn wewnętrzny) z macierzą kolumnową jego współrzędnych w danej bazie ortonormalnej.
Jasne jest również czym jest funkcjonał liniowy stowarzyszony z danym wektorem
\begin{equation*}
    \bra{\psi} = \mqty[a_1 \\ \vdots \\ a_n]^\dagger = \mqty[a_1^* & \ldots & a_n^*]\,,
\end{equation*}
gdzie \(\dagger\) oznacza \textit{sprzężenie hermitowskie} macierzy, tj. sprzężoną macierz
transponowaną. W dalszej części skupimy się głównie na skończenie wymiarowych przestrzeniach
Hilberta \(((\mathbb{C}^n, \mathbb{C}, +, \cdot),\braket{\cdot})\), gdyż stanowią one podstawę opisu
teorii obliczeń kwantowych i kwantowej teorii informacji. Należy zdawać sobie jednak sprawę, iż
stanowi to duże uproszczenie w stosunku do wymagań pełnoprawnych teorii fizycznych (mechanika
falowa, kwantowa teoria pola), w których niezbędna jest teoria nieskończenie wymiarowych przestrzeni
Hilberta.

\subsection{Elementy teorii operatorów liniowych}

\begin{definition}
\textit{Operatorem liniowym} \(\oper{A}\) w przestrzeni \(\mathscr{H} = (\mathbb{V},
\braket{\cdot})\) nazywamy odwzorowanie liniowe 
\begin{equation*}
    \oper{A} : D(\oper{A}) \mapsto D(\oper{A})\,,
\end{equation*}
gdzie \(D(\oper{A})\) jest pewną podprzestrzenią wektorową przestrzeni \(\mathbb{V}\). Dodatkowo
zakładamy, iż dziedziny operatorów są gęste, to znaczy ich domknięcia są równe \(\mathscr{H}\).
\end{definition}

Zgodnie z wcześniejszymi komentarzami nie będziemy wnikali w subtelne problemy wynikające z faktu,
iż w nieskończenie wymiarowej przestrzeni Hilberta pojęcie operatora liniowego jest nieodłącznie
związane z pojęciem dziedziny tego operatora, który w ogólności nie jest określony na całej
przestrzeni Hilberta, a tylko na pewnym jej podzbiorze. Komplikacje te nie występują w skończenie
wymiarowych przestrzeniach Hilberta wymiaru \(n\), w których operatory liniowe możemy utożsamiać z
\textit{liniowymi endomorfizmami} tej przestrzeni w siebie
\begin{equation*}
\oper{A}: V \mapsto V\,.
\end{equation*}

Jak wiadomo z elementarnej algebry w przypadku \(n\)--wymiarowej przestrzeni wektorowej każdemu
endomorfizmowi \(\oper{A}\) możemy przyporządkować macierz wymiaru \(n\times n\), której elementy w
danej bazie ortonormalnej \(\{\phi_1,\ldots,\phi_n\}\) są dane przez wartości \(\oper{A}\) na
wektorach bazowych
\begin{equation*}
\begin{split}
\oper{A}\ket{\phi_1} &= A_{11}\ket{\phi_1} + A_{21}\ket{\phi_2} + \ldots + A_{n1}\ket{\phi_n}\\
&\vdots\\
\oper{A}\ket{\phi_n} &= A_{1n}\ket{\phi_1} + A_{2n}\ket{\phi_2} + \ldots + A_{nn}\ket{\phi_n}
\end{split}\quad,
\end{equation*}
skąd element \(A_{ij}\) macierzy \(\oper{A}\) w bazie ortonormalnej \(\{\phi_i\}\) jest dany przez
\begin{equation*}
A_{ij} = \bra{\phi_i}\oper{A}\ket{\phi_j}\,.
\end{equation*}

\begin{definition}
\textit{Sprzężeniem} operatora \(\oper{A}\) nazywamy operator \(\oper{A}^\dagger\) zdefiniowany
(pomijając wszelkie problemy związane z określeniem dziedzin operatorów) przez równanie
\begin{equation*}
\forall \psi,\phi\in\mathscr{H}: \bra{\psi}\oper{A}^\dagger\ket{\phi} = \bra{\phi}\oper{A}\ket{\psi}^*\,.
\end{equation*}
\end{definition}

Podstawiając w miejsce wektorów \(\psi\), \(\phi\) wektory bazy otrzymujemy (w przypadku skończenie
wymiarowych przestrzeni Hilberta) zależność między macierzą \(\oper{A}\) i jej sprzężeniem
hermitowskim \(\oper{A}^\dagger\)
\begin{equation*}
A^\dagger_{ij} = A_{ji}^*\,.
\end{equation*} 

\begin{definition}
\textit{Komutatorem} operatorów \(\oper{A}\), \(\oper{B}\) operator \([\oper{A},\oper{B}]\)
zdefiniowany jako
\begin{equation*}
\forall \psi \in D : [\oper{A},\oper{B}]\psi = \oper{A}\oper{B}\psi - \oper{B}\oper{A}\psi\,. 
\end{equation*}
Jeśli \([\oper{A},\oper{B}] = \oper{\mathsf{0}}\) (gdzie \(\oper{\mathsf{0}}\) oznacza
\textit{operator zerowy} \(\oper{\mathsf{0}}\psi=\mathsf{0}\)), to mówimy, że operatory
\(\oper{A}\), \(\oper{B}\) \textit{komutują}.
\end{definition}

\begin{definition}
\textit{Antykomutatorem} operatorów \(\oper{A}\), \(\oper{B}\) nazywamy operator
\(\{\oper{A},\oper{B}\}\) zdefiniowany jako
\begin{equation*}
    \{\oper{A},\oper{B}\}\psi := \oper{A}\oper{B}\psi + \oper{B}\oper{A}\psi\,.
\end{equation*}
\end{definition}

\begin{theorem}
Dla dowolnych operatorów \(\oper{A}\), \(\oper{B}\), \(\oper{C}\) zakładając odpowiednie dziedziny,
zachodzi:
\begin{itemize}
    \item \([\oper{A} + \oper{B},\oper{C}] = [\oper{A},\oper{C}] + [\oper{B},\oper{C}]\) ;
    \item \([\oper{A}\oper{B},\oper{C}] = \oper{A}[\oper{B},\oper{C}] +
    [\oper{A},\oper{C}]\oper{B}\) .
    \item \([[\oper{A}, \oper{B}], \oper{C}] + [[\oper{B},\oper{C}],\oper{A}] + [[\oper{C},
    \oper{A}], \oper{B}] = \oper{0}\) (\textit{tożsamość Jacobiego})
\end{itemize}   
\end{theorem}

\begin{definition}
\textit{Ślad operatora \(\oper{A}\)} definiujemy jako liczbę \(\Trace \oper{A} \in \mathbb{C}\)
równą
\begin{equation*}
\Trace{\oper{A}} := \sum_{i} \bra{\phi_i}\oper{A}\ket{\phi_i}\,,
\end{equation*}
gdzie \(\{\phi_i\}\) jest dowolną ortonormalną bazą przestrzeni \(\mathscr{H}\).
\end{definition}

\begin{theorem}
Ślad operatora nie zależy od wyboru ortonormalnej bazy przestrzeni Hilberta.    
\end{theorem}

\begin{theorem} Podstawowe własności śladu.
\begin{itemize}
    \item Operacja wzięcia śladu operatora jest liniowa tj. 
    \begin{equation*}
        \Trace(\alpha\oper{A} + \beta\oper{B}) = \alpha\Trace(\oper{A}) + \beta\Trace(\oper{B})\,.
    \end{equation*}

    \item Dla dowolnych operatorów \(\oper{A}\), \(\oper{B}\), \(\oper{C}\) zachodzi
    \begin{equation*}
        \Trace{\oper{ABC}} = \Trace{\oper{BCA}} = \Trace{\oper{CAB}}\,.
    \end{equation*}

    \item \(\Trace{\oper{A}} = (\Trace{\oper{A}^\dagger})^*\)
    \item \(\det{\e^{\oper{A}}} = \e^{\Trace{\oper{A}}}\)
\end{itemize}  
\end{theorem}

\begin{definition}[\textit{Funkcja operatora}]
Niech \(f(x):\mathbb{R}\mapsto\mathbb{R}\) będzie funkcją zmiennej rzeczywistej taką, że istnieje
szereg potęgowy
\begin{equation*}
\sum_{n=0}^\infty\frac{a_n}{n!}x^n\,,
\end{equation*}
który jest zbieżny jednostajnie na \(\mathbb{R}\) do \(f\). Wówczas funkcję operatora
\(f(\oper{A})\) definiujemy jako
\begin{equation*}
f(\oper{A}) := \sum_{n=0}^\infty \frac{a_n}{n!}\oper{A}^n\,,
\end{equation*}
gdzie przyjmujemy \(\oper{A}^0 := \oper{I}\). W szczególności mamy
\begin{itemize}
    \item \(\exp(\oper{A}) := \sum_{n=0}^\infty \frac{1}{n!}\oper{A}^n\)

    \item \(\sin(\oper{A}) := \sum_{n=0}^\infty \frac{(-1)^n}{(2n+1)!}\oper{A}^{2n+1}\)

    \item \(\cos(\oper{A}) := \sum_{n=0}^\infty \frac{(-1)^n}{(2n)!}\oper{A}^{2n}\)
\end{itemize}
\end{definition}

W teorii kwantowej główną rolę odgrywają trzy rodziny operatorów: operatory samosprzężone, rzutowe i
unitarne.

\linesep

\begin{definition}
    \textit{Operatorem samosprzężonym} (pomijając wszelkie problemy związane z określeniem dziedzin
    operatorów) nazywamy operator \(\oper{A}\), dla którego \(\oper{A} = \oper{A}^\dagger\).
\end{definition}

W przypadku skończenie wymiarowych przestrzeni Hilberta definicja ta jest pełna, a operatory
samosprzężony możemy utożsamiać z macierzami hermitowskimi tj. macierzami, których elementy
spełniają związek
\begin{equation*}
    A_{ij} = A_{ji}^*\,.
\end{equation*}

W przypadku nieskończenie wymiarowych przestrzeni Hilberta definicja ta jest niepełna gdyż trzeba
mieć świadomość, iż równość operatorów oznacza z definicji równość ich dziedzin, co wymaga
wprowadzenia rozróżnienia między operatorami jedynie \textit{symetrycznymi} (tj. spełniającymi
równość \(\bra{\psi}\oper{A}\ket{\phi} = \bra{\phi}\oper{A}\ket{\psi}\) dla dowolnych \(\psi,\phi\in
D(\oper{A})\)), a operatorami samosprzężonymi.

Operatory samosprzężone odgrywają wyróżnioną rolę w teorii kwantowej ze względu na trzy twierdzenia,
które są dla nich spełnione.

\begin{theorem}
Wartości własne operatora samosprzężonego są liczbami rzeczywistymi.   
\end{theorem}

\begin{theorem}
Zbiór wektorów własnych operatora samosprzężonego rozpina przestrzeń \(\mathscr{H}\).   
\end{theorem}

\begin{theorem}
Jeśli widmo operatora samosprzężonego nie jest zdegenerowane, to wektory własne tworzą zbiór
ortogonalny.
\end{theorem}

\linesep

\begin{definition}
\textit{Operatorem rzutowym} nazywamy operator \(\oper{P}\) taki, że \(\oper{P} = \oper{P}^\dagger\)
(samosprzężoność) i \(\oper{P}^2 = \oper{P}\) (idempotentność). 
\end{definition}

Ważnym przykładem operatora rzutowego jest operator rzutowania na jednowymiarową podprzestrzeń
rozpiętą na unormowanym wektorze \(\ket{\phi}\) (rzutowanie na kierunek wektora \(\phi\)), który w
notacji Diraca możemy zapisać jako \(\oper{P} = \ket{\phi}\bra{\phi}\) tj. \(\forall \psi :
\oper{P}(\psi) = \braket{\phi}{\psi}\phi\). Jest to oczywiście operator liniowy, gdyż dla dowolnych
wektorów \(\ket{\psi_1}\), \(\ket{\psi_2}\) i skalarów \(\alpha\), \(\beta\) mamy
\begin{equation*}
    \begin{split}
        &\ket{\phi}\bra{\phi}(\alpha\ket{\psi_1} + \beta\ket{\psi_2}) = \ket{\phi}\braket{\phi}{\alpha\psi_1 + \beta\psi_2} \\
        &= \alpha\ket{\phi} \braket{\phi}{\psi_1} + \beta\ket{\phi}\braket{\phi}{\psi_2}\,.
    \end{split}
\end{equation*}
Jest również idempotentny, gdyż 
\begin{equation*}
    \ket{\phi}\bra{\phi}(\ket{\phi}\braket{\phi}{\psi}) =  \ket{\phi}\braket{\phi}{\psi}
\end{equation*}
z założenia \(\braket{\phi} = 1\) oraz samosprzężony
\begin{equation*}
    (\braket{\psi_1}{\phi}\braket{\phi}{\psi_2})^* = \braket{\psi_1}{\phi}^*\braket{\phi}{\psi_2}^* = \braket{\phi}{\psi_1}\braket{\psi_2}{\phi} = \braket{\psi_2}{\phi}\braket{\phi}{\psi_1}\,.
\end{equation*}
Łatwo pokazać również, iż jeśli \(\{\phi_i\}\) jest ortonormalnym zbiorem wektorów, to
\begin{equation*}
    \oper{P} = \sum_i \ket{\phi_i}\bra{\phi_i}
\end{equation*}
jest operatorem rzutowym. W szczególności, jeśli \(\{\phi_i\}\) jest ortonormalną bazą przestrzeni
\(\mathscr{H}\), to
\begin{equation*}
    \sum_i \ket{\phi_i}\bra{\phi_i} = \oper{I}\,.
\end{equation*}

\linesep

\begin{definition}
\textit{Operatorem unitarnym} nazywamy operator \(\oper{U}\) taki, że
\begin{equation*}
    \oper{U}\oper{U}^\dagger = \oper{U}^\dagger\oper{U} = \oper{I}\,.
\end{equation*}
\end{definition}

Przekształcenia unitarne reprezentowane przez operatory unitarne mają użyteczną własność polegającą
na zachowywaniu wartości iloczynu wewnętrznego dwóch wektorów, a zatem w szczególności normy wektora
\begin{equation*}
    \braket{\oper{U}\psi}{\oper{U}\phi} = \braket{\oper{U}^\dagger\oper{U}\psi}{\phi} = \braket{\psi}{\phi}\,.
\end{equation*}

\linesep

\begin{textbox}
\begin{theorem}[\textit{spektralne}]
Niech \(\mathscr{H}\) będzie przestrzenią Hilberta. Dla każdego samosprzężonego operatora liniowego
\(\oper{A}\) w \(\mathscr{H}\) istnieje unikalna rodzina operatorów rzutowych \(\oper{P}(\lambda)\)
indeksowanych ciągłym parametrem \(\lambda \in \mathbb{R}\) taka, że
\begin{itemize}
    
    \item \(\oper{P}(\lambda_1)\oper{P}(\lambda_2) = \oper{P}(\min(\lambda_1,\lambda_2))\)

    \item \(\forall \lambda: \lim_{\epsilon\to0^+}\oper{P}(\lambda+\epsilon) = \oper{P}(\lambda)\)
    
    \item \(\lim_{\lambda\to-\infty}\oper{P}(\lambda) = \oper{0}\)

    \item \(\lim_{\lambda\to+\infty}\oper{P}(\lambda) = \oper{I}\)

    \item \(\oper{A} = \int\limits_{-\infty}^{+\infty}\lambda \dd{\oper{P}(\lambda)}\)

\end{itemize}
gdzie ostatnia całka to tzw. \textit{całka Riemanna--Stieltjesa} względem miary operatorowej
zdefiniowana jako
\begin{equation*}
        \int\limits_a^b f(x) \dd{\sigma(x)} := \lim_{n\to\infty}\sum_{k=1}^nf(x_k)\left[\sigma(x_k) - \sigma(x_{k-1})\right]\,,
\end{equation*}
dla 
\begin{equation*}
    f: \mathbb{R} \mapsto \mathbb{R}\,,\quad \sigma: \mathbb{R} \mapsto X\,,
\end{equation*}
gdzie \([a;b] = \bigcup_{k=1}^{n}[x_{k-1}; x_k]\) jest podziałem normalnym odcinka \([a;b]\).
Dodatkowo dla dowolnej funkcji operatora \(f\) zachodzi
\begin{equation*}
    f(\oper{A}) = \int\limits_{-\infty}^{+\infty} f(\lambda) \dd{\oper{P}(\lambda)}\,.
\end{equation*}
\end{theorem}        
\end{textbox}

W szczególnym przypadku, gdy operator samosprzężony \(\oper{A}\) ma niezdegenerowane widmo
\(\{\lambda_i\}\) będące zbiorem przeliczalnym, wiemy, że zbiór unormowanych wektorów własnych
\(\{\phi_i\}\) jest bazą ortonormalną przestrzeni \(\mathscr{H}\), czyli dla dowolnego wektora
\(\psi \in \mathscr{H}\) możemy zapisać
\begin{equation*}
    \ket{\psi} = \sum_{i} c_i\ket{\phi_i} = \sum_{i} \braket{\phi_i}{\psi}\ket{\phi_i}\,,
\end{equation*}
gdzie \(c_i = \braket{\phi_i}{\psi} \in \mathbb{C}\) to współrzędne wektora w zadanej bazie.
Działając operatorem \(\oper{A}\) na wektor \(\psi\) mamy
\begin{equation*}
    \oper{A}\ket{\psi} = \sum_{i} \braket{\phi_i}{\psi}\oper{A}\ket{\phi_i} =
    \sum_{i} \braket{\phi_i}{\psi}\lambda_i\ket{\phi_i} = \left(\sum_i\lambda_i\ket{\phi_i}\bra{\phi_i}\right)\ket{\psi} \,.
\end{equation*}
Całka Stieltjesa z twierdzenia spektralnego przechodzi więc w tym przypadku w sumę (być może
nieskończoną) operatorów rzutowych rzutujących na jednowymiarowe podprzestrzenie rozpięte na
kolejnych wektorach własnych operatora
\begin{equation*}
    \oper{A} = \sum_{i} \lambda_i \ket{\phi_i}\bra{\phi_i}\,.
\end{equation*}

\subsection{POSTULATY TEORII KWANTÓW}

Poniżej przedstawiono postulaty ogólnej, abstrakcyjnej teorii kwantów. Postulaty te obowiązują we
wszystkich realizacjach teorii kwantów np. mechanice falowej, czy kwantowej teorii pola, jednak ze
względu na swój ogólny charakter same w sobie nie dostarczają narzędzi do rozwiązywania żadnych
konkretnych problemów fizycznych. Nie należy ich również traktować jako podstaw do aksjomatyzacji
teorii kwantowej. Stanowią one raczej sposób uporządkowania w spójną strukturę wiedzy dotyczącej
konkretnych realizacji teorii kwantów
\medskip

\begin{enumerate}[label=\Roman*.]
    
    \item \textbf{O modelu matematycznym.} Modelem matematycznym teorii kwantów jest teoria
    przestrzeni Hilberta nad ciałem liczb zespolonych i teoria operatorów liniowych działających w
    tej przestrzeni.

    \item \textbf{O pytaniach elementarnych.} Pytaniem elementarnym nazwiemy pytanie, na które
    odpowiedź może brzmieć jedynie ,,TAK'' lub ,,NIE''. Pytanie elementarne nazwiemy rozstrzygalnym
    w obrębie danej teorii kwantowej iff istnieje wzajemnie jednoznaczne przyporządkowanie tego
    pytania do pewnego operatora rzutowego \(\oper{P}\). Będziemy wówczas mówili, iż dane pytanie
    elementarne jest reprezentowane przez \(\oper{P}\). Każde pytanie elementarne reprezentowane
    przez \(\oper{P}\) można zanegować otrzymując pytanie reprezentowane przez \(\oper{I} -
    \oper{P}\) (zauważmy, że \((\oper{I} - \oper{P})^2 = \oper{I} - \oper{P}\)), natomiast dwa
    pytania elementarne reprezentowane przez \(\oper{P}_1\) i \(\oper{P}_2\) można połączyć
    spójnikiem
    \begin{itemize}
        \item ,,I''; otrzymując pytanie reprezentowane przez \(\oper{P}_1\oper{P}_2\), przy czym z
        oczywistych względów musi zachodzić \([\oper{P}_1, \oper{P}_2] = \oper{\mathsf{0}}\)

        \item ,,LUB''; otrzymując pytanie reprezentowane przez \(\oper{P}_1 + \oper{P}_2\), przy
        czym musi zachodzić \(\oper{P}_1\oper{P}_2 = \oper{\mathsf{0}}\) (istotnie \((\oper{P}_1 +
        \oper{P}_2)^2 = \oper{P}_1 + \oper{P}_2 \)).

    \end{itemize}

    \item \textbf{O stanach układu.} Stan prostego układu fizycznego jest reprezentowany przez
    unormowany wektor \(\ket{\Psi}\) w abstrakcyjnej przestrzeni Hilberta \(\mathscr{H} =
    (\mathbb{V}, \braket{\cdot})\), przy czym utożsamiamy ze sobą wektory różniące się jedynie
    globalnym czynnikiem fazowym tj. \(\ket{\Psi} \cong \e^{\im\alpha}\ket{\Psi}\) dla \(\alpha \in
    \mathbb{R}\).

    \item \textbf{O prawdopodobieństwach.} Teoria kwantowa dostarcza jedynie probabilistycznych
    odpowiedzi na rozstrzygalne pytania elementarne. Prawdopodobieństwo \(p\), iż odpowiedź na
    pytanie elementarne reprezentowane przez \(\oper{P}\) jest twierdząca, dla układu
    reprezentowanego przez \(\Psi\) wynosi
    \begin{equation*}
        p = \bra{\Psi}\oper{P}\ket{\Psi}\,.
    \end{equation*}
    Zauważmy, iż trywialnie \(p\in\mathbb{R}\) oraz z nierówności Cauchy'ego--Schwarza mamy
    \begin{equation*}
        p^2 \leq \braket{\Psi} \braket{\oper{P}\Psi} = \bra{\Psi}\oper{P}^2\ket{\Psi} = p\,,
    \end{equation*}
    skąd \(p(p-1) \leq 0\), czyli \(p \in [0;1]\).

    \item \textbf{O wielkościach fizycznych.} Każda wielkość fizyczna \(A\) występująca w danej
    teorii kwantowej jest reprezentowana przez samosprzężony operator liniowy \(\oper{A}\) i
    stowarzyszoną z nim na mocy twierdzenia spektralnego rodzinę operatorów rzutowych
    \(\oper{P}_A(\lambda)\). Operator rzutowy \(\oper{P}_A(\lambda)\) reprezentuje pytanie:
    \textit{czy wielkość fizyczna \(A\) ma wartość nie większą od \(\lambda\)?}, natomiast operator
    rzutowy \(\oper{I}-\oper{P}_A(\lambda)\): \textit{czy wielkość fizyczna \(A\) ma wartość większą
    od \(\lambda\)?} Na mocy postulatu drugiego możemy skonstruować pytanie: \textit{czy wielkość
    fizyczna \(A\) ma wartość z przedziału \((\lambda_1;\lambda_2]\)?}, reprezentowane przez
    operator 
    \begin{equation*}
        (\oper{I} - \oper{P}_A(\lambda_1))\oper{P}_A(\lambda_2) =
        \oper{P}_A(\lambda_2)-\oper{P}_A(\lambda_1)\,.
    \end{equation*}
    Wartość oczekiwaną wielkości \(A\) dla układu reprezentowanego przez \(\Psi\) obliczamy jako
    \begin{equation*}
        \langle A \rangle = \bra{\Psi}\oper{A}\ket{\Psi}\,.
    \end{equation*}

    \item \textbf{O ewolucji układu w czasie.} Prawdopodobieństwo \(p\) odpowiedzi twierdzącej na
    pytanie \(\oper{P}\) dla układu reprezentowanego przez \(\Psi\) ewoluuje w czasie zgodnie z
    \begin{equation*}
        p(t) = \bra{\Psi(t)}\oper{P}\ket{\Psi(t)}\,,
    \end{equation*}
    gdzie wektory stanu \(\ket{\Psi}\) ewoluują zgodnie z \textit{równaniem Schr\"{o}dingera}
    \begin{equation*}
        \oper{H}\ket{\Psi} = \im\hbar\partial_t\ket{\Psi}\,,
    \end{equation*}
    gdzie w ogólności \(\oper{H} = \oper{H}(t)\) jest operatorem Hamiltona danego układu tworzonym
    wedle określonych reguł w danej realizacji teorii kwantów, natomiast \(\hbar\) to stała fizyczna
    o wymiarze działania
    \begin{equation*}
        \hbar = \frac{h}{2\pi} = 1.054571817\ldots \cdot 10^{-34}\,\text{J}\cdot\text{s}
    \end{equation*}
    zwana \textit{zredukowaną stałą Plancka}.
    
    \item \textbf{O układach złożonych.} Przestrzeń Hilberta \(\mathscr{H}\) układu złożonego ma
    strukturę iloczynu tensorowego przestrzeni Hilberta układów prostych wchodzących w jego skład
    \(\mathscr{H}=\mathscr{H}_1\otimes\mathscr{H}_2\otimes\ldots\otimes\mathscr{H}_n\) .

\end{enumerate}

\subsection{Zasada nieoznaczoności}

Niech \(A\) będzie pewną wielkością fizyczną reprezentowaną przez operator \(\oper{A}\). Zdefiniujmy
odchylenie standardowe \(\sigma_A \geq 0\) wielkości \(A\) dla układu w stanie \(\Psi\) jako
\begin{equation*}
    \sigma_A^2 := \langle (A - \langle A \rangle)^2 \rangle = \braket{(\oper{A}-a)\Psi}{(\oper{A}-a)\Psi}\,,
\end{equation*}
gdzie \(a = \expval{A}\) jest wartością oczekiwaną wielkości \(A\). Dla dowolnych dwóch wielkość
fizycznych \(A\) i \(B\) w układzie reprezentowanym przez \(\Psi\) mamy z nierówności
Cauchy'ego--Schwarza
\begin{equation*}
    \sigma_A^2\sigma_B^2 \geq \left|\braket{(\oper{A}-a)\Psi}{(\oper{B}-b)\Psi}\right|^2\,.
\end{equation*}
Jednocześnie dla dowolnego \(z = x + \im y \in\mathbb{C}\) mamy
\begin{equation*}
    |z|^2 = x^2 + y^2 = \left(\frac{z + z^*}{2}\right)^2 + \left(\frac{z-z^*}{2\im}\right)^2\,.
\end{equation*}
Z powyższego mamy więc
\begin{equation*}
    \sigma_A^2\sigma_B^2 \geq \left[\frac{1}{2\im}\left(\braket{\oper{\mathcal{A}}\Psi}{\oper{\mathcal{B}}\Psi} - \braket{\oper{\mathcal{B}}\Psi}{\oper{\mathcal{A}}\Psi}\right)\right]^2 + \left[\frac{1}{2}(\braket{\oper{\mathcal{A}}\Psi}{\oper{\mathcal{B}}\Psi} + \braket{\oper{\mathcal{B}}\Psi}{\oper{\mathcal{A}}\Psi})\right]^2\,,
\end{equation*}
gdzie \(\oper{\mathcal{A}} := \oper{A} - a\), \(\oper{\mathcal{B}} := \oper{B} - b\). Jednocześnie
\begin{equation*}
    \begin{split}
        &\braket{\oper{\mathcal{A}}\Psi}{\oper{\mathcal{B}}\Psi} - \braket{\oper{\mathcal{B}}\Psi}{\oper{\mathcal{A}}\Psi} =
        \bra{\Psi}(\oper{\mathcal{A}}\oper{\mathcal{B}}\ket{\Psi}-\oper{\mathcal{B}}\oper{\mathcal{A}}\ket{\Psi})=\\
        &\bra{\Psi}[\oper{\mathcal{A}},\oper{\mathcal{B}}]\ket{\Psi} = \bra{\Psi}[\oper{A},\oper{B}]\ket{\Psi}\,.
    \end{split}
\end{equation*}
oraz
\begin{equation*}
    \braket{\oper{\mathcal{A}}\Psi}{\oper{\mathcal{B}}\Psi} + \braket{\oper{\mathcal{B}}\Psi}{\oper{\mathcal{A}}\Psi} = \bra{\Psi}\{\oper{\mathcal{A}},\oper{\mathcal{B}}\}\ket{\Psi} = \bra{\Psi}\{\oper{A},\oper{B}\}\ket{\Psi} - 2\expval{A}\expval{B}
\end{equation*}
Ostatecznie otrzymujemy więc
\begin{equation*}
    \boxed{
    \sigma_A^2\sigma_B^2 \geq \left(\frac{1}{2\im}\bra{\Psi}[\oper{A},\oper{B}]\ket{\Psi}\right)^2 + \left(\frac{1}{2}\bra{\Psi}\{\oper{A},\oper{B}\}\ket{\Psi} - \expval{A}\expval{B}\right)^2
    }\quad.
\end{equation*}
Powyższą nierówność nazywamy \textit{uogólnioną zasadą nieoznaczoności}.

\subsection{Twierdzenie Ehrenfesta}

Niech \(A\) będzie pewną wielkością fizyczną reprezentowaną przez operator \(\oper{A}\), wówczas
\begin{equation*}
        \dv{\expectationvalue{A}}{t} = \dv{}{t} \braket{\Psi}{\oper{A}\Psi}
        = \braket{\pdv{\Psi}{t}}{\oper{A}\Psi} + \braket{\Psi}{\oper{A}\pdv{\Psi}{t}} + \braket{\Psi}{\pdv{\oper{A}}{t}\Psi}\,.
\end{equation*}
Jednocześnie z równania Schr\"{o}dingera mamy \(\oper{H}\Psi=\im\hbar{\partial_t\Psi}\), skąd
\begin{equation*}
    \dv{\expectationvalue{A}}{t} = \frac{\im}{\hbar}\braket{\oper{H}\Psi}{\oper{A}\Psi} - \frac{\im}{\hbar}\braket{\Psi}{\oper{A}\oper{H}\Psi} + \bra{\Psi}\pdv{\oper{A}}{t}\ket{\Psi}\,,
\end{equation*}
ale ze względu na fakt, iż \(\oper{H}\) jest operatorem samosprzężonym mamy
\begin{equation*}
    \boxed{
    \dv{\expectationvalue{A}}{t} = \bra{\Psi}\pdv{\oper{A}}{t}\ket{\Psi} + \frac{\im}{\hbar}\bra{\Psi}[\oper{H},\oper{A}]\ket{\Psi}
    }\quad.
\end{equation*}
Powyższe równanie nazywamy \textit{twierdzeniem Ehrenfesta}.

\subsection{Kwantowe układy dwupoziomowe}

Przedstawimy teraz ważną realizację abstrakcyjnej teorii kwantów -- teorię układów dwupoziomowych,
które stanowią podstawę teorii obliczeń kwantowych i kwantowej teorii informacji. Modelem
matematycznym tej teorii jest skończenie wymiarowa przestrzeń Hilberta
\begin{equation*}
    \mathscr{H} = ((\mathbb{C}^2, \mathbb{C}, +, \cdot), \braket{\cdot})
\end{equation*}
i teoria operatorów liniowych w tej przestrzeni, które możemy utożsamiać z zespolonymi macierzami
\(2\times 2\).

\subsubsection{Macierze Pauliego}

Macierze Pauliego definiujemy jako zespolone macierze \(2\times 2\)
\begin{equation*}
    \oper{X} := \mqty[0 & 1 \\ 1 & 0]\,,\quad \oper{Y} := \mqty[0 & -\im \\ \im & 0]\,,\quad \oper{Z} := \mqty[1 & 0 \\ 0 & -1]\,.
\end{equation*}
Przydatne jest zdefiniowanie \textit{wektora macierzy Pauliego} \(\boldsymbol{\sigma}\)
\begin{equation*}
    \boldsymbol{\sigma} = (\oper{X}, \oper{Y}, \oper{Z})\,,
\end{equation*}
dzięki któremu możemy łatwo zapisać sumę przeskalowanych macierzy Pauliego jako
\(\mathbf{c}\cdot\boldsymbol{\sigma}\), gdzie \(\mathbf{c}\in\mathbb{C}^3\) jest pewnym wektorem o
elementach zespolonych. Wybrane własności macierzy Pauliego:

\begin{itemize}
    
    \item \(\det\oper{X} = \det\oper{Y} = \det\oper{Z} = -1\)

    \item \(\Tr\oper{X} = \Tr\oper{Y} = \Tr\oper{Z} = 0\)

    \item \(\oper{X}^2 = \oper{Y}^2 = \oper{Z}^2 = \oper{I}\)

    \item Iloczyny macierzy Pauliego spełniają związki
    \begin{equation*}
        \begin{split}
            &\oper{X}\oper{Y} = \im\oper{Z} = -\oper{Y}\oper{X} \\
            &\oper{Y}\oper{Z} = \im\oper{X} = -\oper{Z}\oper{Y} \\
            &\oper{Z}\oper{X} = \im\oper{Y} = -\oper{X}\oper{Z}
        \end{split}\quad.
    \end{equation*}

    \item Komutatory i antykomutatory macierzy Pauliego wynoszą
    \begin{equation*}
        \begin{split}
            [\oper{X},\oper{Y}] = 2\im\oper{Z}\,,\quad \{\oper{X}, \oper{Y}\} = \oper{\mathsf{0}}\\
            [\oper{Y},\oper{Z}] = 2\im\oper{X}\,,\quad \{\oper{Y}, \oper{Z}\} = \oper{\mathsf{0}}\\
            [\oper{Z},\oper{X}] = 2\im\oper{Y}\,,\quad \{\oper{Z}, \oper{X}\} = \oper{\mathsf{0}}\\
        \end{split}\quad.
    \end{equation*}

    \item Iloczyn \((\mathbf{a}\cdot\boldsymbol{\sigma})\cdot(\mathbf{b}\cdot\boldsymbol{\sigma})\)
    dla \(\mathbf{a} = (a_x,a_y,a_z)\), \(\mathbf{b}=(b_x,b_y,b_z)\) wynosi
    \begin{equation*}
        \begin{split}
            &(\mathbf{a}\cdot\boldsymbol{\sigma})\cdot(\mathbf{b}\cdot\boldsymbol{\sigma}) = \\
            &= (a_x\oper{X} + a_y\oper{Y} + a_z\oper{Z})(b_x\oper{X} + b_y\oper{Y} + b_z\oper{Z}) =\\
            &= (\mathbf{a}\cdot\mathbf{b})\oper{I} + \im(\mathbf{a}\times\mathbf{b})\cdot\boldsymbol{\sigma}\,.
        \end{split}
    \end{equation*}

    \item Wielkość \(\exp(\im\mathbf{a}\cdot\boldsymbol{\sigma})\) wynosi
    \begin{equation*}
            \exp(\im\mathbf{a}\cdot\boldsymbol{\sigma}) = \oper{I} \cos{|\mathbf{a}|} + \im\left(\frac{\mathbf{a}}{|\mathbf{a}|}\cdot\boldsymbol{\sigma}\right)\sin{|\mathbf{a}|}\,.
    \end{equation*}

\end{itemize}

Zauważmy, że dowolny operator samosprzężony \(\oper{A}\) w rozpatrywanej przestrzeni \(\mathscr{H}\)
ma postać
\begin{equation*}
    \oper{A} = \mqty[a_0 + a_z & a_x - \im a_y \\ a_x + \im a_y & a_0 - a_z] = a_0\oper{I} + \mathbf{a}\cdot\boldsymbol{\sigma}\,, 
\end{equation*}
gdzie \(a_0,a_x,a_y,a_z \in \mathbb{R}\). Jednocześnie bez straty ogólności możemy przyjąć \(a_0 =
0\), gdyż stała ta przesuwa jedynie widmo operatora \(\oper{A}\) o ustaloną wartość, co pozwala
przedstawić dowolny operator samosprzężony jako wektor \(\mathbf{a}\) w trójwymiarowej przestrzeni.

\subsubsection{Sfera Blocha}

W trójwymiarowej przestrzeni możemy również przedstawić wektor stanu \(\Psi\). Istotnie wektor stanu
jest określony przez 2 zmienne zespolone
\begin{equation*}
    \ket{\Psi} = \mqty[\alpha \\ \beta]\,,
\end{equation*}
czyli 4 zmienne rzeczywiste, ale ze względu na warunek unormowania \(|\alpha|^2 + |\beta|^2 = 1\)
mamy tylko 3 zmienne niezależne. Dodatkowo pamiętając, iż globalna faza wektora stanu nie ma
znaczenia możemy wyeliminować jeszcze jedną zmienną i zapisać wektor \(\Psi\) jako
\begin{equation*}
    \ket{\Psi} = \cos\frac{\theta}{2}\mqty[1 \\ 0] + \e^{\im\phi}\sin\frac{\theta}{2}\mqty[0\\1] = \cos\frac{\theta}{2}\ket{0} + \e^{\im\phi}\sin\frac{\theta}{2}\ket{1}
\end{equation*}
dla pewnych parametrów \(\phi,\theta\in\mathbb{R}\). Powyżej wektory bazy ortonormalnej oznaczyliśmy
jako \(\{\ket{0}, \ket{1}\}\) zgodnie z oznaczeniami stosowanymi w teorii obliczeń kwantowych (w
szczególności \(\ket{0}\) \textit{nie oznacza} w powyższej notacji wektora zerowego \(\mathsf{0}\)).
Zmienne \(\phi\), \(\theta\) możemy interpretować odpowiednio jako kąt azymutalny i kąt zenitalny
punktu na sferze jednostkowej, którą nazywamy \textit{sferą Blocha}. Zauważmy, iż przy wybranej
parametryzacji bieguny sfery określają odpowiednio stany \(\ket{0}\) i \(\ket{1}\).

\begin{figure}[ht]
    \centering
    \includegraphics[width=0.45\columnwidth]{figs/Bloch_sphere.png}
    \label{fig:bloch_sphere}
    \caption{Sfera Blocha}
\end{figure}

Możemy połączyć oba przedstawienia tj. przedstawienia operatora i wektora stanu jeśli tylko zamiast
wektora stanu użyjemy operatora rzutowania na stan \(\Psi\). Możemy rozłożyć go wówczas (jak każdy
operator samosprzężony) na macierze Pauliego
\begin{equation*}
    \ket{\Psi}\bra{\Psi} = s_0\oper{I} + \mathbf{s}\cdot\boldsymbol{\sigma}\,,
\end{equation*}
przy czym parametry \(s_0,s_x,s_y,s_z\) muszą spełniać
\begin{equation*}
    \ket{\Psi}\bra{\Psi}\ket{\Psi}\bra{\Psi} = \ket{\Psi}\bra{\Psi}\,,
\end{equation*}
czyli
\begin{equation*}
    s_0\oper{I} + \mathbf{s}\cdot\boldsymbol{\sigma} = (s_0^2 + |\mathbf{s}|^2)\oper{I} + 2s_0\mathbf{s}\cdot\boldsymbol{\sigma}\,,
\end{equation*}
skąd \(s_0 = |\mathbf{s}| = 1/2\). Operator rzutowania \(\ket{\Psi}\bra{\Psi}\) możemy zatem w
ogólności rozłożyć na macierze Pauliego w następujący sposób
\begin{equation*}
    \ket{\Psi}\bra{\Psi} = \frac{1}{2}(\oper{I} + \mathbf{s}\cdot\boldsymbol{\sigma})\,,
\end{equation*}
gdzie przeskalowaliśmy zmienne \(s_x,s_y,s_z\) tak, że teraz \(\boxed{|\mathbf{s}| = 1}\).

Widzimy więc, iż stan \(\Psi\) możemy reprezentować jako wektor \(\mathbf{s}\) określający punkt na
sferze jednostkowej w trójwymiarowej przestrzeni, a operator samosprzężony jako dowolny wektor
\(\mathbf{a}\) w tej przestrzeni. Przejście od rzeczywistego wektora \(\mathbf{s}\) do
abstrakcyjnego wektora stanu \(\ket{\Psi}\) odbywa sie poprzez określenie współrzędnych sferycznych
\((\theta,\phi)\) wektora \(\mathbf{s}\) i zmapowanie ich zgodnie z przepisem
\begin{equation*}
    \ket{\Psi} = \cos\frac{\theta}{2}\ket{0} + \e^{\im\phi}\sin\frac{\theta}{2}\ket{1}\,.
\end{equation*}


\subsubsection{Ewolucja układu dwupoziomowego}

Dla układów dwupoziomowych ogólne równania ewolucji amplitud prawdopodobieństwa wektora stanu
\begin{equation*}
    \ket{\Psi} = \mqty[\alpha \\ \beta]
\end{equation*}
dla hamiltonianu postaci
\begin{equation*}
    \oper{H} = \mqty[H_{11}(t) & H_{12}(t) \\ H_{21}(t) & H_{22}(t)]
\end{equation*}
mają postać
\begin{equation*}\boxed{
    \begin{cases}
        \im\hbar\dot{\alpha} = H_{11}(t)\alpha + H_{12}(t)\beta\\
        \im\hbar\dot{\beta}  = H_{21}(t)\alpha + H_{22}(t)\beta
    \end{cases}}
\end{equation*}

\linesep

Równanie ewolucji możemy zapisać również wykorzystując przedstawienie geometryczne wektorów stanu i
operatorów na sferze Blocha. Istotnie różniczkując operator rzutowy \(\oper{\rho} =
\ket{\Psi}\bra{\Psi}\) mamy
\begin{equation*}
    \begin{split}
        \pdv{\oper{\rho}}{t} &= \ket{\dot{\Psi}}\bra{\Psi} + \ket{\Psi}\bra{\dot{\Psi}} = \frac{1}{\im\hbar}\ket{\oper{H}\Psi}\bra{\Psi} - \frac{1}{\im\hbar}\ket{\Psi}\bra{\oper{H}\Psi}\\
        &=\frac{1}{\im\hbar}(\oper{H}\oper{\rho}-\oper{\rho}\oper{H}) = \frac{1}{\im\hbar}[\oper{H},\oper{\rho}]\,.
    \end{split}
\end{equation*}

Z powyższego mamy więc dla \(\oper{\rho} = \frac{1}{2}(\oper{I} +
\mathbf{s}\cdot\boldsymbol{\sigma})\) i \(\oper{H} = \mathbf{H}(t)\cdot\boldsymbol{\sigma}\)
\begin{equation*}\boxed{
    \frac{\hbar}{2}\dot{\mathbf{s}} = \mathbf{H}(t)\times\mathbf{s}}
\end{equation*}

\linesep

\textcolor{blue}{\textbf{Oscylacje Rabiego.} Rozważmy układ opisany hamiltonianem
\begin{equation*}
    \oper{H} = -\frac{1}{2}\gamma\hbar\mathbf{B}(t)\boldsymbol{\sigma}\,,
\end{equation*}    
gdzie
\begin{equation*}
    \mathbf{B}(t) = B_0(\sin\beta\cos\Omega t,\sin\beta\sin\Omega t, \cos\beta)\,,
\end{equation*}
dla pewnych stałych \(\gamma\), \(B_0\), \(\beta\), \(\Omega\), który opisuje interakcję spinu z
zewnętrznym polem magnetycznym. Chcemy rozwiązać równanie Schr\"{o}dingera
\begin{equation*}
    \mqty(\dot{\psi}_1 \\ \dot{\psi}_2) = \frac{\im\gamma B_0}{2}\mqty(\cos\beta & \e^{-\im\Omega t}\sin\beta \\ \e^{+\im\Omega t}\sin\beta & \cos\beta) \mqty(\psi_1 \\ \psi_2)\,.
\end{equation*}
Poszukajmy rozwiązań postaci
\begin{equation*}
    \mqty(\psi_1 \\ \psi_2) = \mqty(\phi_1\e^{-\frac{\im}{2}\Omega t} \\ \phi_2 \e^{+\frac{\im}{2}\Omega t})\,,
\end{equation*}
skąd otrzymujemy
\begin{equation*}
    \mqty(\dot{\phi}_1 \\ \dot{\phi}_2) = \frac{\im}{2} \mqty(\Omega +\omega_0\cos\beta & \omega_0\sin\beta \\ \omega_0\sin\beta & -(\Omega +\omega_0\cos\beta)) \mqty(\phi_1 \\ \phi_2)\,,
\end{equation*}
gdzie \(\omega_0 := \gamma B_0\). Wartości własne powyższej macierzy to
\begin{equation*}
    \lambda = \pm\frac{\im}{2}\sqrt{\omega_0^2 + \Omega^2 + 2\omega_0\Omega\cos\beta} = \pm\frac{\im}{2}\Lambda\,,
\end{equation*}
natomiast wektory własne mają postać
\begin{equation*}
    \alpha_\pm \mqty( -\omega_0\sin\beta \\ \Omega + \omega_0\cos\beta \mp \Lambda)\,,
\end{equation*}
dla pewnych niezerowych stałych \(\alpha_+\), \(\alpha_-\). Rozwiązanie na wektor \(\ket{\phi(t)}\)
ma więc postać
\begin{equation*}
    \begin{split}
        \ket{\phi(t)} &= \alpha_+\mqty(-\omega_0\sin\beta \\ \Omega+\omega_0\cos\beta - \Lambda)\e^{\im\Lambda t/2} \\
        &+ \alpha_- \mqty(-\omega_0\sin\beta \\ \Omega + \omega_0\cos\beta+\Lambda)\e^{-\Lambda t/2}\,.
    \end{split}
\end{equation*}
Zakładając, iż w stanie początkowym \(\ket{\Psi(0)} = \ket{2}\) możemy obliczyć prawdopodobieństwo
przejścia do stanu \(\ket{1}\)
\begin{equation*}
    p_{2\to1}(t) = |\braket{1}{\Psi}|^2 = \frac{\omega_0^2\sin^2\beta}{\Lambda^2}\sin^2\left(\frac{\Lambda t}{2}\right)\,.
\end{equation*}
Zauważmy, że prawdopodobieństwo przejścia oscyluje z amplitudą zależną od częstości wymuszenia
\begin{equation*}
    p_\text{max}(\Omega) = \frac{\sin^2\beta}{1 + \left(\frac{\Omega}{\gamma B_0}\right)^2 + 2\left(\frac{\Omega}{\gamma B_0}\right) \cos\beta} \,.
\end{equation*}
Amplituda ta przyjmuje wartość maksymalną dla częstości wymuszenia równej \(|\Omega_\text{rez}| =
|\omega_0 \cos\beta|\). Zauważmy również, iż niezerowa szerokość połówkowa nie wynika z żadnych
procesów dyssypatywnych, jak ma to miejsce w np. w przypadku klasycznego oscylatora harmonicznego z
tłumieniem i wymuszeniem, tylko z samej teorii kwantowej. }\noindent\rule{\columnwidth}{0.5pt}

\newpage
\section{OBLICZENIA KWANTOWE}









\end{document}
