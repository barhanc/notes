\documentclass{myclass}
\begin{document}

\subsubsection*{Szkic rozwiązania zadania 1.}
Niech \(\mathbf{r}_1\), \(\mathbf{r}_2\), \(\mathbf{r}\) oznaczają odpowiednio
wektory wodzące psów i sań względem pewnego układu odniesienia nieporuszającego
się względem ziemi. Ponieważ liny łączące psy z saniami są napięte, więc musi
zachodzić
\begin{equation}
    \begin{split}
        &0=\dv[]{}{t}|\mathbf{r_1}-\mathbf{r}|^2=2(\mathbf{r}_1-\mathbf{r})\cdot (\mathbf{v}_1-\mathbf{V})\\
        &0=\dv[]{}{t}|\mathbf{r_2}-\mathbf{r}|^2=2(\mathbf{r}_2-\mathbf{r})\cdot (\mathbf{v}_2-\mathbf{V})
    \end{split}\quad\,.
\end{equation}
Przyjmijmy układ współrzędnych kartezjańskich \((x,y)\), w którym oś \(x\)
pokrywa się z wektorem \(\mathbf{V}\). Niech \(\eta\), \(\xi\) oznaczają
odpowiednio katy jakie tworzą wektory \(\mathbf{v}_1\), \(\mathbf{v}_2\) z osią
\(x\). Oczywiście \(\eta+\xi=\alpha\). Z powyższego otrzymujemy więc
\begin{equation}
    \begin{split}
        &v_1=V\cos\eta\\
        &v_2=V\cos\xi=V\cos(\alpha-\eta)
    \end{split}\quad\,.
\end{equation}
Z pierwszego równania mamy \(\cos\eta=\frac{v_1}{V}\) i
\(\sin\eta=\frac{\sqrt{V^2-v_1^2}}{V}\), zatem
\begin{equation}
    v_2=v_1\cos\alpha+\sqrt{V^2-v_1^2}\sin\alpha\,.
\end{equation}
Podnosząc stronami do kwadratu i rozwiązując względem \(V\) otrzymujemy
odpowiedź
\begin{equation}
    V=\frac{\sqrt{v_1^2-2v_1v_2\cos\alpha+v_2^2}}{\sin\alpha}\,.
\end{equation}
Oczywiście zakładaliśmy tutaj, iż \(\alpha\neq 0\). Jeśli \(\alpha=0\) to
spełnienie warunku (1) jest możliwe wtedy i tylko wtedy, gdy \(V=v_1=v_2\)
\(\blacksquare\)

\subsubsection*{Szkic rozwiązania zadania 2.}
Przyjmijmy układ współrzędnych kartezjańskich, w którym oś X jest pozioma i
skierowana od drabiny do bloku, a oś Y pionowa i skierowana w dół. Niech
\(\mathbf{F}\) oznacza siłę reakcji działającą na drabinę w miejscu styku
drabiny i bloku. Ponieważ nie występuje tarcie siła ta jest prostopadła do
drabiny. Jeżeli drabina jest nieruchoma to wypadkowy moment siły działający na
nią względem punktu podparcia drabiny o podłoże wynosi 0. Niech \(s\) oznacza
położenie człowieka na drabinie mierząc wzdłuż drabiny. Mamy zatem
\begin{equation}
    F\sqrt{h^2+b^2}=mgs\frac{b}{\sqrt{h^2+b^2}}\,,
\end{equation}
skąd
\begin{equation}
    F=mgs\frac{b}{b^2+h^2}\,.
\end{equation}
Zgodnie z III zasadą dynamiki na blok działa więc siła \(-\mathbf{F}\), której
składowe \(F_x\), \(F_y\) są dane wzorami
\begin{equation}
    \begin{split}
        &F_x=F\frac{h}{\sqrt{h^2+b^2}}=mgs\frac{bh}{(b^2+h^2)^{3/2}}\\
        &F_y=F\frac{b}{\sqrt{h^2+b^2}}=mgs\frac{b^2}{(b^2+h^2)^{3/2}}\,.
    \end{split}
\end{equation}
Aby dla pewnego położenia \(s\) blok zaczął się obracać bez wcześniejszego
przesuwania musi zachodzić
\begin{equation}
    \quad F_xh-F_yb\geq Mg\frac{b}{2}\quad\text{oraz}\quad \mu\geq \frac{F_x}{Mg+F_y}\,.
\end{equation}
Z pierwszego warunku otrzymujemy
\begin{equation}
    ms\frac{h^2-b^2}{(h^2+b^2)^{3/2}}\geq \frac{1}{2}M\,,
\end{equation}
czyli
\begin{equation}
    s\geq \frac{M}{2m}\frac{(h^2+b^2)^{3/2}}{h^2-b^2}\,,
\end{equation}
a zatem najmniejsze położenie \(s\), przy którym blok zacznie się obracać wynosi
\begin{equation}
    s_\text{min}=\frac{M}{2m}\frac{(h^2+b^2)^{3/2}}{h^2-b^2}\,.
\end{equation}
Jednocześnie zauważmy, że zawsze zachodzi \(s\leq\sqrt{h^2+b^2}\), zatem aby
blok mógł dla pewnego położenia \(s\) zacząć się przewracać musi zachodzić
\(s_\text{min}\leq \sqrt{h^2+b^2}\), czyli
\begin{equation}
    \frac{h^2-b^2}{h^2+b^2}\geq \frac{M}{2m}
\end{equation}
i jest to poszukiwany warunek. Z drugiego warunku mamy natomiast
\begin{equation}
\begin{split}
    \mu&\geq \frac{mbh}{\frac{M}{s}(h^2+b^2)^{3/2}+mb^2}\\
    &\geq \frac{mbh}{\frac{M}{s_\text{min}}(h^2+b^2)^{3/2}+mb^2}=\frac{bh}{2h^2-b^2}\,,
\end{split}
\end{equation}
zatem
\begin{equation}
    \mu_\text{min}=\frac{bh}{2h^2-b^2}\quad\blacksquare
\end{equation}
\subsubsection*{Szkic rozwiązania zadania 3.}

Niech \(\phi\), \(\theta\) oznaczają odpowiednio kąty jakie tworzą nić z pionem
i pręt z podłożem. Zakładając, że nić jest cały czas napięta zachodzi
\begin{equation}\label{eq:wiezy1}
    \cos\phi+\sin\theta=\frac{H}{l}\,,
\end{equation}
skąd
\begin{equation}\label{eq:wiezy2}
    \Omega\sin\phi=\omega\cos\theta\,,
\end{equation}
gdzie
\begin{equation}
    \omega:=\dv[]{\theta}{t}\,,\quad\Omega:=\dv[]{\phi}{t}\,.
\end{equation}
Całkowita energia mechaniczna układu jest oczywiście zachowana, więc z tw.
K{\"o}niga i elementarnej geometrii mamy
\begin{equation}
    \frac{mgH}{4}=\frac{1}{2}mv_\text{CM}^2+\frac{1}{2}\frac{1}{12}ml^2\omega^2+\frac{mgl\sin\theta}{2}\,,
\end{equation}
gdzie \(m\) jest masą pręta, a \(v_\text{CM}\) oznacza szybkość środka masy
pręta w dowolnej chwili. Z powyższego
\begin{equation}
    v_\text{CM}^2=\frac{1}{2}gH-gl\sin\theta-\frac{1}{12}l^2\omega^2\,.
\end{equation}
Zauważmy, że \(\omega^2\geq 0\) oraz \(\sin\theta\geq H/l-1\), zatem
\begin{equation}
    v_\text{CM}^2\leq\frac{1}{2}gH-gH+gl=g\left(l-\frac{H}{2}\right)\,,
\end{equation}
czyli wartość \(v_\text{CM}\) jest ograniczona z góry przez
\(v_\text{m}=\sqrt{g\left(l-\frac{H}{2}\right)}\). Pozostaje wykazać, iż
istnieje chwila w trakcie ruchu pręta gdy \(v_\text{CM}=v_\text{m}\). Istotnie,
gdy nić jest pionowa mamy \(\phi=0\), a zatem z równania (\ref{eq:wiezy2})
\(\omega=0\) i (z równania (\ref{eq:wiezy1})) \(\sin\theta=H/l-1\), co kończy
rozwiązanie zadania \(\blacksquare\)

\subsubsection*{Szkic rozwiązania zadania 4.}
\textit{W rozwiązaniu kropka nad daną wielkością oznacza różniczkowanie po
czasie, tj. \(\dot{f}:=\dv[]{f}{t}\), \(\ddot{f}:=\dv[2]{f}{t}\), ...}\\

Niech \(\mathbf{r}=[l\cos\phi,l\sin\phi]\) oznacza wektor wodzący kulki o masie
\(m\) zaczepiony w miejscu zawiasu, \(\phi\) -- kąt między prętem i podłożem, a
\(\mathbf{v}=v\mathbf{\hat{x}}\) -- prędkość klocka. Równanie więzów układu (do
momentu utraty kontaktu między kulką i klockiem) ma postać
\begin{equation}
    v=-\dot{\phi}l\sin\phi\,.
\end{equation}
Niech \(\mathbf{N}\) oznacza siłę reakcji między kulką i klockiem. Ponieważ
zakładamy, że powierzchnie są idealnie gładkie i nie występuje tarcie, więc
\(\mathbf{N}=N\mathbf{\hat{x}}\). Z II zasady dynamiki \(N=M\dot{v}\), więc w
momencie utraty kontaktu zachodzi
\begin{equation}
    \frac{1}{\sqrt{3}}\varepsilon=-\omega^2\,,
\end{equation}
gdzie \(\varepsilon\), \(\omega\) oznaczają odpowiednio \(\ddot{\phi}\) i
\(\dot{\phi}\) w momencie utraty kontaktu. Jednocześnie energia mechaniczna
układu jest zachowana więc w dowolnej chwili 
\begin{equation}\label{energy}
    mgl=\frac{1}{2}(M\sin^2\phi+m)\dot{\phi}^2l^2+mgl\sin\phi\,,
\end{equation}
skąd
\begin{equation}\label{dotp}
    \omega^2=-\frac{1}{\sqrt{3}}\varepsilon=\frac{g}{l}\frac{4m}{M+4m}\,.
\end{equation}
Różniczkując (\ref{energy}) po czasie otrzymujemy natomiast
\begin{equation}
    -\frac{1}{2}Ml^2\dot{\phi}^2\sin{2\phi}-mgl\cos\phi=(M\sin^2\phi+m)l^2\ddot{\phi}\,.
\end{equation}
Z powyższego więc
\begin{equation}
    Ml\varepsilon-2\sqrt{3}mg=\varepsilon l(M+4m)\,.
\end{equation}
Podstawiając do powyższego (\ref{dotp}) otrzymujemy ostatecznie
\begin{equation}
    \frac{M}{m}=4\,.
\end{equation}
Szybkość klocka wynosi natomiast
\begin{equation}
    V=-\omega l\sin\frac{\pi}{6}=\sqrt{\frac{gl}{8}}\,,
\end{equation}
co kończy rozwiązanie zadania \(\blacksquare\)
\subsubsection*{Szkic rozwiązania zadania 5.}
Z definicji ciepła molowego i I zasady termodynamiki dla gazu jednoatomowego
mamy
\begin{equation}
    2R=C=\frac{\delta Q}{n\dd{T}}=\frac{3}{2}R+\frac{p}{n}\dv[]{V}{T}\,,
\end{equation}
skąd, podstawiając \(\frac{p}{n}=\frac{RT}{V}\) otrzymujemy
\begin{equation}
    \frac{1}{2}\frac{\dd{T}}{T}=\frac{\dd{V}}{V}\,.
\end{equation}
Całkując obustronnie powyższe równanie odpowiednio od \(V\) do \(2V\) i od \(T\)
do \(T'\) otrzymujemy
\begin{equation}
    \log\frac{T'}{T}=\log 4\,,
\end{equation}
skąd wnioskujemy, że temperatura wzrosła czterokrotnie \(\blacksquare\)
\subsubsection*{Szkic rozwiązania zadania 6.}
\textit{W rozwiązaniu kładę \(4\pi\epsilon_0=1\).}\\

\begin{figure}[ht]
    \centering
    \begin{tikzpicture}[line cap=round,line join=round,>=triangle 45,x=1cm,y=1cm,scale=0.7]
\clip(-17,8) rectangle (-7,18); \draw
[shift={(-11.790134107302215,13.171364848763671)},line width=0pt] (0,0) --
(123.55925804040555:0.39144749639608173) arc
(123.55925804040555:162.32882180271702:0.39144749639608173) -- cycle; \draw
[line width=0.4pt] (-11.790134107302215,13.171364848763671) circle
(4.236654832859354cm); \draw [line width=0.4pt]
(-15.078357285291432,14.218948455633742)--
(-14.332367964179264,16.560507983796263); \draw [line width=0.4pt]
(-14.332367964179264,16.560507983796263)--
(-10.644735402281466,9.092479464332367); \draw [line width=0.4pt]
(-10.644735402281466,9.092479464332367)--
(-9.57054040469577,12.464232196598486); \draw [line width=0.4pt,dashed]
(-14.13215323265367,16.70183090713026)--
(-10.983038924500262,9.012297460671915); \draw [line width=0.4pt,dashed]
(-10.348392018535602,9.187569683061824)--
(-14.500268036077266,16.427807426315567); \draw [line width=0.4pt,dashed]
(-14.500268036077266,16.427807426315567)--
(-15.192405184725748,14.25528257998549); \draw [line width=0.4pt,dashed]
(-14.937466207955822,14.17406245642328)--
(-14.13215323265367,16.70183090713026); \draw [line width=0.4pt,dashed]
(-10.983038924500262,9.012297460671915)--
(-9.854479906346574,12.554691496704184); \draw [line width=0.4pt,dashed]
(-10.348392018535602,9.187569683061824)--
(-9.329006537384975,12.387282760321643); \draw [line width=0.4pt]
(-15.826879416607003,14.45741752287788)--
(-7.753388797997426,11.885312174649465); \draw [line width=0.4pt]
(-11.790134107302215,13.171364848763671)--
(-14.13215323265367,16.70183090713026);
\begin{scriptsize}
\draw [fill=black] (-11.790134107302215,13.171364848763671) circle (1.5pt);
\draw[color=black] (-11.409055341311891,13.460258566002118) node {$O$}; \draw
[fill=black] (-7.753388797997426,11.885312174649465) circle (1.5pt);
\draw[color=black] (-7.494580377351073,11.803130831258711) node {$B$}; \draw
[fill=black] (-15.826879416607003,14.45741752287788) circle (1.5pt);
\draw[color=black] (-16.30214904626291,14.634601055190359) node {$A$}; \draw
[fill=black] (-12.819817049971508,13.497357527550303) circle (1.5pt);
\draw[color=black] (-12.70083207941896,13.877802562157937) node {$P$}; \draw
[fill=black] (-14.332367964179264,16.560507983796263) circle (1.5pt);
\draw[color=black] (-14.42320106356172,16.878900034527888) node {$X$}; \draw
[fill=black] (-10.644735402281466,9.092479464332367) circle (1.5pt);
\draw[color=black] (-10.600063848759989,8.762888609249154) node {$Y$}; \draw
[fill=black] (-15.078357285291432,14.218948455633742) circle (1.5pt);
\draw[color=black] (-15.114758307194798,13.877802562157937) node {$X'$}; \draw
[fill=black] (-9.57054040469577,12.464232196598486) circle (1.5pt);
\draw[color=black] (-9.373528360052266,12.860039071528128) node {$Y'$}; \draw
[color=black] (-9.329006537384975,12.387282760321643)-- ++(-1.5pt,0 pt) --
++(3pt,0 pt) ++(-1.5pt,-1.5pt) -- ++(0 pt,3pt); \draw [color=black]
(-9.854479906346574,12.554691496704184)-- ++(-1.5pt,0 pt) -- ++(3pt,0 pt)
++(-1.5pt,-1.5pt) -- ++(0 pt,3pt); \draw [color=black]
(-15.192405184725748,14.25528257998549)-- ++(-1.5pt,0 pt) -- ++(3pt,0 pt)
++(-1.5pt,-1.5pt) -- ++(0 pt,3pt); \draw [color=black]
(-14.937466207955822,14.17406245642328)-- ++(-1.5pt,0 pt) -- ++(3pt,0 pt)
++(-1.5pt,-1.5pt) -- ++(0 pt,3pt); \draw[color=black]
(-12.73997682905857,15.20872404990461) node {$R$};
\end{scriptsize}
\end{tikzpicture}
    \caption{Konstrukcja poszukiwanego rozkładu \(\sigma\)}
    \label{fig:discproof}
\end{figure}
Udowodnimy, że rozkład ładunku na powierzchni odosobnionego, metalowego dysku
naładowanego ładunkiem \(Q\) jest taki, że infinitezymalny ładunek
\(\dd{q}=\sigma \dd{\Sigma}\) znajdujący się w dowolnym punkcie \(X'\) na
powierzchni dysku jest równy ładunkowi znajdującemu się w punkcie przecięcia
prostej prostopadłej do płaszczyzny dysku przechodzącej przez \(X'\) z
naładowaną jednorodnie z gęstością powierzchniową \(\sigma_0=Q/4\pi R^2\) sferą,
której kołem wielkim jest ów dysk. Istotnie rozpatrzmy dowolny punkt \(P\) na
kole wielkim \(AB\) naładowanej (z gęstością \(\sigma_0\)) sfery oraz dowolną
cięciwę \(XY\) sfery zawierającą \(P\). Infinitezymalny ładunek zgromadzony w
\(X\) i \(Y\) wynosi odpowiednio
    \begin{equation}
        \dd{q}_X=\sigma_0\dd{\Sigma}_X\,,\quad \dd{q}_Y=\sigma_0\dd{\Sigma}_Y\,,
    \end{equation}
    gdzie \(\dd{\Sigma}_X\), \(\dd{\Sigma}_Y\) są infitezymalnymi fragmentami
    powierzchni sfery otaczającymi odpowiednio punkt \(X\) i \(Y\). W granicy
    zachodzi
    \begin{equation}
        \frac{\dd{\Sigma}_X}{\dd{\Sigma_Y}}=\frac{PX^2}{PY^2}\,.
    \end{equation}
    Rozpatrzmy teraz osobną sytuację, w której mamy cienki, przewodzący dysk
    \(AB\) naładowany w taki sposób, że \(\dd{q}_{X'}=\dd{q}_X\) i
    \(\dd{q}_{Y'}=\dd{q}_Y\), gdzie \(X'\), \(Y'\) są rzutami prostokątnymi
    punktów \(X\), \(Y\) na koło wielkie \(AB\). Oczywiście całkowity ładunek
    dysku wynosi wówczas \(Q\). Pole elektryczne w punkcie \(P\) pochodzące od
    ładunków \(\dd{q}_{X'}\) i \(\dd{q}_{Y'}\) wynosi
    \begin{equation}
        \delta\mathbf{E}(P)=\left(\frac{\sigma_0\dd{\Sigma}_X}{X'P^2}-\frac{\sigma_0\dd{\Sigma}_Y}{Y'P^2}\right)\frac{\overrightarrow{AB}}{AB}\,.
    \end{equation}
Z podobieństwa trójkątów \(\Delta X'XP\) i \(\Delta Y'YP\) mamy
\begin{equation}
        \frac{X'P}{XP}=\frac{Y'P}{YP}\,,
\end{equation}
skąd otrzymujemy
\begin{equation}
\begin{split}
   \frac{\dd{\Sigma_X}}{X'P^2}\div\frac{\dd{\Sigma_Y}}{Y'P^2}=1\,,
\end{split}
\end{equation}
czyli \(\delta\mathbf{E}(P)=\overline{0}\). Powtarzając konstrukcję dla
wszystkich punktów dysku (po obu jego stronach) otrzymujemy
\(\mathbf{E}(P)=\overline{0}\), ale punkt \(P\) wybraliśmy dowolnie, zatem dla
każdego punktu \(P\) należącego do dysku \(\mathbf{E}(P)=\overline{0}\), więc
potencjał wewnątrz dysku jest stały. Zauważmy jednak, iż w ten sposób
znaleźliśmy rozkład ładunku na powierzchni dysku, dla którego potencjał dysku
jest stały. Na mocy tw. o jednoznaczności (dla danej geometrii przewodnika
istnieje dokładnie jeden sposób rozłożenia ładunku \(Q\) na jego powierzchni, w
taki sposób, aby jego powierzchnia była powierzchnią ekwipotencjalną)
skonstruowany rozkład jest zatem poszukiwanym rozkładem dla metalowego dysku.
Pozostaje wyznaczyć \(\sigma(X')\). Mamy
\begin{equation}
    \sigma(X')\dd{\Sigma}_{X'}=\sigma_0\dd{\Sigma}_X
\end{equation}
oraz z prostych rozważań geometrycznych
\begin{equation}
    \dd{\Sigma}_{X'}=\dd{\Sigma}_X\frac{XX'}{R}=\dd{\Sigma}_X\frac{\sqrt{R^2-OX'^2}}{R}\,.
\end{equation}
Z powyższego otrzymujemy zatem
\begin{equation}
    \sigma(X')=\frac{\sigma_0R}{\sqrt{R^2-OX'^2}}=\frac{Q}{4\pi R}\frac{1}{\sqrt{R^2-OX'^2}}
\end{equation}
po każdej stronie dysku. Oznaczając \(OX'=s\) mamy 
\begin{equation}
    \sigma(s)=\frac{Q}{4\pi R}\frac{1}{\sqrt{R^2-s^2}}\,.
\end{equation}
Potencjał dysku \(V_0\) możemy znaleźć wykonując elementarne całkowanie.
Przyjmując \(V(|\mathbf{r}|\to\infty)=0\) mamy
\begin{equation}
    V_0=4\pi\int\limits_0^R\frac{\sigma(s)}{s}s\dd{s}=\frac{\pi Q}{2R}\,.
\end{equation}
Pojemność metalowego dysku o promieniu \(R\) wynosi z powyższego
\begin{equation}
    C_\text{disc}=\frac{Q}{V_0}=\frac{2R}{\pi}\,,
\end{equation}
jednocześnie pojemność metalowej sfery o promieniu \(R\) wynosi oczywiście
\begin{equation}
    C_\text{sphere}=R\,.
\end{equation}
Stosunek tych pojemności jest zatem równy
\begin{equation}
    \frac{C_\text{disc}}{C_\text{sphere}}=\frac{2}{\pi}\,.
\end{equation}
\textit{Ten stosunek został wyznaczony doświadczalnie w XVIII w. przez H.
Cavendisha, który otrzymał wartość \(1\div1.57\)} \(\blacksquare\)
\subsubsection*{Szkic rozwiązania zadania 7.}
\textit{W rozwiązaniu kładę \(4\pi\epsilon_0=1\).}\\

\textit{Udowodnimy najpierw następujący lemat. Niech \(\Gamma'\) oznacza
przewodnik powstały z usunięcia pewnej części sfery o średnicy \(f\). Jeśli na
przewodniku \(\Gamma'\) będzie utrzymywany stały potencjał \(V_0\) to różnica
gęstości powierzchniowych ładunku po zewnętrznej i wewnętrznej stronie wynosi
\(V_0/2\pi f\).}\\

Wykażemy, że \(\Delta\sigma\) jest niezależna od wyboru punktu \(P\). Istotnie
rozpatrzmy przewodnik powstały przez wycięcie pewnej części \(\Gamma\) z bardzo
cienkiego, nieskończonego arkusza metalu danego równaniem \(z=0\). Załóżmy, że
przewodnik \(\Gamma\) został uziemiony, a w punkcie \((0,0,h)\) umieszczono
punktowy ładunek elektryczny \(-q\). Niech \(\sigma_o(P)\) oznacza gęstość
ładunku indukowanego wówczas w \(P\in\Gamma\) po stronie bliższej ładunkowi
\(-q\) (\(z=0^+\)), a \(\sigma_i(P)\) -- analogiczną wielkości po stronie
dalszej (\(z=0^-\)). Jeśli zamiast ładunku \(-q\) w \(O=(0,0,h)\) umieścimy
ładunek \(q\) w \(O'=(0,0,-h)\) to gęstości powierzchniowe wyindukowanego
ładunku w \(P\) wynoszą oczywiście \(-\sigma_i(P)\) dla \(z=0^+\) i
\(-\sigma_o(P)\) dla \(z=0^-\). Dokonując superpozycji obu sytuacji otrzymujemy,
iż dla ładunków \(-q\) i \(q\) umieszczonych odpowiednio w \(O\) i \(O'\)
gęstość ładunku w \(P\in\Gamma\) po stronie \(z=0^+\) wynosi
\(\sigma_o-\sigma_i\). Możemy jednak tą wielkość obliczyć, gdyż dla takiego
ułożenia ładunków \(z=0\) jest powierzchnią ekwipotencjalną, zatem
\(\sigma_o-\sigma_i\) będzie równa gęstości powierzchniowej ładunku indukowanego
na nieskończonej, uziemionej płaszczyźnie \(z=0\) pod wpływem ładunku \(-q\) w
\(O\). Jak łatwo sprawdzić
\begin{equation}
    \sigma_o-\sigma_i=\frac{qh}{2\pi|OP|^3}\,.
\end{equation}
Dokonajmy teraz inwersji przewodnika \(\Gamma\) (gdy po jednej stronie ma
ładunek o gęstości \(\sigma_o\), a po drugiej \(\sigma_i\)) względem sfery o
środku \(O\) i promieniu \(\mathscr{R}\). Obrazem inwersyjnym przewodnika
\(\Gamma\) będzie pewien fragment \(\Gamma'\) sfery o średnicy
\(f=\frac{\mathscr{R}^2}{h}\), na którym utrzymywany jest stały potencjał
\(V_0=\frac{q}{\mathscr{R}}\). Gęstości powierzchniowe, w inwersji, transformują
się jak \(\frac{|OP|^3}{\mathscr{R}^3}\), zatem
\begin{equation}
\begin{split}
    \Delta\sigma'&=\sigma_o'-\sigma_i'=\frac{|OP|^3}{\mathscr{R}^3}(\sigma_o-\sigma_i)\\
    &=\frac{qh}{2\pi\mathscr{R}^3}=\frac{V_0}{2\pi f}\,,
\end{split}
\end{equation}
co kończy dowód lematu \(\square\)

\noindent\rule[0.5ex]{\linewidth}{1pt}

Niech \(C\) oznacza pojemność rozpatrywanego przewodnika. Niech \(Q_i\), \(Q_o\)
oznaczają odpowiednio ładunki zgromadzone na wewnętrznej i zewnętrznej
powierzchni półsfery, gdy jej potencjał wynosi \(V_0\) względem punktu w
nieskończoności. Zachodzi
\begin{equation}
    Q_o+Q_i=CV_0\,.
\end{equation}
Jednocześnie, korzystając z udowodnionego lematu, mamy
\begin{equation}
    Q_o-Q_i=\Delta\sigma\cdot2\pi R^2=\frac{V_0R}{2}\,.
\end{equation}
Wyznaczmy stosunek \(Q_o/Q_i\)
\begin{equation}
    \frac{Q_o}{Q_i}=\frac{2C+R}{2C-R}\,.
\end{equation}
Ten stosunek jest dodatni, gdyż gęstość powierzchniowa ładunku indukowanego na
odosobnionym przewodniku \(\Gamma\) ma w każdym punkcie taki sam znak lub jest
zerowa, tj.
\(\forall\,p\in\Gamma:\text{sgn}(\sigma(p))=\text{sgn}(V_0)\,\lor\,\sigma(p)=0\).
Istotnie w przeciwnym przypadku w przewodniku płynąłby prąd\footnote{Fakt ten
można uzasadnić ściślej korzystając z \textit{zasady maksimum} dla funkcji
harmonicznych (tj. funkcji spełniających równanie Laplace'a). Istotnie z zasady
maksimum funkcja \(f\) harmoniczna w obszarze \(D\subset\mathbb{R}^3\) może
osiągać wartości ekstremalne tylko na brzegu \(\partial D\) tego obszaru.
Rozpatrzmy zatem dowolną krzywą gładką \(l\) łączącą pewien punkt \(p\) na
powierzchni odosobnionego naładowanego przewodnika \(\Gamma\) z punktem w
nieskończoności, która jest normalna do powierzchni przewodnika w \(p\). Z
powyższego twierdzenia pochodna \(V\) wzdłuż tej krzywej nie może zmieniać
znaku, gdyż \(V\) osiąga wartości ekstremalne odpowiednio na powierzchni
przewodnika i w punkcie w nieskończoności. W takim razie \(\forall p\in\Gamma:
\text{sgn}\left(\pdv{V}{\mathbf{l}}\big|_p\right)=-\text{sgn}(V_0)\,\lor\,\pdv{V}{\mathbf{l}}\big|_p=0\),
gdzie \(V_0\) jest potencjałem przewodnika względem nieskończoności, ale
jednocześnie \(4\pi\sigma(p)=-\pdv{V}{\mathbf{l}}\big|_p\), skąd otrzymujemy
tezę. }. W takim razie z powyższego otrzymujemy\footnote{Dokładna wartość
poszukiwanej pojemności wynosi \(C=R\left(\frac{1}{2}+\frac{1}{\pi}\right)\).
Patrz np.
\href{https://www.nonstopsystems.com/radio/pdf-ant/antenna-article-calc-capacitance.pdf}{Yu.
Ya. Iossel', E. S. Kochanov and M. G. Strunskiy, \textit{The Calculation of
Electrical Capacitance}, s.163} }
\begin{equation*}
    C>\frac{R}{2}\quad\quad\blacksquare
\end{equation*}




\subsubsection*{Szkic rozwiązania zadania 8.}
\textit{W rozwiązaniu kropka nad daną wielkością oznacza różniczkowanie po
czasie, tj. \(\dot{f}:=\dv[]{f}{t}\), \(\ddot{f}:=\dv[2]{f}{t}\), ...}\\

Niech \(\mathbf{r}=[x,y,z]\) oznacza wektor wodzący cząstki. Ponieważ ruch
cząstki jest ograniczony do powierzchni sfery o promieniu \(l\), więc dogodnym
wyborem współrzędnych, w których będziemy opisywać ruch są katy \(\theta\),
\(\phi\) zdefiniowane jako
\begin{equation}
        x=l\sin\theta\cos\phi\,,\quad y=l\sin\theta\sin\phi\,,\quad z=l\cos\theta\,.
\end{equation}
Do rozwiązania zadania wystarczy nam znajomość dwóch całek ruchu dla powyższego
układu. Istotnie, ponieważ składowa magnetyczna siły Lorentza nie wykonuje
pracy, więc w dowolnym momencie ruchu
\begin{equation}
    \frac{m}{2}(\dot{x}^2+\dot{y}^2+\dot{z}^2)=\frac{ml^2}{2}(\dot{\theta}^2+\dot{\phi}^2\sin^2\theta)=\text{const.}
\end{equation}
Jednocześnie, zgodnie z drugą zasadą dynamiki dla ruchu obrotowego i wzorem na
siłę Lorentza mamy
\begin{equation}
    \dv[]{\mathbf{J}}{t}=qB\mathbf{r}\times\dot{\mathbf{r}}\times\mathbf{\hat{z}}=qB\dot{\mathbf{r}}(\mathbf{r}\cdot\mathbf{\hat{z}})=qBz\dot{\mathbf{r}}\,,
\end{equation}
gdzie skorzystaliśmy z reguły BAC--CAB i faktu, że
\(0=\dv[]{(\mathbf{r}\cdot\mathbf{r})}{t}=2\mathbf{r}\cdot\dot{\mathbf{r}}\). Z
powyższego mamy zatem
\begin{equation}
    \dv[]{}{t}\left(J_z-\frac{qBz^2}{2}\right)=0\,,
\end{equation}
gdzie
\begin{equation}
    J_z=[\mathbf{r}\times m\dot{\mathbf{r}}]_z=ml^2\dot{\phi}\sin^2\theta\,.
\end{equation}
Otrzymaliśmy zatem drugą całkę ruchu
\begin{equation}
    ml^2\dot{\phi}\sin^2\theta-\frac{qBz^2}{2}=\text{const.}
\end{equation}
Oznaczmy powyższe całki ruchu odpowiednio przez \(\mathscr{E}\) i
\(\mathscr{J}\). Korzystając z warunków początkowych mamy oczywiście
\(\mathscr{E}=\frac{1}{2}mv^2\) i \(\mathscr{J}=0\). Jednocześnie przypomnijmy,
że \(z=l\cos\theta\) i \(\dot{z}=-\dot{\theta}l\sin\theta\). Z powyższego mamy
\begin{equation}
    \dot{\phi}=\frac{qB}{2ml^2}\frac{z^2}{\sin^2\theta}\,,
\end{equation}
co po podstawieniu daje
\begin{equation}
    \dot{z}^2l^2=v^2l^2-v^2z^2-\left(\frac{qB}{2m}\right)^2z^4\,.
\end{equation}
Maksymalne \(z\) musi odpowiadać \(\dot{z}=0\), zatem
\begin{equation}
    z_\text{m}=\frac{m\sqrt{2}}{qB}\sqrt{-v^2+\sqrt{v^4+\left(\frac{qBlv}{m}\right)^2}}\,.
\end{equation}
Zauważmy, że \(z_\text{m}\leq l\). Istotnie,
\(v^4+\frac{q^2B^2l^2v^2}{m^2}\leq\left(v^2+\frac{q^2B^2l^2}{2m^2}\right)^2\),
przy czym równość zachodzi tylko dla \(qBl=0\) \(\blacksquare\)

\subsubsection*{Szkic rozwiązania zadania 9.}
Siła elektryczna działająca na dielektryk umieszczony wewnątrz kondensatora o
pojemności \(C\), na którym utrzymywane jest stałe napięcie \(U\) jest dana
wzorem
\begin{equation}
    F_\text{e}=\frac{1}{2}U^2\dv{C}{x}\,.
\end{equation}
Rozpatrzmy zatem układ opisany w treści zadania w momencie, w którym ciecz jest
na wysokości \(x\) licząc od dolnych krawędzi kondensatora. Podzielmy obszar
wewnątrz kondensatora na dwie części: 1 - obszar, w którym nie występuje ciecz;
2 - obszar, w którym występuje ciecz o gęstości \(\rho\) i względnej
przenikalności elektrycznej \(\epsilon_r\). Natężenie pola elektrycznego
pomiędzy okładkami wynosi \(E=U/d\) i nie zależy od rozpatrywanego obszaru.
Pojemność układu jest równa
\begin{equation}
    C=\epsilon_0(\epsilon_r-1)\frac{a}{d}x+\epsilon_0\frac{a^2}{d}\,.
\end{equation}
Korzystając zatem ze wzoru na siłę elektryczną działającą na dielektryk mamy
\begin{equation}
    F_\text{e}=\frac{1}{2}U^2\dv{C}{x}=\epsilon_0(\epsilon_r-1)\frac{U^2a}{2d}\,.
\end{equation}
Jeżeli ciecz sięga górnych krawędzi okładek, siła ciężkości działająca na ciecz
wynosi \(F_\text{c}=\rho a^2gd\). Z warunków równowagi musi zachodzić
\(F_\text{e}=F_\text{c}\), zatem
\begin{equation}
    \rho a^2gd=\epsilon_0(\epsilon_r-1)\frac{U^2a}{2d}\,,
\end{equation}
skąd otrzymujemy
\begin{equation}
    \epsilon_r=1+\frac{2\rho agd^2}{\epsilon_0U^2}\,,
\end{equation}
co stanowi rozwiązanie zadania \(\blacksquare\)

\subsubsection*{Szkic rozwiązania zadania 10.}
W najwyższym punkcie trajektorii składowa y-owa prędkości pocisku wynosi 0.
Niech \((v_{10})_y\), \((v_{20})_y\), \((v_{30})_y\) oznaczają y-owe składowe
prędkości początkowych odpowiednich fragmentów w momencie wybuchu. Z zasady
zachowania pędu mamy
\begin{equation}
    (v_{10})_y+(v_{20})_y+(v_{30})_y=0\,.
\end{equation}
Zależność \(y(t)\) danego fragmentu ma postać
\begin{equation}
    y_i(t)=h_0+(v_{i0})_yt-\frac{1}{2}gt^2\quad(i=1,2,3)\,,
\end{equation}
gdzie \(h_0\) oznacza szukaną wysokość początkową, będącą jednocześnie
wysokością, na jakiej eksplodował pocisk. Ponieważ 2 z 3 fragmentów spadły po
takim samym czasie, ich składowe y-owe prędkości początkowych musiały być takie
same. Bez straty ogólności możemy przyjąć \((v_{10})_y=(v_{20})_y\). Równanie
(5) ma zatem postać \(2(v_{10})_y=-(v_{30})_y\). Mamy zatem dwie zależności:
\begin{equation}
    \begin{cases}
    y_1(t)=y_2(t)=h_0+(v_{10})_yt-\frac{1}{2}gt^2\\
    y_3(t)=h_0-2(v_{10})_yt-\frac{1}{2}gt^2
    \end{cases}\,.
\end{equation}
Niech \(\tau\) oznacza czas, po którym spadł pierwszy fragment. Z treści zadania
mamy zatem
\begin{equation}
    \begin{cases}
    y_1(2\tau)=0=h_0+2(v_{10})_y\tau-2g\tau^2\\
    y_3(\tau)=0=h_0-2(v_{10})_y\tau-\frac{1}{2}g\tau^2
    \end{cases} \,,
\end{equation}
skąd otrzymujemy
\begin{equation*}
    h_0=\frac{5}{4}g\tau^2\,,
\end{equation*}
co stanowi rozwiązanie zadania \(\blacksquare\)

\subsubsection*{Szkic rozwiązania zadania 11.}
1) W przypadku, gdy powierzchnia stożka jest pomalowana na czarno światło jest
całkowicie pochłaniane. Niech \(\Delta F\) oznacza siłę od promieniowania
działającą na niewielką powierzchnię stożka. Niech \(p\) oznacza całkowity pęd
fotonów uderzających w tą niewielką powierzchnię. Z II zasady dynamiki Newtona
mamy
\begin{equation}
    \Delta F=\frac{\Delta p}{\Delta t}=\frac{p}{\Delta t}\,.
\end{equation}
Niech \(n\) oznacza całkowitą liczbę fotonów uderzających w stożek. Całkowita
siła działająca na stożek w przypadku czarnej powierzchni wynosi więc
\begin{equation}
    F_1=n\Delta F=\frac{np}{\Delta t}\,.
\end{equation}
\indent 2) W przypadku drugim światło jest całkowicie odbijane. Niech \(\Delta
F_\perp\) oznacza siłę od promieniowania działającą na niewielką powierzchnię
stożka i prostopadłą do niej. Z II zasady dynamiki mamy
\begin{equation}
    \Delta F_\perp=\frac{\Delta p_\perp}{\Delta t}=\frac{2p_\perp}{\Delta t}\,,
\end{equation}
gdzie \(p_\perp\) -- składowa całkowitego pędu fotonów uderzających w niewielką
powierzchnię stożka, prostopadła do powierzchni bocznej stożka. Z zależności
geometrycznych \(p_\perp=p\sin(\theta/2)\), gdzie \(\theta\) jest kątem
rozwarcia stożka. Ponieważ na przyspieszenie stożka ma wpływ jedynie składowa
siły równoległa do osi stożka i ponieważ, ponownie z zależności geometrycznych,
\((\Delta F_\perp)_\text{oś}=\Delta F_\perp\sin(\theta/2)\), zatem całkowita
siła od promieniowania działająca na stożek w przypadku powierzchni odbijającej
światło wynosi
\begin{equation}
    F_2=\frac{2np\sin^2(\theta/2)}{\Delta t}\,.
\end{equation}
Rozpatrzmy stosunek
\begin{equation}
    \frac{F_2}{F_1}=\frac{a_2}{a_1}=0.7=\frac{2np\sin^2(\theta/2)/\Delta t}{np/\Delta t}=2\sin^2(\theta/2)\,,
\end{equation}
skąd otrzymujemy odpowiedź.
\begin{equation*}
    \theta=2\arcsin\frac{\sqrt{7}}{2\sqrt{5}}\approx72.5^\circ\,,
\end{equation*}
co stanowi rozwiązanie zadania \(\blacksquare\)

\subsubsection*{Szkic rozwiązania zadania 12.}
Zauważmy, że kula o indeksie 4 będzie znajdować się na symetralnej odcinka
\(L\). Ze względu na symetrię możemy zatem rozpatrywać tylko kule po jednej
stronie, gdyż wszystko będzie analogiczne po stronie drugiej. Ponumerujmy więc
kule po jednej stronie od 1 do 4, gdzie kula o indeksie 1 jest kulą najniżej
położoną, a kula o indeksie 4 skrajną, unieruchomioną kulą. Na kulę o indeksie
\(1<i<4\) działają następujące siły: \(m\mathbf{g}\), \(\mathbf{R}_{(i-1)i}\) i
\(\mathbf{R}_{(i+1)i}\), gdzie \(\mathbf{R}_{mn}\) oznacza siłę reakcji z jaką
kula o indeksie \(m\) działa na kulę o indeksie \(n\). Oczywiście z III zasady
dynamiki zachodzi \(|\mathbf{R}_{mn}|=|\mathbf{R}_{nm}|\). Niech \(\alpha_i\)
oznacza kąt jaki tworzy pręt między kulami \(i\), \(i+1\) z poziomem. Z treści
zadania mamy \(\alpha_1=30^\circ\). Wyznaczymy \(\alpha_2\) i \(\alpha_3\).
Rozpatrzmy kulę o indeksie 1. Z warunków równowagi dla tej kuli mamy
\begin{equation}
    mg=2R_{21}\sin 30^\circ =R_{21}\,.
\end{equation}
Rozpatrzmy kulę o indeksie \(1<i<4\). Z warunków równowagi
\begin{equation}
\begin{split}
   &R_{(i-1)i}\cos\alpha_{i-1}=R_{i(i+1)}\cos\alpha_i\,,\\
    &R_{(i-1)i}\sin\alpha_{i-1}+mg=R_{i(i+1)}\sin\alpha_i\,.
\end{split}
\end{equation}
Łatwo sprawdzić, że warunek zerowania się momentu siły (np. względem osi
przechodzącej przez \(i+1\) kulę i prostopadłej do płaszczyzny układu) nie
wprowadza żadnego dodatkowego ograniczenia.\\
Dla \(i=2\) mamy zatem
\begin{equation}
\begin{split}
    R_{12}\cos30^\circ=mg\cos30^\circ=R_{23}\cos\alpha_2\\
    R_{12}\sin30^\circ+mg=R_{23}\sin\alpha_2\,,
\end{split}
\end{equation}
eliminując \(R_{23}\) z tych równań otrzymujemy
\begin{equation}
    \tan \alpha_2=\frac{1+\sin30^\circ}{\cos 30^\circ}=\sqrt{3}\,,
\end{equation}
skąd \(\alpha_2=60^\circ\) oraz \(R_{23}=\sqrt{3}mg\). Analogicznie dla \(i=3\)
otrzymujemy
\begin{equation}
    \begin{split}
        R_{23}\cos\alpha_2=R_{34}\cos\alpha_3\\
        R_{23}\sin\alpha_2+mg=R_{34}\sin\alpha_3\,,
    \end{split}
\end{equation}
skąd otrzymujemy
\begin{equation}
    \tan\alpha_3=\frac{5}{\sqrt{3}}\,,
\end{equation}
a zatem \(\cos\alpha_3=\sqrt{3}/2\sqrt{7}\). Ostatecznie zatem
\begin{equation}
\begin{split}
    L&=2\cdot l(\cos\alpha_1+\cos\alpha_2+\cos\alpha_3)\\
    &=\frac{\sqrt{3}+\sqrt{7}+\sqrt{21}}{\sqrt{7}}l
\end{split}
\end{equation}
co stanowi rozwiązanie zadania \(\blacksquare\)

\subsubsection*{Szkic rozwiązania zadania 13.}
Układ możemy traktować jako pętlę o rezystancji \(\mathcal{R}\approx2\rho
R/r^2\) i pojemności \(\mathcal{C}\approx\epsilon_0\pi r^2/d\). Zgodnie z regułą
strumienia SEM indukowana w pętli wynosi
\begin{equation}
    \mathcal{E}=|-\alpha\pi R^2\dot{z}|=\alpha \pi R^2v\,,
\end{equation}
gdzie \(v\) jest szybkością pętli w danej chwili. Niech \(I\) oznacza natężenie
prądu płynącego w chwili \(t\) w pętli. Zachodzi
\begin{equation}
    \alpha \pi R^2 v(t)=I\mathcal{R}+\frac{1}{\mathcal{C}}Q\,,
\end{equation}
różniczkując po czasie otrzymujemy
\begin{equation}\label{eq:1}
    \alpha\pi R^2 a(t)=\dot{I}\mathcal{R}+\frac{1}{\mathcal{C}}I\,.
\end{equation}
Siła magnetyczna działająca na pętlę wynosi 
\begin{equation}
    F_m=2\pi RI(t)B_s(R)\,,
\end{equation}
gdzie \(B_s\) jest składową radialną indukcji pola magnetycznego. Można ją w
prosty sposób wyznaczyć korzystając z faktu, że dla pola magnetycznego całkowity
strumień przez powierzchnię zamkniętą wynosi zero \(\Phi_\text{tot}=0\).
Istotnie rozpatrzmy powłokę walcową zakreślaną przez spadającą pętlę w czasie
\(\dd{t}\) o wysokości \(v\dd{t}=\dd{z}\) i promieniu \(R\). Całkowity strumień
przez tą powierzchnię zamkniętą wynosi
\begin{equation}
    \Phi_\text{tot}=[B_z(z+\dd{z})-B_z(z)]\pi R^2-B_s(R)\cdot 2\pi R\dd{z}=0\,,
\end{equation}
skąd \(B_s(R)=\alpha R/2\). Z drugiej zasady dynamiki mamy zatem
\begin{equation}\label{eq:2}
    ma(t)=mg-\pi\alpha R^2I(t)\,.
\end{equation}
Korzystając z (\ref{eq:1}) otrzymujemy zatem
\begin{equation}
    \frac{\mathcal{R}}{\alpha \pi R^2}\dot{I}+\frac{1}{\mathcal{C}\alpha\pi R^2}I=g-\frac{\pi\alpha R^2}{m}I\,.
\end{equation}
Jest to niejednorodne liniowe równanie różniczkowe postaci
\begin{equation}
    A_1\dot{I}+B_1I+C_1=0\,,
\end{equation}
gdzie
\begin{equation}
    A_1=\frac{\mathcal{R}}{\alpha \pi R^2}\,,\quad B_1=\frac{\pi\alpha R^2}{m}+\frac{1}{\mathcal{C}\alpha\pi R^2}\,,\quad C_1=-g\,.
\end{equation}
Szkic rozwiązania z uwzględnieniem warunku brzegowego \(I(0)=0\) ma postać 
\begin{equation}
    I(t)=-\frac{C_1}{B_1}\left(1-e^{-\frac{B_1}{A_1}t}\right)\,.
\end{equation}
Z (\ref{eq:2}) otrzymujemy zatem
\begin{equation}
    a(t)=g+\frac{\pi\alpha R^2}{m}\frac{C_1}{B_1}\left(1-e^{-\frac{B_1}{A_1}t}\right)=A+Be^{Ct}\,,
\end{equation}
gdzie poszukiwane stałe są dane wzorami
\begin{equation}
    \begin{split}
        &A=g-B\\
        &B=\frac{\alpha^2\pi^2 R^4\mathcal{C}g}{\alpha^2 \pi^2 R^4\mathcal{C}+m}\\
        &C=-\frac{\alpha^2\pi^2R^4\mathcal{C}+m}{m\mathcal{RC}}\,,
    \end{split}
\end{equation}
gdzie \(\mathcal{R}=2\rho R/r^2\) i \(\mathcal{C}=\epsilon_0\pi r^2/d\)
\(\blacksquare\)

\subsubsection*{Szkic rozwiązania zadania 16.}
Wskazania woltomierza nie są równe rzeczywistym napięciom na rezystorach ze
względu na obecność wewnętrznej rezystancji miernika. Nie wpływa ona na pomiar
SEM idealnego ogniwa, a zatem \(\mathcal{E}=6\,\text{V}\), natomiast wpływa na
pomiar napięcia na rezystorach, gdyż zmienia natężenie prądu pobieranego ze
źródła. Niech \(V_\alpha\) (\(\alpha=1,2\)) oznacza napięcie zmierzone na
rezystorze \(R_\alpha\). Bez straty ogólności możemy przyjąć \(V_1=2\,\text{V}\)
i \(V_2=3\,\text{V}\). Niech \(I\), \(I'\) oznaczają odpowiednio natężenie prądu
pobieranego ze źródła w przypadku podłączenia woltomierza do \(R_1\) lub
\(R_2\). Z napięciowego prawa Kirchhoffa
\begin{equation}
    \mathcal{E}=V_1+IR_2=V_2+I'R_1\,.
\end{equation}
Niech \(R_0\) oznacza opór wewnętrzny miernika. Z prawa Ohma mamy
\begin{equation}
\begin{split}
    I=V_1\left(\frac{1}{R_1}+\frac{1}{R_0}\right)\,,\\
    I'=V_2\left(\frac{1}{R_2}+\frac{1}{R_0}\right)\,.
\end{split}
\end{equation}
Łącząc powyższe równania otrzymujemy
\begin{equation}
\begin{split}
    &V_1\left(\frac{1}{R_1}+\frac{1}{R_0}\right)=\frac{\mathcal{E}-V_1}{R_2}\,,\\
    &V_2\left(\frac{1}{R_2}+\frac{1}{R_0}\right)=\frac{\mathcal{E}-V_2}{R_1}\,.
\end{split}
\end{equation}
Dzieląc odpowiednio przez \(V_1\) i \(V_2\) i odejmując stronami otrzymujemy
\begin{equation}
    \frac{R_2-R_1}{R_1R_2}=\frac{\mathcal{E}V_2R_1-V_1V_2R_1-\mathcal{E}V_1R_2+V_1V_2R_2}{V_1V_2R_1R_2}\,.
\end{equation}
Po przekształceniach otrzymujemy
\begin{equation}
    \frac{V_2}{V_1}=\frac{R_2}{R_1}\,.
\end{equation}
Rzeczywiste napięcia \(U_\alpha\) na rezystorach są więc dane wzorami
\begin{equation}
\begin{split}
   U_1=\frac{\mathcal{E}}{1+\frac{R_2}{R_1}}=\frac{\mathcal{E}V_1}{V_1+V_2}=2.4\,\text{V}\,,\\
    U_2=\frac{\mathcal{E}}{1+\frac{R_1}{R_2}}=\frac{\mathcal{E}V_2}{V_1+V_2}=3.6\,\text{V}\,,
\end{split}
\end{equation}
co kończy rozwiązanie zadania \(\blacksquare\)

\subsubsection*{Szkic rozwiązania zadania 28.}
\textit{Problem jest dość typowy. Można go znaleźć w wielu podręcznikach akademickich do podstaw elektrodynamiki (patrz np. D.J. Griffiths ,,Podstawy elektrodynamiki'') jako pokaz wykorzystania magnetycznego potencjału skalarnego (sic!). Poniżej prezentujemy elementarne rozwiązania będące w zasięgu wiedzy licealnej.}

\begin{figure}[h]
    \centering
    \includegraphics[scale=0.4]{figs/owwwowoowow (1).png}
    \label{fig:magneticrod}
\end{figure}

Przede wszystkim stwierdzamy, że jednorodna magnetyzacja
\(\mathbf{M}=M\mathbf{\hat{x}}\) odpowiada prądowi powierzchniowemu
\(\mathbf{K}(\theta)=M\sin\theta\mathbf{\hat{z}}\) płynącemu po powierzchni
cylindra (to powinno zostać umieszczone jako \textit{wskazówka} wraz z definicją
prądu powierzchniowego), gdzie \(\theta\) jest kątem między \(\mathbf{r}\) i
\(\mathbf{\hat{x}}\). Jednocześnie ponieważ cylinder jest bardzo długi, więc
możemy traktować problem jako dwuwymiarowy (tj. tak jakby cylinder był
nieskończony). Rozważmy układ w płaszczyźnie \(xy\). Skorzystamy z faktu, iż
pole magnetyczne wytwarzane przez nieskończony prostoliniowy przewód, przez
który płynie prąd o natężeniu \(I=K\dd{l}\) wynosi
\begin{equation}
    \mathsf{B}=\frac{\mu_0K\dd{l}}{2\pi s}\,,
\end{equation}
gdzie \(s\) jest odległością od przewodu. Niech \(P\) będzie dowolnym punktem
wewnątrz cylindra w rozważanej płaszczyźnie. Poprowadźmy proste prostopadłe
\(AC\) i \(BD\) przechodzące przez \(P\) oraz zakreślmy niewielkie łuki o
środkach \(A\), \(B\), \(C\), \(D\) takie, że kąty o wierzchołku \(P\)
wyznaczone przez ich końce są sobie równe i wynoszą \(\dd{\omega}\). Niech
\(\theta_A\), \(\theta_B\), \(\theta_C\), \(\theta_D\) oznaczają kąty skierowane
między osią \(x\) (kierunkiem magnetyzacji), a odcinkami odpowiednio \(OA\),
\(OB\), \(OC\) i \(OD\), gdzie \(O\) jest środkiem okręgu. Przyczynek do
całkowitego pola magnetycznego od pary punktów \(A\), \(C\) wynosi
\begin{equation}
    \dd{\mathsf{B}_1}=\frac{\mu_0M}{2\pi}\left(\frac{\dd{l_A}\sin\theta_A}{AP}-\frac{\dd{l_C}\sin\theta_C}{CP}\right)\,,
\end{equation}
gdzie \(\dd{l}\) oznacza długość zakreślonego łuku wokół danego punktu.
Przyczynek ten jest skierowany wzdłuż \(BD\). Analogicznie przyczynek od pary
\(B\), \(D\) wynosi
\begin{equation}
    \dd{\mathsf{B}_2}=\frac{\mu_0M}{2\pi}\left(\frac{\dd{l_B}\sin\theta_B}{BP}-\frac{\dd{l_D}\sin\theta_D}{DP}\right)
\end{equation}
i jest skierowany wzdłuż \(AC\). Z założenia o równości kątów wyznaczonych przez
zaznaczone łuki mamy jednak
\begin{equation}
\begin{split}
    &\frac{\dd{l_A}}{AP}=\frac{\dd{l_C}}{CP}=\frac{\dd{\omega}}{\cos\arcangle OAC}\,,\\ &\frac{\dd{l_B}}{BP}=\frac{\dd{l_D}}{DP}=\frac{\dd{\omega}}{\cos\arcangle OBD}\,,    
\end{split}
\end{equation}
przy czym oczywiście \(\cos\arcangle OBD=\sin(\frac{\theta_D-\theta_B}{2})\) i
\(\cos\arcangle OAC=\sin(\frac{\theta_C-\theta_A}{2})\). Z powyższego zatem
\begin{equation}
\begin{split}
    &\dd{\mathsf{B}_1}=\frac{\mu_0M\dd{\omega}}{2\pi}\frac{\sin\theta_A-\sin\theta_C}{\sin\frac{\theta_C-\theta_A}{2}}\,,\\
    &\dd{\mathsf{B}_2}=\frac{\mu_0M\dd{\omega}}{2\pi}\frac{\sin\theta_B-\sin\theta_D}{\sin\frac{\theta_D-\theta_B}{2}}\,.
\end{split}
\end{equation}
Udowodnimy, że wypadkowa tych przyczynków jest skierowana równolegle do osi
\(x\). Istotnie niech \(\gamma\) oznacza kąt skierowany między prostą równoległą
do \(x\) przechodzącą przez \(P\), a prostą \(AC\). Udowodnimy, że
\begin{equation}
    \frac{\dd{\mathsf{B}_1}}{\dd{\mathsf{B}_2}}=\frac{\sin\theta_A-\sin\theta_C}{\sin\theta_B-\sin\theta_D}\frac{\sin\frac{\theta_D-\theta_B}{2}}{\sin\frac{\theta_C-\theta_A}{2}}=\tan\gamma\,.
\end{equation}
Istotnie łatwo pokazać (np. korzystając z liczb zespolonych), że
\begin{equation}
    \tan\gamma=\frac{\sin\theta_C-\sin\theta_A}{\cos\theta_C-\cos\theta_A}=-\frac{\cos\theta_D-\cos\theta_B}{\sin\theta_D-\sin\theta_B}\,,
\end{equation}
z powyższego po prostych przekształceniach mamy zatem
\begin{equation}
    \tan\gamma=\tan\frac{\theta_B+\theta_D}{2}
\end{equation}
oraz
\begin{equation}
    \theta_A+\theta_C=\theta_B+\theta_D+(2k-1)\pi\,,\quad k\in\mathbb{Z}\,.
\end{equation}
Jednocześnie po prostych przekształceniach
\begin{equation}
    \frac{\sin\theta_A-\sin\theta_C}{\sin\theta_B-\sin\theta_D}\frac{\sin\frac{\theta_D-\theta_B}{2}}{\sin\frac{\theta_C-\theta_A}{2}}=\frac{\cos\frac{\theta_A+\theta_C}{2}}{\cos\frac{\theta_B+\theta_D}{2}}\,,
\end{equation}
czyli z powyższego
\begin{equation}
    \frac{\sin\theta_A-\sin\theta_C}{\sin\theta_B-\sin\theta_D}\frac{\sin\frac{\theta_D-\theta_B}{2}}{\sin\frac{\theta_C-\theta_A}{2}}=\tan\gamma\quad\square
\end{equation}
W takim razie przyczynek od czwórki punktów \(A\), \(B\), \(C\), \(D\) jest
skierowany zgodnie z wektorem magnetyzacji i wynosi
\begin{equation}
    \dd{\mathsf{B}}=\sqrt{\dd{\mathsf{B}_1}^2+\dd{\mathsf{B}_2}^2}=\frac{\mu_0M\dd{\omega}}{2\pi}\cdot 2=\frac{\mu_0M}{\pi}\dd{\omega}\,.
\end{equation}
Całkowita indukcja pola magnetycznego w punkcie \(P\) wynosi zatem
\begin{equation}
    \boldsymbol{\mathsf{B}}=\frac{\mu_0\mathbf{M}}{\pi}\cdot\frac{\pi}{2}=\frac{1}{2}\mu_0\mathbf{M}\,,
\end{equation}
czyli wewnątrz cylindra pole magnetyczne jest jednorodne \(\blacksquare\)

\subsubsection*{Szkic rozwiązania zadania 29.}
Pole elektryczne światła odbitego od płytki będzie superpozycją pól
elektrycznych promieni świetlnych odbitych kolejno od górnej i dolnej
powierzchni płytki. Ponieważ drgania pola elektrycznego są koherentne, zatem aby
obliczyć wypadkowe natężenie światła musimy najpierw obliczyć wypadkowe pole
elektryczne (nie możemy dodawać natężeń światła). Ponieważ \(I\sim E^2\), więc
amplituda pola elektrycznego promienia po przejściu przez granicę
powietrze-szkło wyniesie \(\sqrt{r}E_0\), a promienia odbitego
\(\sqrt{1-r}E_0\). Wprowadźmy parametr \(l=\sqrt{1-r}\). Pole elektryczne
promienia odbitego od górnej powierzchni od strony powietrza wynosi
\begin{equation}
    lE_0\sin(-\omega x/c+\omega t+\pi)=lE_0\sin(\phi+\pi)\,.
\end{equation}
Pole elektryczne \(n\)-tego promienia wychodzącego z płytki po górnej stronie po
\(n\) odbiciach od dolnej powierzchni płytki wynosi
\begin{equation}
    E_n=rl^{2n-1}E_0\sin(\phi+n\delta)\,,
\end{equation}
gdzie \(\delta=\frac{4\pi}{\lambda}h\) (\(h\) - grubość płytki). Istotnie
amplituda pola elektrycznego promienia po przejściu przez granicę
powietrze-szkło wynosi \(\sqrt{r}E_0\), każde odbicie mnoży ją przez czynnik
\(l\), a odbić tych jest zawsze nieparzyście wiele, dodatkowo przy przejściu
ponownym przez granicę szkło-powietrze należy ją pomnożyć przez czynnik
\(\sqrt{r}\). Wypadkowe pole elektryczne światła odbitego od płytki wynosi więc
\begin{equation}
\begin{split}
    E_\text{tot}&=lE_0\sin(\phi+\pi)+\\
    &+rE_0\left[l\sin(\phi+\delta)+l^3\sin(\phi+2\delta)+...\right]\,.
\end{split}
\end{equation}
Podstawiając \(h=\lambda/4\) otrzymujemy \(\delta=\pi\) zatem
\begin{equation}
\begin{split}
     E_\text{tot}&=-lE_0\sin\phi +\\
     &+rE_0\sin\phi\left[(-l-l^5-l^9-...)+(l^3+l^7+...)\right]\,.
\end{split}
\end{equation}
Wyrażenie w nawiasie kwadratowym jest równe sumie dwóch szeregów geometrycznych,
które z łatwością można obliczyć
\begin{equation}
\begin{split}
    E_\text{tot}&=-lE_0\sin\phi+rE_0\sin\phi\frac{-l}{1+l^2}\\
    &=-E_0\sin\phi \frac{l+l^3+rl}{1+l^2}\\
    &=-E_0\sin\phi \frac{2l}{1+l^2}\,.
\end{split}
\end{equation}
Amplituda pola elektrycznego światła odbitego wynosi zatem
\begin{equation}
    E_0\frac{2l}{1+l^2}\,.
\end{equation}
Natężenie światła odbitego jest więc równe
\begin{equation}
    I_{o}=I\frac{4l^2}{(1+l^2)^2}=4I\frac{1-r}{(2-r)^2}\,.
\end{equation}
Oczywiście ze względu na zachowanie energii natężenie światła przechodzącego
przez płytkę jest równe
\begin{equation}
    I_p=I-I_o=I-4I\frac{1-r}{(2-r)^2}\quad\blacksquare
\end{equation}

\subsubsection*{Szkic rozwiązania zadania 36.}
Droga optyczna światła wewnątrz kuli jest dana przez
\begin{equation}
\begin{split}
     c(t_B-t_A)&=\int\limits_{r_A}^{r_B}n(r)\sqrt{1+r^2\phi'(r)^2}\dd{r}\\
     &=\int\limits_{r_A}^{r_B}F(\phi',r)\dd{r}\,.
\end{split}
\end{equation}
Zgodnie z zasadą Fermata całka ta musi być stacjonarna, zatem z tw.
Eulera--Lagrange'a musi zachodzić
\begin{equation}
    \pdv{F}{\phi'}=P=\text{const.}\,,
\end{equation}
skąd
\begin{equation}
    \dv{\phi}{r}=\frac{P}{r}\frac{1}{\sqrt{n^2(r)r^2-P^2}}\,.
\end{equation}
Dla zadanej trajektorii zachodzi
\begin{equation}
    \phi(r)=\arcsin\left(\frac{r^2-R^2}{2rb}\right)\,,
\end{equation}
skąd
\begin{equation}
    n(r)=\frac{2P\sqrt{R^2+b^2}}{r^2+R^2}\,.
\end{equation}
Całkę ruchu \(P\) wyznaczymy z warunku \(n(R)=1\), skąd
\(P=R^2/\sqrt{R^2+b^2}\), zatem
\begin{equation}
    n(r)=\frac{2}{1+\left(\frac{r}{R}\right)^2}\quad\blacksquare
\end{equation}
\subsubsection*{Szkic rozwiązania zadania 40.}
\textit{W rozwiązaniu kładę \(c=1\).}
\medskip

Niech \(\boldsymbol{\beta}_\text{e}\), \(\boldsymbol{\beta}_\text{p}\),
\(\boldsymbol{\beta}_\nu\) oznaczają prędkości poszczególnych cząstek, a
\(\mathscr{E}_\text{e}\), \(\mathscr{E}_\text{p}\), \(\mathscr{E}_\nu\) -- ich
energie relatywistyczne. Z zasady zachowania czterowektora energii--pędu mamy
\begin{equation}
    \begin{split}
        -\frac{m_\text{e}\boldsymbol{\beta}_\text{e}}{\sqrt{1-\beta^2_\text{e}}}=\frac{m_\text{p}\boldsymbol{\beta}_\text{p}}{\sqrt{1-\beta^2_\text{p}}}+\frac{m_\nu\boldsymbol{\beta}_\nu}{\sqrt{1-\beta^2_\nu}}
    \end{split}
\end{equation}
oraz
\begin{equation}\label{eq:energy}
    m_\text{n}=\mathscr{E}_\text{e}+\mathscr{E}_\text{p}+\mathscr{E}_\nu\,.
\end{equation}
Podnosząc pierwsze równanie skalarnie do kwadratu i korzystając z zależności
\begin{equation}
    \beta=\sqrt{1-\frac{m^2}{\mathscr{E}^2}}
\end{equation}
otrzymujemy
\begin{equation}
\mathscr{E}^2_\text{e}-m_\text{e}^2=\mathscr{E}^2_\text{p}-m_\text{p}^2+\mathscr{E}^2_\nu-m_\nu^2+2\mathscr{E}_\text{p}\mathscr{E}_\nu\beta_\text{p}\beta_\nu\cos\Psi\,,
\end{equation}
gdzie \(\Psi\) jest kątem między \(\boldsymbol{\beta}_\text{p}\) i
\(\boldsymbol{\beta}_\nu\). Ponieważ \(\cos\Psi\leq 1\), więc z powyższego
otrzymujemy nierówność
\begin{equation}
    \mathscr{E}^2_\text{e}-m_\text{e}^2\leq \mathscr{E}^2_\text{p}-m_\text{p}^2+\mathscr{E}^2_\nu-m_\nu^2+2\mathscr{E}_\text{p}\mathscr{E}_\nu\beta_\text{p}\beta_\nu\,.
\end{equation}
Jednocześnie podnosząc równanie (\ref{eq:energy}) do kwadratu mamy
\begin{equation}
    (m_\text{n}-\mathscr{E}_\text{e})^2=\mathscr{E}^2_\text{p}+\mathscr{E}^2_\nu+2\mathscr{E}_\text{p}\mathscr{E}_\nu\,.
\end{equation}
Z powyższego zatem
\begin{equation}
    m_\text{n}(2\mathscr{E}_\text{e}-m_\text{n})\leq (m_\text{e}^2-m_\text{p}^2-m_\nu^2)-2\mathscr{E}_\text{p}\mathscr{E}_\nu(1-\beta_\text{p}\beta_\nu)\,,
\end{equation}
czyli
\begin{equation}
    \mathscr{E}_\text{e}\leq \frac{m_\text{n}^2-m_\text{p}^2+m_\text{e}^2-m_\nu^2}{2m_\text{n}}-\frac{m_\text{p}m_\nu}{m_\text{n}}\frac{1-\beta_\text{p}\beta_\nu}{\sqrt{(1-\beta_\text{p}^2)(1-\beta_\nu^2)}}\,.
\end{equation}
Zauważmy, że
\begin{equation}
    1-\beta_\text{p}\beta_\nu = \frac{(1-\beta_\text{p})(1+\beta_\nu)+(1+\beta_\text{p})(1-\beta_\nu)}{2}\,,
\end{equation}
skąd
\begin{equation}
\begin{split}
    &\frac{1-\beta_\text{p}\beta_\nu}{\sqrt{(1-\beta_\text{p}^2)(1-\beta_\nu^2)}}=\\
    &=\frac{1-\beta_\text{p}\beta_\nu}{\sqrt{(1-\beta_\text{p})(1+\beta_\nu)}\sqrt{(1+\beta_\text{p})(1-\beta_\nu)}}=\\
    &=\frac{1}{2}\frac{\sqrt{(1-\beta_\text{p})(1+\beta_\nu)}}{\sqrt{(1+\beta_\text{p})(1-\beta_\nu)}}+\frac{1}{2}\frac{\sqrt{(1+\beta_\text{p})(1-\beta_\nu)}}{\sqrt{(1-\beta_\text{p})(1+\beta_\nu)}}\geq 1\,,
\end{split}
\end{equation}
gdzie ostatnia nierówność wynika z elementarnej nierówności dla dodatnich liczb
rzeczywistych
\begin{equation}
    x+\frac{1}{x}\geq 2\,.
\end{equation}
W takim razie
\begin{equation}
\begin{split}
    \mathscr{E}_\text{e}&\leq\frac{m_\text{n}^2-m_\text{p}^2+m_\text{e}^2-m_\nu^2}{2m_\text{n}}-\frac{m_\text{p}m_\nu}{m_\text{n}}=\\
    &=\frac{m_\text{n}^2+m_\text{e}^2-(m_\text{p}+m_\nu)^2}{2m_\text{n}}\\
    &\approx\frac{m_\text{n}^2-m_\text{p}^2}{2m_\text{n}}\approx 1.29\,\text{MeV}\quad\quad\blacksquare
\end{split}
\end{equation}

\end{document}
