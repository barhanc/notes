\documentclass{myclass}
\usepackage[polish]{babel}

% TODO

% * Composite systems made of 2-state systems (2 x 2 & later n x 2 or maybe n x 2 immediately(?))
% * Composite systems: entanglement, density operator, partial trace, no-cloning theorem,
%   decoherence, Bell's theory
%
% * later (after extensively studying composite systems, entanglement, Bell's theory) begin
%   description of quantum registers, gates, idea of computation using those devices (Deutsch,
%   Grover, QFTrans, Shor), applications to number theory, connection with Church-Turing hypothesis.
%   Examples
%
% * possible physical realizations of quantum computing (gates + adiabatic/topological(?)), I'll
%   probably focus on Josephson junctions and charge qubits as they seem most elegant (maybe qdots
%   as well)
%
% * A short overview of classical and quantum information theory (I don't want to go very deep there
%   at the moment): Shannon/Von Neumann entropy, channels, quantum information transfer/cryptography
%   ------
%
%
%

\title{Podstawy teorii kwantów}
\author{Bartosz Hanc}
% My notes regarding abstract foundations of quantum theory and introduction to quantum computing

\begin{document}
\tableofcontents
\section{Abstrakcyjna teoria kwantów}

\subsection{Elementy teorii przestrzeni Hilberta}

Przez \(\mathbb{V} := (V,\mathbb{C},+,\cdot)\) będziemy oznaczać przestrzeń wektorową nad ciałem
liczb zespolonych.

\begin{definition}
Odwzorowanie \(d: V \times V \mapsto \mathbb{R}\) będziemy nazywać \textit{metryką} w zbiorze \(V
\neq \emptyset\) iff
\begin{itemize}

\item \(\forall u,v \in V : d(u,v) \geq 0\), przy czym równość zachodzi iff \(u = v\)
(\textit{nieujemność})

\item \(\forall u,v \in V : d(u,v) = d(v,u)\) (\textit{symetria})

\item \(\forall u,v,w \in V : d(u,v) + d(v,w) \geq d(u,w)\) (\textit{nierówność trójkąta})

\end{itemize}
Parę \((V,d(\cdot,\cdot))\) będziemy nazywać \textit{przestrzenią metryczną}.
\end{definition}

\begin{definition}
Niech \((V,d)\) będzie przestrzenią metryczną. Mówimy, iż dany ciąg \((u_n)\) elementów zbioru \(V\)
jest zbieżny do \(g\in V\) tj. \(u_n \to g\) przy \(n \to \infty\) iff \(d(u_n,g) \to 0\) przy \(n
\to \infty\).   
\end{definition}

\begin{definition}
Ciąg \((u_n)\) elementów \(u_n \in V\) będziemy nazywać \textit{ciągiem Cauchy'ego} w przestrzeni
metrycznej \((V,d(\cdot,\cdot))\) iff spełnia on kryterium Cauchy'ego tj.
\begin{equation*}
    \forall \epsilon > 0 : \exists N : \forall n,m > N : d(u_n,u_m) < \epsilon\,.
\end{equation*}  
\end{definition}

\begin{theorem}
Każdy ciąg zbieżny w przestrzeni metrycznej \((V,d)\) jest ciągiem Cauchy'ego w tej przestrzeni.
\end{theorem}

\begin{definition}
Przestrzeń metryczną \((V,d(\cdot,\cdot))\) nazwiemy \textit{zupełną} iff każdy ciąg Cauchy'ego
\((u_n)\) elementów \(u_n \in V\) jest zbieżny do granicy \(g \in V\).
\end{definition}

\begin{definition}
Niech \(\mathbb{V}\) będzie przestrzenią wektorową. Odwzorowanie \(\braket{\cdot} : V \times V
\mapsto \mathbb{C}\) nazwiemy \textit{iloczynem wewnętrznym} wektorów iff
\begin{itemize}
    
    \item \(\forall u,v \in V : \braket{u}{v}^* = \braket{v}{u}\)
    
    \item \(\forall u,v_1,v_2 \in V : \forall \alpha, \beta \in \mathbb{C} : \braket{u}{\alpha v_1 +
    \beta v_2} = \alpha \braket{u}{v_1} + \beta \braket{u}{v_2}\)

    \item \(\forall u \in V : \braket{u} \geq 0\), przy czym równość zachodzi iff \(u =
    \mathsf{0}\). Zauważmy tutaj, iż z pierwszego aksjomatu \(\braket{u} \in \mathbb{R}\), gdyż
    \(\braket{u} = \braket{u}^* \implies \Im{\braket{u}} = 0\). 

\end{itemize}
Parę \((\mathbb{V},\braket{\cdot})\) będziemy nazywać \textit{przestrzenią unitarną}.
\end{definition}

\begin{theorem}
Każda przestrzeń unitarna jest metryczna z metryką indukowaną przez iloczyn wewnętrzny \(d(u,v) :=
\norm{u - v} = \sqrt{\braket{u-v}}\).
\end{theorem}

\begin{theorem}[\textit{Nierówność Cauchy'ego--Schwarza}]
Niech \((\mathbb{V},\braket{\cdot})\) -- przestrzeń unitarna. Wówczas
\begin{equation*}
    \forall u,v \in V : |\braket{u}{v}|^2 \leq \braket{u}\braket{v}\,.
\end{equation*}  
\end{theorem}

\begin{definition}
Przeliczalny zbiór wektorów \(\{v_1,...,v_n\}\) nazwiemy \textit{ortogonalnym} iff 
\begin{equation*}
    \forall i\neq j; i,j\in\{1,...,n\} : \braket{v_i}{v_j} = 0\,.
\end{equation*}
Ten sam zbiór wektorów nazwiemy \textit{ortonormalnym} iff 
\begin{equation*}
    \forall i,j\in\{1,...,n\} : \braket{v_i}{v_j} = \delta_{ij}\,,
\end{equation*}
gdzie \(\delta_{ij}\) jest deltą Kroneckera.
\end{definition}

\begin{theorem}
Każda przestrzeń unitarna \((\mathbb{V}, \braket{\cdot})\) posiada bazę ortonormalną, tj. bazę,
której wektory bazowe tworzą zbiór ortonormalny.
\end{theorem}

\begin{definition}
\textit{Przestrzenią Hilberta} \(\mathscr{H}=(\mathbb{V},\braket{\cdot})\) nazwiemy zupełną
przestrzeń unitarną.
\end{definition}

\begin{definition}
Niech \(\mathscr{H} = (\mathbb{V},\braket{\cdot})\) będzie przestrzenią Hilberta. Odwzorowanie
liniowe \(F:V\mapsto\mathbb{C}\) nazwiemy \textit{funkcjonałem liniowym} w przestrzeni
\(\mathscr{H}\).
\end{definition}

\begin{theorem}
Niech \(V^*\) oznacza zbiór wszystkich funkcjonałów liniowych \(F:V\mapsto\mathbb{C}\). Wówczas
\(\mathbb{V}^*:=(V^*,\mathbb{C},+,\cdot)\), gdzie
\begin{itemize}
    \item \(\forall F_1,F_2 \in V^* : \forall v \in V : (F_1+F_2)(v) = F_1(v) + F_2(v)\)

    \item \(\forall F \in V^* : \forall \alpha \in \mathbb{C} : \forall v \in V : (\alpha \cdot
    F)(v) = \alpha F(v)\)
\end{itemize}
jest przestrzenią wektorową, którą nazywamy \textit{przestrzenią dualną}.
\end{theorem}

\begin{theorem}[\textit{Riesza}]
Niech \(\mathscr{H}=(\mathbb{V},\braket{\cdot})\) będzie przestrzenią Hilberta, a \(\mathbb{V}^*\)
jej przestrzenią dualną. Wówczas istnieje wzajemnie jednoznaczne odwzorowanie wektorów \(v \in V\)
na funkcjonały liniowe \(F \in V^*\). Dodatkowo dla każdego funkcjonału \(F\) istnieje dokładnie
jeden wektor \(u \in V\) taki, że
\begin{equation*}
    \forall v \in V : F(v) = \braket{u}{v}\,.
\end{equation*}
\end{theorem}

\begin{definition}
\textit{Iloczynem Kroneckera} macierzy \(\oper{A} = [a_{ij}]_{n\times m}\) i \(\oper{B} =
[b_{ij}]_{n'\times m'}\) nazywamy macierz wymiaru \(nn'\times mm'\) postaci
\begin{equation*}
    \begin{split}
        \oper{A} \otimes \oper{B} &= \mqty[a_{11} & \cdots & a_{1m} \\ \vdots & \ddots & \vdots \\ a_{n1} & \cdots & a_{nm}] \otimes \mqty[b_{11} & \cdots & b_{1m'} \\ \vdots & \ddots & \vdots \\ b_{n'1} & \cdots & b_{n'm'}] = \mqty[a_{11}\oper{B} & \cdots & a_{1m}\oper{B} \\ \vdots & \ddots & \vdots \\ a_{n1}\oper{B} & \cdots & a_{nm}\oper{B}]
    \end{split}
\end{equation*}
\end{definition}

\begin{definition}
\textit{Iloczynem tensorowym} przestrzeni Hilberta \(\mathscr{H}_1\) i \(\mathscr{H}_2\) o bazach
ortonoromalnych odpowiednio \(\{\phi_i^{(1)}\}\) i \(\{\phi_j^{(2)}\}\) nazywamy przestrzeń Hilberta
\(\mathscr{H} = \mathscr{H}_1 \otimes \mathscr{H}_2\) taką, że:
\begin{itemize}
    
    \item Jej bazą ortonormalną jest zbiór \(\{\phi_i^{(1)} \otimes \phi_j^{(2)}\}\).

    \item Iloczyn wewnętrzny w przestrzeni \(\mathscr{H}_1 \otimes \mathscr{H}_2\) jest zdefiniowany
    jako
    \begin{equation*}
        \braket{\chi_1 \otimes \chi_2}{\psi_1 \otimes \psi_2} := \braket{\chi_1}{\psi_1}_1 \cdot \braket{\chi_2}{\psi_2}_2\,,
    \end{equation*}
    gdzie \(\chi_i,\psi_i \in \mathscr{H}_i\) to pewne wektory, a \(\braket{\cdot}_i\) to iloczyn
    wewnętrzny w \(\mathscr{H}_i\).
\end{itemize}
\end{definition}

\subsubsection*{Notacja Diraca}

Niech \(\mathscr{H}\) będzie przestrzenią Hilberta. Wprowadzimy teraz kompaktową notacją wektorów i
funkcjonałów liniowych wymyśloną przez P.A.M. Diraca. Aby uprościć zapis, będziemy mówili o
wektorach należących do przestrzeni \(\mathscr{H}\) (używając nawet symbolu należenia \(\in
\mathscr{H}\)), mając oczywiście formalnie na myśli wektory należące do zbioru \(V\).

Wektory należące do \(\mathscr{H}\) będziemy oznaczać jako
\begin{equation*}
    \ket{\psi},\, \ket{\phi},\, ...\,,
\end{equation*}
przy czym \(\ket{\cdot}\) to tzw. \textit{ket} i formalnie jest to odwzorowanie \(\ket{\cdot} :
\mathsf{S} \mapsto V\), gdzie \(\mathsf{S}\) jest zbiorem znaków, których używamy do oznaczenia
konkretnych wektorów ze zbioru \(V\). Nie będziemy jednak przestrzegali tego formalnego znaczenia,
utożsamiając dla wygody również  sam symbol z wektorem.

Funkcjonały liniowe należące do przestrzeni dualnej będziemy oznaczać jako
\begin{equation*}
    \bra{\psi},\, \bra{\phi},\, ...\,,
\end{equation*}
przy czym \(\bra{\cdot}\) to tzw. \textit{bra} i formalnie jest to odwzorowanie \(\bra{\cdot} :
\mathsf{S}^* \mapsto V^*\), gdzie \(\mathsf{S}^*\) jest zbiorem znaków, których używamy do
oznaczenia konkretnych funkcjonałów ze zbioru \(V^*\). Ponieważ z tw. Riesza istnieje wzajemnie
jednoznaczne odwzorowanie funkcjonałów liniowych na wektory, więc możemy utożsamić \(\mathsf{S}^* =
\mathsf{S}\).

\subsubsection*{Skończenie wymiarowa przestrzeń Hilberta nad $\mathbb{C}$}

Rozważymy teraz konstrukcję skończenie wymiarowej przestrzeni Hilberta złożonej ze skończenie
wymiarowej przestrzeni wektorowej \(\mathbb{V} = (\mathbb{C}^n, \mathbb{C}, +, \cdot)\), której
elementy będziemy w danej bazie \textit{ortonormalnej} \(\{\phi_1, \ldots, \phi_n\}\) zapisywać jako
\begin{equation*}
    \mathbb{V} \ni \ket{\psi} = \sum_{i=1}^n a_i \ket{\phi_i} = \mqty[a_1 \\ \vdots \\ a_n]\,,
\end{equation*}
gdzie \(a_i = \braket{\phi_i}{\psi}\in\mathbb{C}\) oraz iloczynu wewnętrznego zdefiniowanego jako
\begin{equation*}
    \braket{\psi}{\phi} := \sum_{i=1}^n a_{i}^*b_i
\end{equation*}
dla
\begin{equation*}
    \psi = \mqty[a_1 \\ \vdots \\ a_n]\,,\quad \phi = \mqty[b_1 \\ \vdots \\ b_n]\,.
\end{equation*}
Powstała w ten sposób skończenie wymiarowa przestrzeń unitarna jest trywialnie zupełna, a zatem
skonstruowaliśmy skończenie wymiarową przestrzeń Hilberta. Wektor w tej przestrzeni możemy utożsamić
(poprzez iloczyn wewnętrzny) z macierzą kolumnową jego współrzędnych w danej bazie ortonormalnej.
Jasne jest również czym jest funkcjonał liniowy stowarzyszony z danym wektorem
\begin{equation*}
    \bra{\psi} = \mqty[a_1 \\ \vdots \\ a_n]^\dagger = \mqty[a_1^* & \ldots & a_n^*]\,,
\end{equation*}
gdzie \(\dagger\) oznacza \textit{sprzężenie hermitowskie} macierzy, tj. sprzężoną macierz
transponowaną. W dalszej części skupimy się głównie na skończenie wymiarowych przestrzeniach
Hilberta \(((\mathbb{C}^n, \mathbb{C}, +, \cdot),\braket{\cdot})\), gdyż stanowią one podstawę opisu
teorii obliczeń kwantowych i kwantowej teorii informacji. Należy zdawać sobie jednak sprawę, iż
stanowi to duże uproszczenie w stosunku do wymagań pełnoprawnych teorii fizycznych (mechanika
falowa, kwantowa teoria pola), w których niezbędna jest teoria nieskończenie wymiarowych przestrzeni
Hilberta.

\subsection{Elementy teorii operatorów liniowych}

\begin{definition}
\textit{Operatorem liniowym} \(\oper{A}\) w przestrzeni \(\mathscr{H} = (\mathbb{V},
\braket{\cdot})\) nazywamy odwzorowanie liniowe 
\begin{equation*}
    \oper{A} : D(\oper{A}) \mapsto D(\oper{A})\,,
\end{equation*}
gdzie \(D(\oper{A})\) jest pewną podprzestrzenią wektorową przestrzeni \(\mathbb{V}\). Dodatkowo
zakładamy, iż dziedziny operatorów są gęste, to znaczy ich domknięcia są równe \(\mathscr{H}\).
\end{definition}

Zgodnie z wcześniejszymi komentarzami nie będziemy wnikali w subtelne problemy wynikające z faktu,
iż w nieskończenie wymiarowej przestrzeni Hilberta pojęcie operatora liniowego jest nieodłącznie
związane z pojęciem dziedziny tego operatora, który w ogólności nie jest określony na całej
przestrzeni Hilberta, a tylko na pewnym jej podzbiorze. Komplikacje te nie występują w skończenie
wymiarowych przestrzeniach Hilberta wymiaru \(n\), w których operatory liniowe możemy utożsamiać z
\textit{endomorfizmami} tej przestrzeni
\begin{equation*}
\oper{A}: V \mapsto V\,.
\end{equation*}

Jak wiadomo z elementarnej algebry w przypadku \(n\)--wymiarowej przestrzeni wektorowej każdemu
endomorfizmowi \(\oper{A}\) możemy przyporządkować macierz wymiaru \(n\times n\), której elementy w
danej bazie ortonormalnej \(\{\phi_1,\ldots,\phi_n\}\) są dane przez wartości \(\oper{A}\) na
wektorach bazowych
\begin{equation*}
\begin{split}
\oper{A}\ket{\phi_1} &= A_{11}\ket{\phi_1} + A_{21}\ket{\phi_2} + \ldots + A_{n1}\ket{\phi_n}\\
&\vdots\\
\oper{A}\ket{\phi_n} &= A_{1n}\ket{\phi_1} + A_{2n}\ket{\phi_2} + \ldots + A_{nn}\ket{\phi_n}
\end{split}\quad,
\end{equation*}
skąd element \(A_{ij}\) macierzy \(\oper{A}\) w bazie ortonormalnej \(\{\phi_i\}\) jest dany przez
\begin{equation*}
A_{ij} = \bra{\phi_i}\oper{A}\ket{\phi_j}\,.
\end{equation*}

\begin{definition}
\textit{Sprzężeniem} operatora \(\oper{A}\) nazywamy operator \(\oper{A}^\dagger\) zdefiniowany
(pomijając wszelkie problemy związane z określeniem dziedzin operatorów) przez równanie
\begin{equation*}
\forall \psi,\phi\in\mathscr{H}: \bra{\psi}\oper{A}^\dagger\ket{\phi} = \bra{\phi}\oper{A}\ket{\psi}^*\,.
\end{equation*}
\end{definition}

Podstawiając w miejsce wektorów \(\psi\), \(\phi\) wektory bazy otrzymujemy (w przypadku skończenie
wymiarowych przestrzeni Hilberta) zależność między macierzą \(\oper{A}\) i jej sprzężeniem
hermitowskim \(\oper{A}^\dagger\)
\begin{equation*}
A^\dagger_{ij} = A_{ji}^*\,.
\end{equation*} 

\begin{definition}
\textit{Komutatorem} operatorów \(\oper{A}\), \(\oper{B}\) operator \([\oper{A},\oper{B}]\)
zdefiniowany jako
\begin{equation*}
\forall \psi \in D : [\oper{A},\oper{B}]\psi = \oper{A}\oper{B}\psi - \oper{B}\oper{A}\psi\,. 
\end{equation*}
Jeśli \([\oper{A},\oper{B}] = \oper{\mathsf{0}}\) (gdzie \(\oper{\mathsf{0}}\) oznacza
\textit{operator zerowy} \(\oper{\mathsf{0}}\psi=\mathsf{0}\)), to mówimy, że operatory
\(\oper{A}\), \(\oper{B}\) \textit{komutują}.
\end{definition}

\begin{definition}
\textit{Antykomutatorem} operatorów \(\oper{A}\), \(\oper{B}\) nazywamy operator
\(\{\oper{A},\oper{B}\}\) zdefiniowany jako
\begin{equation*}
    \{\oper{A},\oper{B}\}\psi := \oper{A}\oper{B}\psi + \oper{B}\oper{A}\psi\,.
\end{equation*}
\end{definition}

\begin{theorem}
Dla dowolnych operatorów \(\oper{A}\), \(\oper{B}\), \(\oper{C}\) zakładając odpowiednie dziedziny,
zachodzi:
\begin{itemize}
    \item \([\oper{A} + \oper{B},\oper{C}] = [\oper{A},\oper{C}] + [\oper{B},\oper{C}]\) ;
    \item \([\oper{A}\oper{B},\oper{C}] = \oper{A}[\oper{B},\oper{C}] +
    [\oper{A},\oper{C}]\oper{B}\) .
    \item \([[\oper{A}, \oper{B}], \oper{C}] + [[\oper{B},\oper{C}],\oper{A}] + [[\oper{C},
    \oper{A}], \oper{B}] = \oper{0}\) (\textit{tożsamość Jacobiego})
\end{itemize}   
\end{theorem}

\begin{definition}
\textit{Ślad operatora \(\oper{A}\)} definiujemy jako liczbę \(\Trace \oper{A} \in \mathbb{C}\)
równą
\begin{equation*}
\Trace{\oper{A}} := \sum_{i} \bra{\phi_i}\oper{A}\ket{\phi_i}\,,
\end{equation*}
gdzie \(\{\phi_i\}\) jest dowolną ortonormalną bazą przestrzeni \(\mathscr{H}\).
\end{definition}

\begin{theorem}
Ślad operatora nie zależy od wyboru ortonormalnej bazy przestrzeni Hilberta.    
\end{theorem}

\begin{theorem} Podstawowe własności śladu.
\begin{itemize}
    \item \(\Trace(\alpha\oper{A} + \beta\oper{B}) = \alpha\Trace(\oper{A}) +
    \beta\Trace(\oper{B})\)
    \item \(\Trace{\oper{ABC}} = \Trace{\oper{BCA}} = \Trace{\oper{CAB}}\)
    \item \(\Trace{\oper{A}} = \left(\Trace{\oper{A}^\dagger}\right)^*\)
    \item \(\det{\e^{\oper{A}}} = \e^{\Trace{\oper{A}}}\)
\end{itemize}  
\end{theorem}

\begin{definition}[\textit{Funkcja operatora}]
Niech \(f(x):\mathbb{R}\mapsto\mathbb{R}\) będzie funkcją zmiennej rzeczywistej taką, że istnieje
szereg potęgowy
\begin{equation*}
\sum_{n=0}^\infty\frac{a_n}{n!}x^n\,,
\end{equation*}
który jest zbieżny jednostajnie na \(\mathbb{R}\) do \(f\). Wówczas funkcję operatora
\(f(\oper{A})\) definiujemy jako
\begin{equation*}
f(\oper{A}) := \sum_{n=0}^\infty \frac{a_n}{n!}\oper{A}^n\,,
\end{equation*}
gdzie przyjmujemy \(\oper{A}^0 := \oper{1}\). W szczególności mamy
\begin{itemize}
    \item \(\exp(\oper{A}) := \sum_{n=0}^\infty \frac{1}{n!}\oper{A}^n\)

    \item \(\sin(\oper{A}) := \sum_{n=0}^\infty \frac{(-1)^n}{(2n+1)!}\oper{A}^{2n+1}\)

    \item \(\cos(\oper{A}) := \sum_{n=0}^\infty \frac{(-1)^n}{(2n)!}\oper{A}^{2n}\)
\end{itemize}
\end{definition}

W teorii kwantowej główną rolę odgrywają trzy rodziny operatorów: operatory samosprzężone, rzutowe i
unitarne.

\linesep

\begin{definition}
    \textit{Operatorem samosprzężonym} (pomijając wszelkie problemy związane z określeniem dziedzin
    operatorów) nazywamy operator \(\oper{A}\), dla którego \(\oper{A} = \oper{A}^\dagger\).
\end{definition}

W przypadku skończenie wymiarowych przestrzeni Hilberta definicja ta jest pełna, a operatory
samosprzężony możemy utożsamiać z macierzami hermitowskimi tj. macierzami, których elementy
spełniają związek
\begin{equation*}
    A_{ij} = A_{ji}^*\,.
\end{equation*}

W przypadku nieskończenie wymiarowych przestrzeni Hilberta definicja ta jest niepełna gdyż trzeba
mieć świadomość, iż równość operatorów oznacza z definicji równość ich dziedzin, co wymaga
wprowadzenia rozróżnienia między operatorami jedynie \textit{symetrycznymi} (tj. spełniającymi
równość \(\bra{\psi}\oper{A}\ket{\phi} = \bra{\phi}\oper{A}\ket{\psi}^*\) dla dowolnych
\(\psi,\phi\in D(\oper{A})\)), a operatorami samosprzężonymi.

Operatory samosprzężone odgrywają wyróżnioną rolę w teorii kwantowej ze względu na trzy twierdzenia,
które są dla nich spełnione.

\begin{theorem}
Wartości własne operatora samosprzężonego są liczbami rzeczywistymi.   
\end{theorem}

\begin{theorem}
Zbiór wektorów własnych operatora samosprzężonego rozpina przestrzeń \(\mathscr{H}\).   
\end{theorem}

\begin{theorem}
Jeśli widmo operatora samosprzężonego nie jest zdegenerowane, to wektory własne tworzą zbiór
ortogonalny.
\end{theorem}

\linesep

\begin{definition}
\textit{Operatorem rzutowym} nazywamy operator \(\oper{P}\) taki, że \(\oper{P} = \oper{P}^\dagger\)
(samosprzężoność) i \(\oper{P}^2 = \oper{P}\) (idempotentność). 
\end{definition}

Ważnym przykładem operatora rzutowego jest operator rzutowania na jednowymiarową podprzestrzeń
rozpiętą na unormowanym wektorze \(\ket{\phi}\) (rzutowanie na kierunek wektora \(\phi\)), który w
notacji Diraca możemy zapisać jako \(\oper{P} = \ket{\phi}\bra{\phi}\) tj. \(\forall \psi :
\oper{P}(\psi) = \braket{\phi}{\psi}\phi\). Jest to oczywiście operator liniowy, gdyż dla dowolnych
wektorów \(\ket{\psi_1}\), \(\ket{\psi_2}\) i skalarów \(\alpha\), \(\beta\) mamy
\begin{equation*}
    \begin{split}
        &\ket{\phi}\bra{\phi}(\alpha\ket{\psi_1} + \beta\ket{\psi_2}) = \ket{\phi}\braket{\phi}{\alpha\psi_1 + \beta\psi_2} \\
        &= \alpha\ket{\phi} \braket{\phi}{\psi_1} + \beta\ket{\phi}\braket{\phi}{\psi_2}\,.
    \end{split}
\end{equation*}
Jest również idempotentny, gdyż 
\begin{equation*}
    \ket{\phi}\bra{\phi}(\ket{\phi}\braket{\phi}{\psi}) =  \ket{\phi}\braket{\phi}{\psi}
\end{equation*}
z założenia \(\braket{\phi} = 1\) oraz samosprzężony
\begin{equation*}
    (\braket{\psi_1}{\phi}\braket{\phi}{\psi_2})^* = \braket{\psi_1}{\phi}^*\braket{\phi}{\psi_2}^* = \braket{\phi}{\psi_1}\braket{\psi_2}{\phi} = \braket{\psi_2}{\phi}\braket{\phi}{\psi_1}\,.
\end{equation*}
Łatwo pokazać również, iż jeśli \(\{\phi_i\}\) jest ortonormalnym zbiorem wektorów, to
\begin{equation*}
    \oper{P} = \sum_i \ket{\phi_i}\bra{\phi_i}
\end{equation*}
jest operatorem rzutowym. W szczególności, jeśli \(\{\phi_i\}\) jest ortonormalną bazą przestrzeni
\(\mathscr{H}\), to
\begin{equation*}
    \sum_i \ket{\phi_i}\bra{\phi_i} = \oper{1}\,.
\end{equation*}

\linesep

\begin{definition}
\textit{Operatorem unitarnym} nazywamy operator \(\oper{U}\) taki, że
\begin{equation*}
    \oper{U}\oper{U}^\dagger = \oper{U}^\dagger\oper{U} = \oper{1}\,.
\end{equation*}
\end{definition}

Przekształcenia unitarne reprezentowane przez operatory unitarne mają użyteczną własność polegającą
na zachowywaniu wartości iloczynu wewnętrznego dwóch wektorów, a zatem w szczególności normy wektora
\begin{equation*}
    \braket{\oper{U}\psi}{\oper{U}\phi} = \braket{\oper{U}^\dagger\oper{U}\psi}{\phi} = \braket{\psi}{\phi}\,.
\end{equation*}

\linesep

\begin{theorem}[\textit{spektralne}]
Niech \(\mathscr{H}\) będzie przestrzenią Hilberta. Dla każdego samosprzężonego operatora liniowego
\(\oper{A}\) w \(\mathscr{H}\) istnieje unikalna rodzina operatorów rzutowych \(\oper{P}(\lambda)\)
indeksowanych ciągłym parametrem \(\lambda \in \mathbb{R}\) taka, że
\begin{itemize}
    
    \item \(\oper{P}(\lambda_1)\oper{P}(\lambda_2) = \oper{P}(\min(\lambda_1,\lambda_2))\)

    \item \(\forall \lambda: \lim_{\epsilon\to0^+}\oper{P}(\lambda+\epsilon) = \oper{P}(\lambda)\)
    
    \item \(\lim_{\lambda\to-\infty}\oper{P}(\lambda) = \oper{0}\)

    \item \(\lim_{\lambda\to+\infty}\oper{P}(\lambda) = \oper{1}\)

    \item \(\oper{A} = \int\limits_{-\infty}^{+\infty}\lambda \dd{\oper{P}(\lambda)}\)

\end{itemize}
gdzie ostatnia całka to tzw. \textit{całka Riemanna--Stieltjesa} względem miary operatorowej
zdefiniowana jako
\begin{equation*}
        \int\limits_a^b f(x) \dd{\sigma(x)} := \lim_{n\to\infty}\sum_{k=1}^nf(x_k)\left[\sigma(x_k) - \sigma(x_{k-1})\right]\,,
\end{equation*}
dla 
\begin{equation*}
    f: \mathbb{R} \mapsto \mathbb{R}\,,\quad \sigma: \mathbb{R} \mapsto X\,,
\end{equation*}
gdzie \([a;b] = \bigcup_{k=1}^{n}[x_{k-1}; x_k]\) jest podziałem normalnym odcinka \([a;b]\).
Dodatkowo dla dowolnej funkcji operatora \(f\) zachodzi
\begin{equation*}
    f(\oper{A}) = \int\limits_{-\infty}^{+\infty} f(\lambda) \dd{\oper{P}(\lambda)}\,.
\end{equation*}
\end{theorem}        

W szczególnym przypadku, gdy operator samosprzężony \(\oper{A}\) ma niezdegenerowane widmo
\(\{\lambda_i\}\) będące zbiorem przeliczalnym, wiemy, że zbiór unormowanych wektorów własnych
\(\{\phi_i\}\) jest bazą ortonormalną przestrzeni \(\mathscr{H}\), czyli dla dowolnego wektora
\(\psi \in \mathscr{H}\) możemy zapisać
\begin{equation*}
    \ket{\psi} = \sum_{i} c_i\ket{\phi_i} = \sum_{i} \braket{\phi_i}{\psi}\ket{\phi_i}\,,
\end{equation*}
gdzie \(c_i = \braket{\phi_i}{\psi} \in \mathbb{C}\) to współrzędne wektora w zadanej bazie.
Działając operatorem \(\oper{A}\) na wektor \(\psi\) mamy
\begin{equation*}
    \oper{A}\ket{\psi} = \sum_{i} \braket{\phi_i}{\psi}\oper{A}\ket{\phi_i} =
    \sum_{i} \braket{\phi_i}{\psi}\lambda_i\ket{\phi_i} = \left(\sum_i\lambda_i\ket{\phi_i}\bra{\phi_i}\right)\ket{\psi} \,.
\end{equation*}
Całka Stieltjesa z twierdzenia spektralnego przechodzi więc w tym przypadku w sumę (być może
nieskończoną) operatorów rzutowych rzutujących na jednowymiarowe podprzestrzenie rozpięte na
kolejnych wektorach własnych operatora
\begin{equation*}
    \oper{A} = \sum_{i} \lambda_i \ket{\phi_i}\bra{\phi_i}\,.
\end{equation*}

\subsection{Postulaty teorii kwantów}

Poniżej przedstawiono postulaty ogólnej, abstrakcyjnej teorii kwantów. Postulaty te obowiązują we
wszystkich realizacjach teorii kwantów np. mechanice falowej, czy kwantowej teorii pola, jednak ze
względu na swój ogólny charakter same w sobie nie dostarczają narzędzi do rozwiązywania żadnych
konkretnych problemów fizycznych. Nie należy ich również traktować jako podstaw do aksjomatyzacji
teorii kwantowej. Stanowią one raczej sposób uporządkowania w spójną strukturę wiedzy dotyczącej
konkretnych realizacji teorii kwantów
\medskip

\begin{enumerate}[label=\Roman*.]
    
    \item \textbf{O modelu matematycznym.} Modelem matematycznym teorii kwantów jest teoria
    przestrzeni Hilberta nad ciałem liczb zespolonych i teoria operatorów liniowych działających w
    tej przestrzeni.

    \item \textbf{O pytaniach elementarnych.} Pytaniem elementarnym nazwiemy pytanie, na które
    odpowiedź może brzmieć jedynie ,,TAK'' lub ,,NIE''. Pytanie elementarne nazwiemy rozstrzygalnym
    w obrębie danej teorii kwantowej iff istnieje wzajemnie jednoznaczne przyporządkowanie tego
    pytania do pewnego operatora rzutowego \(\oper{P}\). Będziemy wówczas mówili, iż dane pytanie
    elementarne jest reprezentowane przez \(\oper{P}\). Każde pytanie elementarne reprezentowane
    przez \(\oper{P}\) można zanegować otrzymując pytanie reprezentowane przez \(\oper{1} -
    \oper{P}\), natomiast dwa pytania elementarne reprezentowane przez \(\oper{P}_1\) i
    \(\oper{P}_2\) można połączyć spójnikiem
    \begin{itemize}
        \item ,,I''; otrzymując pytanie reprezentowane przez \(\oper{P}_1\oper{P}_2\), przy czym
        musi zachodzić \([\oper{P}_1, \oper{P}_2] = \oper{\mathsf{0}}\).

        \item ,,LUB''; otrzymując pytanie reprezentowane przez \(\oper{P}_1 + \oper{P}_2\), przy
        czym musi zachodzić \(\oper{P}_1\oper{P}_2 = \oper{\mathsf{0}}\).

    \end{itemize}

    \item \textbf{O stanach układu.} Stan układu fizycznego jest reprezentowany przez unormowany
    wektor \(\ket{\Psi}\) w pewnej przestrzeni Hilberta \(\mathscr{H}\), przy czym utożsamiamy ze
    sobą wektory różniące się jedynie globalnym czynnikiem fazowym tj. \(\ket{\Psi} \sim
    \e^{\im\alpha}\ket{\Psi}\), \(\alpha \in \mathbb{R}\).

    \item \textbf{O prawdopodobieństwach.} Teoria kwantowa dostarcza jedynie probabilistycznych
    odpowiedzi na rozstrzygalne pytania elementarne. Prawdopodobieństwo \(p\), iż odpowiedź na
    pytanie elementarne reprezentowane przez \(\oper{P}\) jest twierdząca, dla układu
    reprezentowanego przez \(\Psi\) wynosi
    \begin{equation*}
        p = \bra{\Psi}\oper{P}\ket{\Psi}\,.
    \end{equation*}
    Zauważmy, że z samosprzężoności operatora \(\oper{P}\) mamy \(p = p^*\), więc \(p
    \in\mathbb{R}\), natomiast z nierówności Cauchy'ego--Schwarza mamy
    \begin{equation*}
        \braket{\Psi}\braket{\oper{P}\Psi} = 1 \cdot \bra{\Psi}\oper{P}\ket{\Psi} = p \geq |\bra{\Psi}\oper{P}\ket{\Psi}|^2 = p^2\,,
    \end{equation*}
    skąd \(p(p-1) \leq 0\), co implikuje \(p \in [0;1]\) zgodnie z aksjomatami prawdopodobieństwa.

    \item \textbf{O wielkościach fizycznych.} Każda wielkość fizyczna \(A\) występująca w danej
    teorii kwantowej jest reprezentowana przez samosprzężony operator liniowy \(\oper{A}\) i
    stowarzyszoną z nim na mocy twierdzenia spektralnego rodzinę operatorów rzutowych
    \(\oper{P}_A(\lambda)\). Operator rzutowy \(\oper{P}_A(\lambda)\) reprezentuje pytanie:
    \begin{quote}
        czy wielkość fizyczna \(A\) ma wartość nie większą od \(\lambda\)?
    \end{quote}
    Na mocy postulatu drugiego możemy skonstruować pytanie:
    \begin{quote}
        czy wielkość fizyczna \(A\) ma wartość z przedziału \((\lambda_1;\lambda_2]\)?
    \end{quote}
    reprezentowane przez operator 
    \begin{equation*}
        \left[\oper{1} - \oper{P}_A(\lambda_1)\right]\oper{P}_A(\lambda_2) =
        \oper{P}_A(\lambda_2)-\oper{P}_A(\lambda_1)\,.
    \end{equation*}
    Wartość oczekiwaną funkcji \(f\) wielkości \(A\) dla zespołu statystycznego układów
    przygotowanych w stanie \(\Psi\) obliczamy jako
    \begin{equation*}
        \begin{split}
            \expval{f(A)} &= \int\limits_{-\infty}^{+\infty} f(\lambda) \dv{p(\lambda)}{\lambda} \dd{\lambda} = \int\limits_{-\infty}^{+\infty} f(\lambda) \dv{\bra{\Psi}\oper{P}_A(\lambda)\ket{\Psi}}{\lambda} \dd{\lambda} \\
            &= \bra{\Psi}\left[\int\limits_{-\infty}^{+\infty} f(\lambda) \dd{\oper{P}_A(\lambda)}\right]\ket{\Psi} = \bra{\Psi}f(\oper{A})\ket{\Psi}\,.
        \end{split}
    \end{equation*}

    \item \textbf{O ewolucji układu w czasie.} Stan układu \(\ket{\Psi}\) ewoluuje zgodnie z
    \textit{równaniem Schr\"{o}dingera}
    \begin{equation*}
        \oper{H}\ket{\Psi} = \im\hbar\partial_t\ket{\Psi}\,,
    \end{equation*}
    gdzie \(\oper{H}\) jest specjalnym operatorem samosprzężonym reprezentującym hamiltonian układu,
    tworzonym wedle określonych reguł w danej realizacji teorii kwantów. Równoważnie możemy
    powiedzieć, iż istnieje operator unitarny \(\oper{U}(t;t_0)\) taki, że
    \begin{equation*}
        \ket{\Psi(t)} = \oper{U}(t;t_0)\ket{\Psi(t_0)}\,.
    \end{equation*}
    Istotnie pokażemy, iż ewolucja zgodna z równaniem Schr\"{o}dingera jest unitarna
    \begin{equation*}
        \dv{\braket{\Psi}}{t} = \braket{\dot{\Psi}}{\Psi} + \braket{\Psi}{\dot{\Psi}} = -\frac{1}{\im\hbar}\bra{\oper{H}\Psi}\ket{\Psi} + \frac{1}{\im\hbar}\bra{\Psi}\ket{\oper{H}\Psi} = 0\,,
    \end{equation*}
    gdzie w ostatniej linijce skorzystaliśmy z samosprzężoności \(\oper{H}\). Powyższy wynik nazywa
    się również zasadą zachowania prawdopodobieństwa.

    \item \textbf{O kolapsie.} Po pomiarze wielkości \(A\) układu w stanie \(\ket{\Psi}\), który
    zwrócił wartość z przedziału \((\lambda_1;\lambda_2]\) stan układu zostaje natychmiastowo
    zredukowany do
    \begin{equation*}
        \ket{\Psi} \to \frac{\oper{P}\ket{\Psi}}{\sqrt{\bra{\Psi}\oper{P}\ket{\Psi}}}\,,
    \end{equation*}
    gdzie \(\oper{P} = \oper{P}_A(\lambda_2) - \oper{P}_A(\lambda_1)\).

    \item \textbf{O układach złożonych.} Przestrzeń Hilberta \(\mathscr{H}\) układu złożonego ma
    strukturę iloczynu tensorowego przestrzeni Hilberta układów prostych wchodzących w jego skład
    \(\mathscr{H}=\mathscr{H}_1\otimes\mathscr{H}_2\otimes\ldots\otimes\mathscr{H}_n\).

\end{enumerate}

\subsubsection*{Zasady nieoznaczoności}

Niech \(A\) będzie pewną wielkością fizyczną reprezentowaną przez operator \(\oper{A}\). Zdefiniujmy
odchylenie standardowe \(\sigma_A\) wielkości \(A\) dla układu w stanie \(\Psi\) jako
\begin{equation*}
    \sigma_A := \sqrt{\expval{(A - \expval{A})^2}} = \sqrt{\bra{\Psi}\oper{\mathcal{A}}^2\ket{\Psi}}\,,
\end{equation*}
gdzie \(\oper{\mathcal{A}} := \oper{A} - \expval{A}\). Dla dowolnych dwóch wielkość fizycznych \(A\)
i \(B\) w układzie reprezentowanym przez \(\Psi\) mamy z nierówności Cauchy'ego--Schwarza
\begin{equation*}
    \sigma_A^2\sigma_B^2 \geq \left|\braket{\oper{\mathcal{A}}\Psi}{\oper{\mathcal{B}}\Psi}\right|^2\,.
\end{equation*}
Jednocześnie dla dowolnego \(z = x + \im y \in\mathbb{C}\) mamy
\begin{equation*}
    |z|^2 = x^2 + y^2 = \left(\frac{z + z^*}{2}\right)^2 + \left(\frac{z-z^*}{2\im}\right)^2\,.
\end{equation*}
Z powyższego mamy więc
\begin{equation*}
    \sigma_A^2\sigma_B^2 \geq \left[\frac{1}{2\im}\left(\braket{\oper{\mathcal{A}}\Psi}{\oper{\mathcal{B}}\Psi} - \braket{\oper{\mathcal{B}}\Psi}{\oper{\mathcal{A}}\Psi}\right)\right]^2 + \left[\frac{1}{2}(\braket{\oper{\mathcal{A}}\Psi}{\oper{\mathcal{B}}\Psi} + \braket{\oper{\mathcal{B}}\Psi}{\oper{\mathcal{A}}\Psi})\right]^2\,,
\end{equation*}
skąd po prostych przekształceniach otrzymujemy
\begin{equation*}
    \sigma_A^2\sigma_B^2 \geq \left(\frac{1}{2\im}\bra{\Psi}[\oper{A},\oper{B}]\ket{\Psi}\right)^2 + \left(\frac{1}{2}\bra{\Psi}\{\oper{A},\oper{B}\}\ket{\Psi} - \expval{A}\expval{B}\right)^2
    \quad.
\end{equation*}
Powyższą nierówność nazywamy \textit{zasadą nieoznaczoności Schr\"{o}dingera}. W szczególności z
powyższego wynika \textit{zasada nieoznaczoności Robertsona}
\begin{equation*}
    \sigma_A^2\sigma_B^2 \geq \left(\frac{1}{2\im}\bra{\Psi}[\oper{A},\oper{B}]\ket{\Psi}\right)^2
    \quad,
\end{equation*}
która dla operatorów spełniających \textit{kanoniczną relację komutacyjną} \([\oper{A},\oper{B}] =
\im\hbar\oper{1}\) przyjmuje postać \textit{zasady nieoznaczonosci Heisenberga}
\begin{equation*}
    \sigma_A\sigma_B \geq \frac{\hbar}{2}\quad.
\end{equation*}
Zauważmy jednakże, iż w rozważanych przez nas \(n\)--wymiarowych przestrzeniach Hilberta nie
istnieją operatory spełniające kanoniczną relację komutacyjną. Istotnie biorąc ślad z powyższego
wyrażenia mamy
\begin{equation*}
    0 = \Tr \oper{AB} - \Tr\oper{BA} = \im n\hbar \neq 0\,.
\end{equation*}


\subsubsection*{Twierdzenie Ehrenfesta}

Niech \(A\) będzie pewną wielkością fizyczną reprezentowaną przez operator \(\oper{A}\), wówczas
\begin{equation*}
        \dv{\expectationvalue{A}}{t} = \dv{}{t} \braket{\Psi}{\oper{A}\Psi}
        = \braket{\pdv{\Psi}{t}}{\oper{A}\Psi} + \braket{\Psi}{\oper{A}\pdv{\Psi}{t}} + \braket{\Psi}{\pdv{\oper{A}}{t}\Psi}\,.
\end{equation*}
Jednocześnie z równania Schr\"{o}dingera mamy \(\oper{H}\Psi=\im\hbar{\partial_t\Psi}\), skąd
\begin{equation*}
    \dv{\expectationvalue{A}}{t} = \frac{\im}{\hbar}\braket{\oper{H}\Psi}{\oper{A}\Psi} - \frac{\im}{\hbar}\braket{\Psi}{\oper{A}\oper{H}\Psi} + \bra{\Psi}\pdv{\oper{A}}{t}\ket{\Psi}\,,
\end{equation*}
ale ze względu na fakt, iż \(\oper{H}\) jest operatorem samosprzężonym mamy
\begin{equation*}
    \dv{\expectationvalue{A}}{t} = \bra{\Psi}\pdv{\oper{A}}{t}\ket{\Psi} + \frac{\im}{\hbar}\bra{\Psi}[\oper{H},\oper{A}]\ket{\Psi}
    \quad.
\end{equation*}
Powyższe równanie nazywamy \textit{twierdzeniem Ehrenfesta}.

\subsubsection*{Zasada nieoznaczoności energii--czasu}

Z twierdzenia Ehrenfesta dla operatora \(\oper{A}\) niezależnego od czasu mamy
\begin{equation*}
    \dv{\expval{A}}{t} = \frac{\im}{\hbar}\bra{\Psi}[\oper{H}, \oper{A}]\ket{\Psi}\,.
\end{equation*}
Jednocześnie z zasady nieoznaczoności Robertsona mamy
\begin{equation*}
    \sigma_H^2\sigma_A^2 \geq \left(\frac{1}{2\im}\bra{\Psi}[\oper{H}, \oper{A}]\ket{\Psi}\right)^2\,.
\end{equation*}
Podstawiając wyrażenie na wartość oczekiwaną komutatora otrzymujemy z powyższego
\begin{equation*}
    \sigma_H\sigma_A \geq \frac{\hbar}{2} \left|\dv{\expval{A}}{t}\right|\,.
\end{equation*}
Jeśli wartość oczekiwana nie zmienia się w czasie to powyższa nierówność nie mówi nic odkrywczego.
Jeśli jednak rozważymy pewien proces, podczas którego wartość oczekiwana zmienia się w czasie
\(\Delta t\) o \(\Delta A\) to możemy dokonać przybliżenia 
\begin{equation*}
    \frac{\sigma_A}{\left|\dv{\expval{A}}{t}\right|} \approx \frac{|\Delta A|}{|\Delta A| / \Delta t} = \Delta t\,,
\end{equation*}
a wówczas z powyższej nierówności otrzymujemy zależność
\begin{equation*}
    \sigma_H\Delta t \gtrsim \frac{\hbar}{2}\,,
\end{equation*}
którą nazywamy \textit{zasadą nieoznaczoności energii--czasu}.


\subsection{Kwantowe układy dwupoziomowe}

Przedstawimy teraz ważną realizację abstrakcyjnej teorii kwantów -- teorię układów dwupoziomowych,
które stanowią podstawę teorii obliczeń kwantowych i kwantowej teorii informacji. Modelem
matematycznym tej teorii jest skończenie wymiarowa przestrzeń Hilberta
\begin{equation*}
    \mathscr{H} = ((\mathbb{C}^2, \mathbb{C}, +, \cdot), \braket{\cdot})
\end{equation*}
i teoria operatorów liniowych w tej przestrzeni, które możemy utożsamiać z zespolonymi macierzami
\(2\times 2\).

\subsubsection{Macierze Pauliego}

Macierze Pauliego definiujemy jako zespolone macierze \(2\times 2\)
\begin{equation*}
    \oper{X} := \mqty[0 & 1 \\ 1 & 0]\,,\quad \oper{Y} := \mqty[0 & -\im \\ \im & 0]\,,\quad \oper{Z} := \mqty[1 & 0 \\ 0 & -1]\,.
\end{equation*}
Przydatne jest zdefiniowanie \textit{wektora macierzy Pauliego} \(\boldsymbol{\sigma}\)
\begin{equation*}
    \boldsymbol{\sigma} = (\oper{X}, \oper{Y}, \oper{Z})\,,
\end{equation*}
dzięki któremu możemy łatwo zapisać sumę przeskalowanych macierzy Pauliego jako
\(\mathbf{c}\cdot\boldsymbol{\sigma}\), gdzie \(\mathbf{c}\in\mathbb{C}^3\) jest pewnym wektorem o
elementach zespolonych.

\renewcommand{\arraystretch}{2}
\begin{table}[ht]
    \centering
    \begin{tabular}{|ccc|}
    \hline
    \multicolumn{1}{|c|}{\(\oper{XY} = \im\oper{Z} = -\oper{YX}\)}            &
    \multicolumn{1}{c|}{\([\oper{X},\oper{Y}] = 2\im\oper{Z}\)}          & \(\{\oper{X}, \oper{Y}\}
    = \oper{\mathsf{0}}\)      \\ \hline
    \multicolumn{1}{|c|}{\(\oper{YZ} = \im\oper{X} = -\oper{ZY}\)}            &
    \multicolumn{1}{c|}{\([\oper{Y},\oper{Z}] = 2\im\oper{X}\)}          & \(\{\oper{Y}, \oper{Z}\}
    = \oper{\mathsf{0}}\)      \\ \hline
    \multicolumn{1}{|c|}{\(\oper{ZX} = \im\oper{Y} = -\oper{XZ}\)}            &
    \multicolumn{1}{c|}{\([\oper{Z},\oper{X}] = 2\im\oper{Y}\)}          & \(\{\oper{Z}, \oper{X}\}
    = \oper{\mathsf{0}}\)      \\ \hline
    \multicolumn{1}{|c|}{\(\oper{X}^2 = \oper{Y}^2 = \oper{Z}^2 = \oper{1}\)} &
    \multicolumn{1}{c|}{\(\Tr\oper{X} = \Tr\oper{Y} = \Tr\oper{Z} = 0\)} & \(\det\oper{X} =
    \det\oper{Y} = \det\oper{Z} = -1\) \\ \hline
    \multicolumn{3}{|c|}{\(\e^{\im\mathbf{a}\cdot\oper{\sigma}} = \oper{1}\cos|\mathbf{a}| + \im
    \mathbf{\hat{a}}\cdot\oper{\sigma}\sin|\mathbf{a}|\)} \\ \hline
    \multicolumn{3}{|c|}{\((\mathbf{a}\cdot\oper{\sigma})\cdot(\mathbf{b}\cdot\oper{\sigma}) =
    (\mathbf{a}\cdot\mathbf{b})\oper{1} + \im(\mathbf{a}\times\mathbf{b})\cdot\oper{\sigma}\)} \\
    \hline
    \end{tabular}
    \caption{Wybrane własności macierzy Pauliego}
\end{table}
\renewcommand{\arraystretch}{1}

Zauważmy, że dowolny operator samosprzężony \(\oper{A}\) w rozpatrywanej przestrzeni \(\mathscr{H}\)
ma postać
\begin{equation*}
    \oper{A} = \mqty[a_0 + a_z & a_x - \im a_y \\ a_x + \im a_y & a_0 - a_z] = a_0\oper{1} + \mathbf{a}\cdot\boldsymbol{\sigma}\,, 
\end{equation*}
gdzie \(a_0,a_x,a_y,a_z \in \mathbb{R}\). Jednocześnie bez straty ogólności możemy przyjąć \(a_0 =
0\), gdyż stała ta przesuwa jedynie widmo operatora \(\oper{A}\) o ustaloną wartość, co pozwala
przedstawić dowolny operator samosprzężony jako wektor \(\mathbf{a}\) w trójwymiarowej przestrzeni.

\subsubsection{Sfera Blocha}

W trójwymiarowej przestrzeni możemy również przedstawić wektor stanu \(\Psi\). Istotnie wektor stanu
jest określony przez 2 zmienne zespolone
\begin{equation*}
    \ket{\Psi} = \mqty[\alpha \\ \beta]\,,
\end{equation*}
czyli 4 zmienne rzeczywiste, ale ze względu na warunek unormowania \(|\alpha|^2 + |\beta|^2 = 1\)
mamy tylko 3 zmienne niezależne. Dodatkowo pamiętając, iż globalna faza wektora stanu nie ma
znaczenia możemy wyeliminować jeszcze jedną zmienną i zapisać wektor \(\Psi\) jako
\begin{equation*}
    \ket{\Psi} = \cos\frac{\theta}{2}\mqty[1 \\ 0] + \e^{\im\phi}\sin\frac{\theta}{2}\mqty[0\\1] = \cos\frac{\theta}{2}\ket{0} + \e^{\im\phi}\sin\frac{\theta}{2}\ket{1}
\end{equation*}
dla pewnych parametrów \(\phi,\theta\in\mathbb{R}\). Powyżej wektory bazy ortonormalnej oznaczyliśmy
jako \(\{\ket{0}, \ket{1}\}\) zgodnie z oznaczeniami stosowanymi w teorii obliczeń kwantowych (w
szczególności \(\ket{0}\) \textit{nie oznacza} w powyższej notacji wektora zerowego \(\mathsf{0}\)).
Zmienne \(\phi\), \(\theta\) możemy interpretować odpowiednio jako kąt azymutalny i kąt zenitalny
punktu na sferze jednostkowej, którą nazywamy \textit{sferą Blocha}. Zauważmy, iż przy wybranej
parametryzacji bieguny sfery określają odpowiednio stany \(\ket{0}\) i \(\ket{1}\).

\begin{figure}[ht]
    \centering
    \includegraphics[width=0.45\columnwidth]{figs/Bloch_sphere.png}
    \label{fig:bloch_sphere}
    \caption{Sfera Blocha}
\end{figure}

Możemy połączyć oba przedstawienia tj. przedstawienia operatora i wektora stanu jeśli tylko zamiast
wektora stanu użyjemy operatora rzutowania na stan \(\Psi\). Możemy rozłożyć go wówczas (jak każdy
operator samosprzężony) na macierze Pauliego
\begin{equation*}
    \ket{\Psi}\bra{\Psi} = s_0\oper{1} + \mathbf{s}\cdot\boldsymbol{\sigma}\,,
\end{equation*}
przy czym parametry \(s_0,s_x,s_y,s_z\) muszą spełniać
\begin{equation*}
    \ket{\Psi}\bra{\Psi}\ket{\Psi}\bra{\Psi} = \ket{\Psi}\bra{\Psi}\,,
\end{equation*}
czyli
\begin{equation*}
    s_0\oper{1} + \mathbf{s}\cdot\boldsymbol{\sigma} = (s_0^2 + |\mathbf{s}|^2)\oper{1} + 2s_0\mathbf{s}\cdot\boldsymbol{\sigma}\,,
\end{equation*}
skąd \(s_0 = |\mathbf{s}| = 1/2\). Operator rzutowania \(\ket{\Psi}\bra{\Psi}\) możemy zatem w
ogólności rozłożyć na macierze Pauliego w następujący sposób
\begin{equation*}
    \ket{\Psi}\bra{\Psi} = \frac{1}{2}(\oper{1} + \mathbf{s}\cdot\boldsymbol{\sigma})\,,
\end{equation*}
gdzie przeskalowaliśmy zmienne \(s_x,s_y,s_z\) tak, że teraz \(|\mathbf{s}| = 1\).

Widzimy więc, iż stan \(\Psi\) możemy reprezentować jako wektor \(\mathbf{s}\) określający punkt na
sferze jednostkowej w trójwymiarowej przestrzeni, a operator samosprzężony jako dowolny wektor
\(\mathbf{a}\) w tej przestrzeni. Przejście od rzeczywistego wektora \(\mathbf{s}\) do
abstrakcyjnego wektora stanu \(\ket{\Psi}\) odbywa sie poprzez określenie współrzędnych sferycznych
\((\theta,\phi)\) wektora \(\mathbf{s}\) i zmapowanie ich zgodnie z przepisem
\begin{equation*}
    \ket{\Psi} = \cos\frac{\theta}{2}\ket{0} + \e^{\im\phi}\sin\frac{\theta}{2}\ket{1}\,.
\end{equation*}


\subsubsection{Ewolucja układu dwupoziomowego}

Dla układów dwupoziomowych ogólne równania ewolucji amplitud prawdopodobieństwa wektora stanu
\begin{equation*}
    \ket{\Psi} = \mqty[\alpha \\ \beta]
\end{equation*}
dla hamiltonianu postaci
\begin{equation*}
    \oper{H} = \mqty[H_{11}(t) & H_{12}(t) \\ H_{21}(t) & H_{22}(t)]
\end{equation*}
mają postać
\begin{equation*}
    \begin{cases}
        \im\hbar\dot{\alpha} = H_{11}(t)\alpha + H_{12}(t)\beta\\
        \im\hbar\dot{\beta}  = H_{21}(t)\alpha + H_{22}(t)\beta
    \end{cases}
\end{equation*}
Równanie ewolucji możemy zapisać również wykorzystując przedstawienie geometryczne wektorów stanu i
operatorów na sferze Blocha. Istotnie różniczkując operator rzutowy \(\oper{\rho} =
\ket{\Psi}\bra{\Psi}\) mamy
\begin{equation*}
    \begin{split}
        \pdv{\oper{\rho}}{t} &= \ket{\dot{\Psi}}\bra{\Psi} + \ket{\Psi}\bra{\dot{\Psi}} = \frac{1}{\im\hbar}\ket{\oper{H}\Psi}\bra{\Psi} - \frac{1}{\im\hbar}\ket{\Psi}\bra{\oper{H}\Psi}\\
        &=\frac{1}{\im\hbar}(\oper{H}\oper{\rho}-\oper{\rho}\oper{H}) = \frac{1}{\im\hbar}[\oper{H},\oper{\rho}]\,.
    \end{split}
\end{equation*}

Z powyższego mamy więc dla \(\oper{\rho} = \frac{1}{2}(\oper{1} +
\mathbf{s}\cdot\boldsymbol{\sigma})\) i \(\oper{H} = \mathbf{H}\cdot\boldsymbol{\sigma}\)
\begin{equation*}
    \frac{\hbar}{2}\dot{\mathbf{s}} = \mathbf{H}\times\mathbf{s}
\end{equation*}

Przedstawienie geometryczne wektora stanu na sferze Blocha nie jest jedynie obserwacją matematyczną.
Pozwala ono zwizualizować semi-klasyczną ewolucję wektorowej wielkości fizycznej \(\mathbf{S}\),
której składowe \(S_x\), \(S_y\), \(S_z\) są w przestrzeni \(\mathscr{H}\) reprezentowane przez
samosprzężone operatory Pauliego \(\oper{X}\), \(\oper{Y}\), \(\oper{Z}\). Istotnie zgodnie z
twierdzeniem Ehrenfesta
\begin{equation*}
    \dv{\expval{\mathbf{S}}}{t} = \frac{\im}{\hbar}\bra{\Psi}[\oper{H}, \boldsymbol{\sigma}]\ket{\Psi} = \frac{\im}{\hbar}\bra{\Psi}\left([\oper{H},\oper{X}],[\oper{H},\oper{Y}],[\oper{H},\oper{Z}]\right)\ket{\Psi}\,.
\end{equation*}
Jednocześnie dla \(\oper{H} = \mathbf{H}\cdot\boldsymbol{\sigma}\)
\begin{equation*}
    \begin{split}
        &[\oper{H},\oper{X}] = -2\im H_y\oper{Z} + 2\im H_z \oper{Y}\\
        &[\oper{H},\oper{Y}] = -2\im H_z\oper{X} + 2\im H_x \oper{Z}\\
        &[\oper{H},\oper{Z}] = -2\im H_x\oper{Y} + 2\im H_y \oper{X}
    \end{split}\quad,
\end{equation*}
skąd
\begin{equation*}
    \frac{\hbar}{2}\dv{\expval{\mathbf{S}}}{t} = \mathbf{H} \times \expval{\mathbf{S}}\,.
\end{equation*}
Widzimy zatem, iż ruch wektora \(\mathbf{s}\) po sferze Blocha odpowiada ewolucji wartości
oczekiwanej wielkości fizycznej \(\mathbf{S}\), którą to ewolucję możemy w przybliżeniu
semi-klasycznym traktować jako faktyczną ewolucję tych wielkości.

\subsubsection{Obroty na sferze Blocha}

Powyższe rozważania pokazują, iż w przedstawieniu geometrycznym stan \(\Psi\) reprezentowany przez
jednostkowy wektor \(\mathbf{s}\) na sferze Blocha ewoluuje w taki sposób, iż efektywnie jego
położenie na sferze Blocha w czasie \(t\) można przedstawić jako obrót wektora położenia w czasie
\(t_0\) o pewien kąt \(\varphi\) wokół ustalonej osi \(\mathbf{\hat{n}}\)
\begin{equation*}
    \mathbf{s}(t) = \overleftrightarrow{\mathbf{R}}_\mathbf{\hat{n}}(\varphi)\mathbf{s}(t_0)\,.
\end{equation*}
W ujęciu abstrakcyjnego wektora \(\ket{\Psi}\) przekształcenie to odpowiada oczywiście pewnemu
przekształceniu unitarnemu \(\oper{U}\), tj.
\begin{equation*}
    \ket{\Psi(t)} = \oper{U}\ket{\Psi(t_0)}\,.
\end{equation*}
Warte zbadania wydaje się więc wyznaczenie operatora \(\oper{U}\), który odpowiada obrotowi na
sferze Blocha. Aby wyznaczyć jawny wzór na operator \(\oper{U}\) rozważmy infinitezymalny obrót
wektora \(\mathbf{s}\) wokół osi \(\mathbf{\hat{n}}\) o kąt \(\epsilon\). Jak łatwo pokazać
\begin{equation*}
    \mathbf{s}' = \overleftrightarrow{\mathbf{R}}_\mathbf{\hat{n}}(\epsilon)\mathbf{s} = \mathbf{s} + \epsilon(\mathbf{s}\times\mathbf{\hat{n}})
\end{equation*}
w przybliżeniu do wyrazów liniowych względem \(\epsilon\). Wektor rzeczywisty \(\mathbf{s}\)
najłatwiej powiązać z abstrakcyjnym wektorem \(\ket{\Psi}\) poprzez operator rzutowy \(\oper{\rho} =
\ket{\Psi}\bra{\Psi}\) dla którego zachodzi
\begin{equation*}
    \ket{\oper{U}\Psi}\bra{\oper{U}\Psi} = \oper{U}\oper{\rho}\oper{U}^\dagger = \frac{1}{2}\left(\oper{1} + \mathbf{s}'\cdot\boldsymbol{\sigma}\right)\,,
\end{equation*}
skąd
\begin{equation*}
    \mathbf{s}\cdot\oper{U}\boldsymbol{\sigma}\oper{U}^\dagger = (\mathbf{s} + \epsilon(\mathbf{s}\times\mathbf{\hat{n}}))\cdot\boldsymbol{\sigma} = \mathbf{s}\cdot(\boldsymbol{\sigma} + \epsilon(\mathbf{\hat{n}}\times\boldsymbol{\sigma}))\,.
\end{equation*}
Otrzymujemy zatem równanie
\begin{equation*}
    \oper{U}\boldsymbol{\sigma}\oper{U}^\dagger = \boldsymbol{\sigma} + \epsilon(\mathbf{\hat{n}}\times\boldsymbol{\sigma})\,.
\end{equation*}
Poszukajmy \(\oper{U}\) spełniających to równanie postaci
\begin{equation*}
    \oper{U} = \oper{1} + \im\epsilon\oper{A}\,,\quad \oper{U}^\dagger = \oper{1} - \im\epsilon\oper{A}^\dagger\,,
\end{equation*}
gdzie \(\oper{A}\) jest operatorem samosprzężonym postaci \(\oper{A} =
\mathbf{a}\cdot\boldsymbol{\sigma}\). Zauważmy, iż tak zdefiniowany \(\oper{U}\) jest unitarny w
przybliżeniu do wyrazów liniowych względem \(\epsilon\). Podstawiając powyższe wzory na \(\oper{U}\)
oraz \(\oper{U}^\dagger\) i ograniczając się do wyrazów liniowych względem \(\epsilon\) otrzymujemy
\begin{equation*}
    \im\epsilon[\oper{A}, \boldsymbol{\sigma}] = \epsilon(\mathbf{\hat{n}}\times\boldsymbol{\sigma}).
\end{equation*}
Widzimy zatem, iż taki, a nie inny strzał na postać operatora \(\oper{U}\) był podyktowany
wyprowadzonymi wcześniej wzorami na komutatory operatora samosprzężonego z operatorami Pauliego,
które naśladują strukturę zwykłego trójwymiarowego iloczynu wektorowego. Z powyższego otrzymujemy
zatem \(\mathbf{a} = \frac{1}{2}\mathbf{\hat{n}}\), skąd
\begin{equation*}
    \oper{U} = \oper{1} + \frac{\im \epsilon}{2} \mathbf{\hat{n}} \cdot \boldsymbol{\sigma}\,.
\end{equation*}
Stąd łatwo możemy już uzyskać operator unitarny \(\oper{U}_\mathbf{\hat{n}}(\varphi)\) odpowiadający
obrotowi o kąt \(\varphi\) wokół osi \(\mathbf{\hat{n}}\)
\begin{equation*}
    \oper{U}_\mathbf{\hat{n}}(\varphi) = \lim_{N\to\infty}\left(\oper{1} + \frac{\im \varphi}{2N} \mathbf{\hat{n}} \cdot \boldsymbol{\sigma}\right)^N = \exp{\frac{\im}{2}\varphi\mathbf{\hat{n}}\cdot\boldsymbol{\sigma}} = \oper{1}\cos\frac{\varphi}{2} + \im(\mathbf{\hat{n}}\cdot\boldsymbol{\sigma})\sin\frac{\varphi}{2}
\end{equation*}
W szczególności dla obrotów wokół osi \(x\), \(y\), \(z\) mamy odpowiednio
\begin{equation*}
    \begin{split}
        &\oper{U}_\mathbf{\hat{x}}(\varphi) = \mqty[\cos\frac{\varphi}{2} & \im\sin\frac{\varphi}{2} \\ \im\sin\frac{\varphi}{2} & \cos\frac{\varphi}{2}]\\
        &\oper{U}_\mathbf{\hat{y}}(\varphi) = \mqty[\cos\frac{\varphi}{2} & \sin\frac{\varphi}{2} \\ -\sin\frac{\varphi}{2} & \cos\frac{\varphi}{2}]\\
        &\oper{U}_\mathbf{\hat{z}}(\varphi) = \mqty[\e^{+\im\varphi/2} & 0 \\ 0 & \e^{-\im\varphi/2}]
    \end{split}\quad.
\end{equation*}

\subsubsection{Przykład -- magnetyczny rezonans jądrowy }
Rozpatrzmy cząstkę obdarzoną momentem magnetycznym \(\mu\) o spinie połówkowym i nie posiadającą
ładunku elektrycznego (neutron), która została umieszczona w wirującym z częstością radiową polu
magnetycznym
\begin{equation*}
    \mathbf{B}(t) = B_\text{rf}\cos(\omega t)\mathbf{\hat{x}} - B_\text{rf}\sin(\omega t) \mathbf{\hat{y}} + B_0\mathbf{\hat{z}}\,.
\end{equation*}
Cząstka ta stanowi pewien układ dwupoziomowy, którego oddziaływanie z polem magnetycznym jest
opisane hamiltonianem Pauliego
\begin{equation*}
    \oper{H} = -\mu\mathbf{B}\cdot\boldsymbol{\sigma} = -\mu\mqty[B_0 & B_\text{rf}\e^{+\im\omega t} \\ B_\text{rf}\e^{-\im\omega t} & -B_0]\,.
\end{equation*}
Równania ewolucji na amplitudy prawdopodobieństwa mają więc postać
\begin{equation*}
    \begin{cases}
        \frac{1}{\im\omega_0}\dot{\alpha} =n e^{+\im\omega t}\beta + \alpha \\
        \frac{1}{\im\omega_0}\dot{\beta} = n \e^{-\im\omega t}\alpha - \beta
    \end{cases}\quad,
\end{equation*}
gdzie wprowadziliśmy parametry \(\omega_0 := \mu B_0 / \hbar\) i \(n := B_\text{rf}/B_0\). Łatwo
sprawdzić, iż rozwiązaniem powyższego układu równań jest
\begin{equation*}
    \begin{cases}
        \alpha(t) = \left(c_1 \e^{+\im\Omega t} + c_2 \e^{-\im \Omega t}\right) n \e^{+\im\omega t / 2}\\
        \beta(t)  = \left[c_1 (\Omega - \delta)\e^{+\im\Omega t} - c_2 (\Omega + \delta)\e^{-\im\Omega t}\right]\omega_0^{-1}\e^{-\im\omega t / 2} 
    \end{cases}\quad,
\end{equation*}
gdzie
\begin{equation*}
    \delta = \omega_0 - \frac{\omega}{2}\,,\quad \Omega = \sqrt{n^2\omega_0^2 + \delta^2}\,.
\end{equation*}
Zdefiniowaną wielkość \(\Omega\) nazywamy \textit{częstością Rabiego}. Załóżmy, iż w chwili
początkowej \(\alpha = 0\) i \(\beta = 1\), wówczas
\begin{equation*}
\begin{split}
    &\alpha(t) = \frac{\im n\omega_0}{\Omega}\sin(\Omega t)\e^{+\im\omega t/2}\\
    &\beta(t) = \left\{\cos(\Omega t) +\frac{\im}{\Omega}\left(\frac{1}{2}\omega - \omega_0\right)\sin(\Omega t)\right\}\e^{-\im\omega t/2}
\end{split}\quad.
\end{equation*}
Z powyższego prawdopodobieństwo \(p_{1\to0}\) znalezienia układu w stanie \(\ket{0}\) wynosi
\begin{equation*}
        p_{1\to0}(t) = |\alpha|^2 = \left(\frac{n\omega_0}{\Omega}\right)^2\sin^2(\Omega t) = \frac{1}{2}\left(\frac{n\omega_0}{\Omega}\right)^2(1-\cos(2\Omega t))\,.
\end{equation*}
Natomiast prawdopodobieństwo \(p_{1\to1}\) znalezienia układu w stanie \(\ket{1}\) wynosi
\begin{equation*}
        p_{1\to1}(t) = |\beta|^2 =\cos^2(\Omega t) + \left(\frac{\omega_0 - \frac{1}{2}\omega}{\Omega}\right)^2\sin^2(\Omega t) = 1 - p_{1\to 0}(t)\,.
\end{equation*}
Prawdopodobieństwa oscylują w czasie z częstością równą podwojonej częstości Rabiego. Amplituda tych
oscylacji wynosi
\begin{equation*}
    \frac{1}{2}\left(\frac{n\omega_0}{\Omega}\right)^2 = \frac{n^2\omega_0^2}{2}\frac{1}{\left(\omega_0 - \frac{1}{2}\omega\right)^2 +
    \left(n\omega_0\right)^2}
\end{equation*}
i przyjmuje wartość maksymalną, gdy spełniony jest \textit{warunek rezonansu} postaci
\begin{equation*}
    \hbar\omega = 2\mu B_0\,.
\end{equation*}

Zauważmy, iż opisane zjawisko stanowi podstawę do manipulacji qbitami. Istotnie w stanie rezonansu
wektor stanu ewoluuje zgodnie z
\begin{equation*}
    \ket{\Psi(t)} = \alpha(t)\ket{0} + \beta(t)\ket{1} = \im\sin(\Omega t)\e^{+\im\omega_0 t}\ket{0} + \cos(\Omega t)\e^{-\im\omega_0 t}\ket{1}
\end{equation*}
Poprzez dostosowanie czasu \(t\) przez jaki układ dwupoziomowy oddziałuje z wirowym polem możemy
sterować stanem \(\Psi\) w jakim się znajduje, przykładowo jeśli początkowo układ był w stanie
\(\ket{1}\) to włączając wirowe pole magnetyczne na czas \(t = \pi/2\Omega\) (tzw. \textit{impuls
\(\pi\)}, gdyż na sferze Blocha odpowiada mu obrót o \(\pi\)) układ przechodzi do stanu \(\ket{0}\)
(pomijając nieistotny globalny czynnik fazowy), natomiast dla \(t = \pi/4\Omega\) (tzw.
\textit{impuls \(\pi/2\)}) układ przechodzi do stanu
\begin{equation*}
    \frac{1}{\sqrt{2}}\left(\im\e^{+\im\omega_0t}\ket{0} + \e^{-\im\omega_0t}\ket{1}\right)\,,
\end{equation*}
w którym mamy jednakowe prawdopodobieństwa odpowiedzi twierdzących na pytania \(\dyad{0}\) i
\(\dyad{1}\).

Dla atomu z dwoma poziomami energetycznymi poddanemu działaniu wiązki laserowej otrzymuje się
identyczne równanie pod warunkiem przyjęcia przybliżenia tzw. \textit{wirującej fali}, które jest
prawdziwe, gdy długość fali jest znacznie większa od rozmiarów próbki. W tym przypadku
\(2\hbar\omega_0\) jest różnicą energii między dwoma poziomami atomowymi, \(\omega\) jest częstością
laserową, a iloczyn \(\mu B_\text{rf}\) przechodzi na iloczyn elektrycznego momentu dipolowego atomu
i amplitudy pola elektrycznego fali elektromagnetycznej \(p E_0\).

\section{Obliczenia kwantowe}

\subsubsection*{Algorytm RSA}

Bob wybiera dwie liczby pierwsze \(p,q\in\mathbb{P}\), oblicza liczbę \(N = pq\) oraz wybiera liczbę
\(c \perp \phi(N) = (p-1)(q-1)\), gdzie \(\phi\) to funkcja Eulera. Następnie oblicza odwrotność
liczby \(c\) modulo \(\phi(N)\), tj. liczbę \(d\) taką, że \(cd \equiv_{\phi(N)} 1\). Istnienie
liczby \(d\) zapewnia twierdzenie Eulera, istotnie wynika z niego, iż \(d\) można obliczyć jako
\begin{equation*}
    c^{\phi(\phi(N))} \equiv_{\phi(N)} c\cdot c^{\phi(\phi(N))-1} \equiv_{\phi(N)} 1  \implies d \equiv_{\phi(N)} c^{\phi(\phi(N))-1}\,.
\end{equation*}
Bob wysyła następnie Alicji niezabezpieczoną drogą liczby \(c\) i \(N\). 

Załóżmy, że Alicja chce wysłać do Boba zaszyfrowaną wiadomość reprezentowaną przez liczbę \(a < N\).
Aby to zrobić Alicja oblicza liczbę \(b \equiv_N a^c\) i wysyła ją do Boba niezabezpieczoną drogą.

Bob odczytuje wiadomość obliczając
\begin{equation*}
    b^d \equiv_N a^{cd} \equiv_N a\,.
\end{equation*}

Powyższa równość wynika z Małego Twierdzenia Fermata. Istotnie zauważmy wpierw, iż jeśli \(p \perp
q\) to \(x \equiv_p y\) i \(x \equiv_q y\) implikują \(x \equiv_{pq} y\). Istotnie z założeń wynika,
iż \(\exists k,k' \in \mathbb{Z}\) takie, że \(x - y = kp = k'q\), ale ponieważ \(p \perp q\) to \(q
| k\) (i analogicznie \(p | k'\)), więc istnieje \(k'' \in \mathbb{Z}\) takie, że \(x - y = k''
pq\). Do udowodnienia powyższej równość wystarczy zatem pokazać, iż \(a^{cd} \equiv_p a\) i \(a^{cd}
\equiv_q  a\). Pokażemy dowód dla \(p\) (dowód dla \(q\) jest analogiczny). Istotnie z definicji
liczb \(c,d\) mamy \(cd = k(p-1)(q-1) + 1 = l(p-1) + 1\) dla pewnych \(k,l\in\mathbb{Z}\). Z
powyższego mamy więc
\begin{equation*}
    a^{cd} \equiv_p a(a^{p-1})^l\,.
\end{equation*}
Jeśli \(a \perp p\) to z MTF mamy \(a^{p-1} \equiv_p 1\), zatem \(a^{cd} \equiv_p a(1)^l \equiv_p
a\), w przeciwnym wypadku mamy natomiast \(p | a\), zatem \(a^{cd} \equiv_p 0 \equiv_p a\), co
kończy dowód.


\section{Informacja kwantowa}

\subsection{Bazy dopełniające}
\begin{definition}
    Niech \(\{\phi_i\}\), \(\{\theta_i\}\)  będą dwiema bazami ortonormalnymi \(N\)--wymiarowej
    przestrzeni Hilberta \(\mathscr{H}\). Bazy te nazwiemy \textit{dopełniającymi} iff
    \begin{equation*}
        \forall i,j\in\{1,\ldots,N\}\,:\, |\braket{\phi_i}{\theta_j}|^2 = \frac{1}{N}\,.
    \end{equation*}
\end{definition}
Jedną z możliwości skonstruowania bazy dopełniającej bazy \(\{\phi_i\}\) jest wykorzystanie
dyskretnej transformacji Fouriera tj. wektory bazy dopełniającej są dane przez
\begin{equation*}
    \ket{\theta_i} = \frac{1}{\sqrt{N}}\sum_{j=1}^N \e^{2\pi\im ij/N}\ket{\phi_j}\,.
\end{equation*}
Nietrudno zauważyć, iż wówczas \(|\braket{\phi_m}{\theta_n}|^2 = 1/N\) dla dowolnych \(m, n\) oraz
dodatkowo mamy
\begin{equation*}
    \begin{split}
        &\braket{\theta_m}{\theta_n} = \frac{1}{N}\sum_{j=1}^N \e^{2\pi\im(m - n)j/N} \\
        &= \begin{cases}
            1\,,&\text{dla \(m=n\)}\\
            N^{-1}\e^{2\pi\im(m-n)/N}\frac{\e^{2\pi\im(m-n)} - 1}{\e^{2\pi\im(m-n)/N} - 1} = 0\,,&\text{dla \(m\neq n\)}
        \end{cases} = \delta_{mn}\,,
    \end{split}
\end{equation*}
gdzie ostatnia równość wynika z faktu, iż \(m-n\in\mathbb{Z}\), zatem \(e^{2\pi\im(m-n)} =
\cos(2\pi(m-n)) + \im\sin(2\pi(m-n)) = 1\), czyli \(\{\theta_i\}\) tworzą bazę ortonormalną.

Załóżmy, że mamy zespół układów kwantowych przygotowanych w stanie \(\ket{\theta_k}\), gdzie
\(\theta_k\) jest (określonym) jednym z wektorów bazowych ortonormalnej bazy \(\{\theta_i\}\), ale
nie wiemy o jaki stan chodzi. Jeśli dokonamy pomiaru używając bazy \(\{\theta_i\}\) to w 100\%
przypadków otrzymamy wynik \(\ket{\theta_k}\) -- uzyskamy więc maksymalną ,,informację'' o układzie.
Jeśli natomiast pomiaru dokonamy w bazie dopełniającej \(\{\phi_i\}\) to otrzymamy wszystkie możliwe
wyniki \(\{\phi_i\}\), każdy z prawdopodobieństwem \(1/N\) -- uzyskamy więc minimalną ,,informację''
o układzie.





\end{document}
